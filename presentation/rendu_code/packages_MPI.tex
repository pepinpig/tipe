%-------------------
% gestion des pages
%-------------------
\usepackage[left=1.5cm,right=1.5cm,top=1cm,bottom=2cm]{geometry}
\usepackage{pdflscape} %pages horizontales dans le PDF
\usepackage{calc} %pour pouvoir écrire \textwidth-1cm

%--pour les numéro de pages/ pieds de pages
%-- voir mise_en_forme_MPI.tex pour mise en oeuvre
\usepackage{fancyhdr}


%-------------------
% sommaire 
%-------------------
\usepackage{tocloft}%mise en forme de la toc
\usepackage{titlesec}%mise en forme des titres de section
\usepackage[pagebackref,hidelinks]{hyperref}%pour les liens hypertextes


%-------------------
% langue et encodage
%-------------------
\usepackage[utf8]{inputenc}
\usepackage[T1]{fontenc}
\usepackage[frenchb]{babel}


%-------------------
% symboles et mode maths
%-------------------
\usepackage{amsmath,amsfonts,amssymb}
\usepackage{textcomp,lmodern}%pour l'euro
\usepackage{mathrsfs}%lettres style manuscrit italique
\usepackage{stmaryrd}%pour \llbracket et \rrbracket

%-------------------
% code
%-------------------
%%-- pour la commande qui évite l'italique
%%-- voir https://github.com/gpoore/minted/issues/71
\usepackage{etoolbox}
\usepackage{newunicodechar}



%-------------------
% environnement, théorèmes et exercices


%-------------------
% tableaux
%-------------------
\usepackage{array,multirow}

%-------------------

% images
%-------------------
\usepackage{xcolor} %pour redefinir des couleurs
\usepackage{caption}%les personaliser (les centrer par ex)
\usepackage{graphicx} % pour includegraphics

\usepackage{tikz} %graphiques et dessins
\usetikzlibrary{shapes}%pr écrire \node[ellipse] par ex
\usetikzlibrary{positioning} %pr écrire \node[below rigth = 3pt and 5pt]
\usetikzlibrary{decorations.pathreplacing}%pr les accolades
\usetikzlibrary{patterns}%pour les hachures
\usepackage{lastpage}
\usepackage{bookmark}





%-------------------
% divers
%-------------------
\usepackage{cancel}%pour barrer


\usepackage{minted}

%-------------------
\usepackage{lipsum}% for dummy text
%-------------------

