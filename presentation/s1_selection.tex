%%%%%%%%%%%%%%%%%%%%%%%%%%%%%%%%%%%%%%%%%%%%%%%%%
\section[Selection et Appariement]{Selection et Appariement}
%------------------------------------------------
\begin{frame}
\frametitle{Titre d'une slide avant la sous-section\esp}
	Ici on n'a pas encore de titre de sous-section dans le badeau du haut.
	\label{slides_hors_subsec}
\end{frame}


%++++++++++++++++++++++++++++++++++++++++++++++++
\subsection{Selection}
%++++++++++++++++++++++++++++++++++++++++++++++++
%------------------------------------------------
\begin{frame}
\frametitle{Algorithme type \lin{Moravec}}
\[
\mathrm{Var}_{(dx, dy)}(x, y) = \frac{1}{N} \sum_{i=-w}^{w} I(x + i \cdot dx,\ y + i \cdot dy)^2 - \left( \frac{1}{N} \sum_{i=-w}^{w} I(x + i \cdot dx,\ y + i \cdot dy) \right)^2
\]

où :
\begin{itemize}
  \item $I(x + i \cdot dx,\ y + i \cdot dy)$ est l’intensité du $i$eme pixel dans la direction $(dx, dy)$,
  \item $N$ est le nombre de pixels valides (dans l’image) dans la fenêtre centrée en $(x, y)$,
  \item $w$ est le demi-rayon de la fenêtre .
\end{itemize}
\end{frame}
%------------------------------------------------
\begin{frame}
\frametitle{Algorithme type \lin{Moravec}}

\vspace{0.5em}
Le score du pixel : \textbf{minimum des variances dans 4 directions} :
\[
\text{score}(x, y) = \min \left\{ \mathrm{Var}_{(0,1)},\ \mathrm{Var}_{(1,0)},\ \mathrm{Var}_{(1,1)},\ \mathrm{Var}_{(1,-1)} \right\}
\]

Un pixel est considéré comme un \textbf{point d’intérêt} si :
\[
\text{score}(x, y) > T
\]
avec $T$ un seuil fixé.
\end{frame}


\begin{frame}{Animation stable et centrée}
  \begin{center}
    \begin{tikzpicture}[remember picture, scale=1]
      \only<1>{

\tikzset{every picture/.style={line width=0.75pt}} %set default line width to 0.75pt        

%uncomment if require: \path (0,202); %set diagram left start at 0, and has height of 202

%Shape: Rectangle [id:dp8976716894219737] 
\draw  [color={rgb, 255:red, 0; green, 0; blue, 0 }  ,draw opacity=1 ][fill={rgb, 255:red, 200; green, 43; blue, 43 }  ,fill opacity=0.23 ] (279.6,109) -- (299.6,109) -- (299.6,129) -- (279.6,129) -- cycle ;
%Shape: Rectangle [id:dp9242842609380733] 
\draw  [color={rgb, 255:red, 0; green, 0; blue, 0 }  ,draw opacity=1 ][fill={rgb, 255:red, 200; green, 43; blue, 43 }  ,fill opacity=0.64 ] (299.6,129) -- (319.6,129) -- (319.6,149) -- (299.6,149) -- cycle ;
%Shape: Rectangle [id:dp20566257715084624] 
\draw  [color={rgb, 255:red, 0; green, 0; blue, 0 }  ,draw opacity=1 ][fill={rgb, 255:red, 200; green, 43; blue, 43 }  ,fill opacity=0.64 ] (279.6,129) -- (299.6,129) -- (299.6,149) -- (279.6,149) -- cycle ;
%Shape: Rectangle [id:dp30919519111032157] 
\draw  [color={rgb, 255:red, 0; green, 0; blue, 0 }  ,draw opacity=1 ][fill={rgb, 255:red, 200; green, 43; blue, 43 }  ,fill opacity=0.69 ] (299.6,149) -- (319.6,149) -- (319.6,169) -- (299.6,169) -- cycle ;
%Shape: Rectangle [id:dp3685950018453733] 
\draw  [color={rgb, 255:red, 0; green, 0; blue, 0 }  ,draw opacity=1 ][fill={rgb, 255:red, 200; green, 43; blue, 43 }  ,fill opacity=0.68 ] (319.6,129) -- (339.6,129) -- (339.6,149) -- (319.6,149) -- cycle ;
%Shape: Rectangle [id:dp8054725442861926] 
\draw  [color={rgb, 255:red, 0; green, 0; blue, 0 }  ,draw opacity=1 ][fill={rgb, 255:red, 200; green, 43; blue, 43 }  ,fill opacity=0.32 ] (299.6,89) -- (319.6,89) -- (319.6,109) -- (299.6,109) -- cycle ;
%Shape: Rectangle [id:dp3185720649820162] 
\draw  [color={rgb, 255:red, 0; green, 0; blue, 0 }  ,draw opacity=1 ][fill={rgb, 255:red, 200; green, 43; blue, 43 }  ,fill opacity=0.64 ] (299.6,109) -- (319.6,109) -- (319.6,129) -- (299.6,129) -- cycle ;
%Shape: Rectangle [id:dp07104960494410428] 
\draw  [color={rgb, 255:red, 0; green, 0; blue, 0 }  ,draw opacity=1 ][fill={rgb, 255:red, 200; green, 43; blue, 43 }  ,fill opacity=0.23 ] (239.6,69) -- (259.6,69) -- (259.6,89) -- (239.6,89) -- cycle ;
%Shape: Rectangle [id:dp5441547618011067] 
\draw  [color={rgb, 255:red, 0; green, 0; blue, 0 }  ,draw opacity=1 ][fill={rgb, 255:red, 200; green, 43; blue, 43 }  ,fill opacity=0.21 ] (259.6,89) -- (279.6,89) -- (279.6,109) -- (259.6,109) -- cycle ;
%Shape: Rectangle [id:dp18624620459313024] 
\draw  [color={rgb, 255:red, 0; green, 0; blue, 0 }  ,draw opacity=1 ][fill={rgb, 255:red, 200; green, 43; blue, 43 }  ,fill opacity=0.25 ] (239.6,89) -- (259.6,89) -- (259.6,109) -- (239.6,109) -- cycle ;
%Shape: Rectangle [id:dp5364348089877012] 
\draw  [color={rgb, 255:red, 0; green, 0; blue, 0 }  ,draw opacity=1 ][fill={rgb, 255:red, 200; green, 43; blue, 43 }  ,fill opacity=0.28 ] (239.6,109) -- (259.6,109) -- (259.6,129) -- (239.6,129) -- cycle ;
%Shape: Rectangle [id:dp9541604569667589] 
\draw  [color={rgb, 255:red, 0; green, 0; blue, 0 }  ,draw opacity=1 ][fill={rgb, 255:red, 200; green, 43; blue, 43 }  ,fill opacity=0.66 ] (259.6,129) -- (279.6,129) -- (279.6,149) -- (259.6,149) -- cycle ;
%Shape: Rectangle [id:dp11982540210791848] 
\draw  [color={rgb, 255:red, 0; green, 0; blue, 0 }  ,draw opacity=1 ][fill={rgb, 255:red, 200; green, 43; blue, 43 }  ,fill opacity=0.77 ] (279.6,149) -- (299.6,149) -- (299.6,169) -- (279.6,169) -- cycle ;
%Shape: Rectangle [id:dp12026638886160845] 
\draw  [color={rgb, 255:red, 0; green, 0; blue, 0 }  ,draw opacity=1 ][fill={rgb, 255:red, 200; green, 43; blue, 43 }  ,fill opacity=0.24 ] (259.6,109) -- (279.6,109) -- (279.6,129) -- (259.6,129) -- cycle ;
%Shape: Rectangle [id:dp16801008580529164] 
\draw  [color={rgb, 255:red, 0; green, 0; blue, 0 }  ,draw opacity=1 ][fill={rgb, 255:red, 200; green, 43; blue, 43 }  ,fill opacity=0.26 ] (259.6,69) -- (279.6,69) -- (279.6,89) -- (259.6,89) -- cycle ;
%Shape: Rectangle [id:dp4471922901641272] 
\draw  [color={rgb, 255:red, 0; green, 0; blue, 0 }  ,draw opacity=1 ][fill={rgb, 255:red, 200; green, 43; blue, 43 }  ,fill opacity=0.22 ] (279.6,89) -- (299.6,89) -- (299.6,109) -- (279.6,109) -- cycle ;
%Shape: Rectangle [id:dp5193535071326038] 
\draw  [color={rgb, 255:red, 0; green, 0; blue, 0 }  ,draw opacity=1 ][fill={rgb, 255:red, 200; green, 43; blue, 43 }  ,fill opacity=0.29 ] (279.6,69) -- (299.6,69) -- (299.6,89) -- (279.6,89) -- cycle ;
%Shape: Rectangle [id:dp24429375415865406] 
\draw  [color={rgb, 255:red, 0; green, 0; blue, 0 }  ,draw opacity=1 ][fill={rgb, 255:red, 200; green, 43; blue, 43 }  ,fill opacity=0.28 ] (299.6,69) -- (319.6,69) -- (319.6,89) -- (299.6,89) -- cycle ;
%Shape: Rectangle [id:dp7327922213381906] 
\draw  [color={rgb, 255:red, 0; green, 0; blue, 0 }  ,draw opacity=1 ][fill={rgb, 255:red, 200; green, 43; blue, 43 }  ,fill opacity=0.65 ] (319.6,109) -- (339.6,109) -- (339.6,129) -- (319.6,129) -- cycle ;
%Shape: Rectangle [id:dp45563998070420675] 
\draw  [color={rgb, 255:red, 0; green, 0; blue, 0 }  ,draw opacity=1 ][fill={rgb, 255:red, 200; green, 43; blue, 43 }  ,fill opacity=0.64 ] (239.6,129) -- (259.6,129) -- (259.6,149) -- (239.6,149) -- cycle ;
%Shape: Rectangle [id:dp4081630774808085] 
\draw  [color={rgb, 255:red, 0; green, 0; blue, 0 }  ,draw opacity=1 ][fill={rgb, 255:red, 200; green, 43; blue, 43 }  ,fill opacity=0.73 ] (239.6,149) -- (259.6,149) -- (259.6,169) -- (239.6,169) -- cycle ;
%Shape: Rectangle [id:dp2684884892783119] 
\draw  [color={rgb, 255:red, 0; green, 0; blue, 0 }  ,draw opacity=1 ][fill={rgb, 255:red, 200; green, 43; blue, 43 }  ,fill opacity=0.74 ] (259.6,149) -- (279.6,149) -- (279.6,169) -- (259.6,169) -- cycle ;
%Shape: Rectangle [id:dp3091667704273665] 
\draw  [color={rgb, 255:red, 0; green, 0; blue, 0 }  ,draw opacity=1 ][fill={rgb, 255:red, 200; green, 43; blue, 43 }  ,fill opacity=0.68 ] (319.6,149) -- (339.6,149) -- (339.6,169) -- (319.6,169) -- cycle ;
%Shape: Rectangle [id:dp7449273860445551] 
\draw  [color={rgb, 255:red, 0; green, 0; blue, 0 }  ,draw opacity=1 ][fill={rgb, 255:red, 200; green, 43; blue, 43 }  ,fill opacity=0.7 ] (319.6,69) -- (339.6,69) -- (339.6,89) -- (319.6,89) -- cycle ;
%Shape: Rectangle [id:dp028827611328890224] 
\draw  [color={rgb, 255:red, 0; green, 0; blue, 0 }  ,draw opacity=1 ][fill={rgb, 255:red, 200; green, 43; blue, 43 }  ,fill opacity=0.63 ] (319.6,89) -- (339.6,89) -- (339.6,109) -- (319.6,109) -- cycle ;
%Shape: Rectangle [id:dp15038609219163346] 
\draw  [color={rgb, 255:red, 30; green, 14; blue, 222 }  ,draw opacity=1 ][fill={rgb, 255:red, 192; green, 177; blue, 255 }  ,fill opacity=0.34 ] (279.6,109) -- (299.6,109) -- (299.6,129) -- (279.6,129) -- cycle ;
%Shape: Rectangle [id:dp9252244711607317] 
\draw  [color={rgb, 255:red, 255; green, 255; blue, 255 }  ,draw opacity=1 ] (209,54.5) -- (451,54.5) -- (451,190.5) -- (209,190.5) -- cycle ;

% Text Node
\draw (352.6,70) node [anchor=north west][inner sep=0.75pt]  [font=\tiny,color={rgb, 255:red, 72; green, 36; blue, 227 }  ,opacity=1 ] [align=left] {pixel considéré};
% Text Node
\draw (304.8,136) node [anchor=north west][inner sep=0.75pt]  [font=\tiny] [align=left] {64};
% Text Node
\draw (324.8,155.2) node [anchor=north west][inner sep=0.75pt]  [font=\tiny] [align=left] {69};
% Text Node
\draw (305.2,115.6) node [anchor=north west][inner sep=0.75pt]  [font=\tiny] [align=left] {65};
% Text Node
\draw (305.2,95.6) node [anchor=north west][inner sep=0.75pt]  [font=\tiny] [align=left] {32};
% Text Node
\draw (285.2,95.6) node [anchor=north west][inner sep=0.75pt]  [font=\tiny] [align=left] {22};
% Text Node
\draw (325.2,115.6) node [anchor=north west][inner sep=0.75pt]  [font=\tiny] [align=left] {65};
% Text Node
\draw (244.4,76) node [anchor=north west][inner sep=0.75pt]  [font=\tiny] [align=left] {23};
% Text Node
\draw (265.2,96) node [anchor=north west][inner sep=0.75pt]  [font=\tiny] [align=left] {21};
% Text Node
\draw (244.8,95.6) node [anchor=north west][inner sep=0.75pt]  [font=\tiny] [align=left] {25};
% Text Node
\draw (264.8,116.4) node [anchor=north west][inner sep=0.75pt]  [font=\tiny] [align=left] {24};
% Text Node
\draw (284.8,135.6) node [anchor=north west][inner sep=0.75pt]  [font=\tiny] [align=left] {64};
% Text Node
\draw (304.8,155.2) node [anchor=north west][inner sep=0.75pt]  [font=\tiny] [align=left] {69};
% Text Node
\draw (264.4,75.6) node [anchor=north west][inner sep=0.75pt]  [font=\tiny] [align=left] {26};
% Text Node
\draw (285.2,75.6) node [anchor=north west][inner sep=0.75pt]  [font=\tiny] [align=left] {29};
% Text Node
\draw (304.8,75.2) node [anchor=north west][inner sep=0.75pt]  [font=\tiny] [align=left] {28};
% Text Node
\draw (324.8,95.2) node [anchor=north west][inner sep=0.75pt]  [font=\tiny] [align=left] {63};
% Text Node
\draw (324.8,76.8) node [anchor=north west][inner sep=0.75pt]  [font=\tiny] [align=left] {70};
% Text Node
\draw (244.8,115.6) node [anchor=north west][inner sep=0.75pt]  [font=\tiny] [align=left] {28};
% Text Node
\draw (264.8,135.6) node [anchor=north west][inner sep=0.75pt]  [font=\tiny] [align=left] {66};
% Text Node
\draw (284.8,155.2) node [anchor=north west][inner sep=0.75pt]  [font=\tiny] [align=left] {77};
% Text Node
\draw (244,136) node [anchor=north west][inner sep=0.75pt]  [font=\tiny] [align=left] {63};
% Text Node
\draw (264,155.2) node [anchor=north west][inner sep=0.75pt]  [font=\tiny] [align=left] {74};
% Text Node
\draw (245.2,155.6) node [anchor=north west][inner sep=0.75pt]  [font=\tiny] [align=left] {73};
% Text Node
\draw (324.8,136.4) node [anchor=north west][inner sep=0.75pt]  [font=\tiny] [align=left] {68};
% Text Node
\draw (285.2,116.4) node [anchor=north west][inner sep=0.75pt]  [font=\tiny] [align=left] {23};
% Text Node
\draw (354.5,86.4) node [anchor=north west][inner sep=0.75pt]  [font=\tiny]  {$w=2$};

%Shape: Rectangle [id:dp7383053127861628] 
\draw  [color={rgb, 255:red, 0; green, 0; blue, 0 }  ,draw opacity=1 ] (209,54.5) -- (451,54.5) -- (451,190.5) -- (209,190.5) -- cycle ;
}
      \only<2>{
\tikzset{every picture/.style={line width=0.75pt}} %set default line width to 0.75pt        

%Shape: Rectangle [id:dp08033051369790078] 
\draw  [color={rgb, 255:red, 0; green, 0; blue, 0 }  ,draw opacity=1 ][fill={rgb, 255:red, 200; green, 43; blue, 43 }  ,fill opacity=0.23 ] (279.6,109) -- (299.6,109) -- (299.6,129) -- (279.6,129) -- cycle ;
%Shape: Rectangle [id:dp49683966747337727] 
\draw  [color={rgb, 255:red, 0; green, 0; blue, 0 }  ,draw opacity=1 ][fill={rgb, 255:red, 200; green, 43; blue, 43 }  ,fill opacity=0.64 ] (299.6,129) -- (319.6,129) -- (319.6,149) -- (299.6,149) -- cycle ;
%Shape: Rectangle [id:dp23104622107220785] 
\draw  [color={rgb, 255:red, 0; green, 0; blue, 0 }  ,draw opacity=1 ][fill={rgb, 255:red, 200; green, 43; blue, 43 }  ,fill opacity=0.64 ] (279.6,129) -- (299.6,129) -- (299.6,149) -- (279.6,149) -- cycle ;
%Shape: Rectangle [id:dp5617749383810637] 
\draw  [color={rgb, 255:red, 0; green, 0; blue, 0 }  ,draw opacity=1 ][fill={rgb, 255:red, 200; green, 43; blue, 43 }  ,fill opacity=0.69 ] (299.6,149) -- (319.6,149) -- (319.6,169) -- (299.6,169) -- cycle ;
%Shape: Rectangle [id:dp5334311904090966] 
\draw  [color={rgb, 255:red, 0; green, 0; blue, 0 }  ,draw opacity=1 ][fill={rgb, 255:red, 200; green, 43; blue, 43 }  ,fill opacity=0.68 ] (319.6,129) -- (339.6,129) -- (339.6,149) -- (319.6,149) -- cycle ;
%Shape: Rectangle [id:dp1721875732765945] 
\draw  [color={rgb, 255:red, 0; green, 0; blue, 0 }  ,draw opacity=1 ][fill={rgb, 255:red, 200; green, 43; blue, 43 }  ,fill opacity=0.32 ] (299.6,89) -- (319.6,89) -- (319.6,109) -- (299.6,109) -- cycle ;
%Shape: Rectangle [id:dp27080417117801914] 
\draw  [color={rgb, 255:red, 0; green, 0; blue, 0 }  ,draw opacity=1 ][fill={rgb, 255:red, 200; green, 43; blue, 43 }  ,fill opacity=0.64 ] (299.6,109) -- (319.6,109) -- (319.6,129) -- (299.6,129) -- cycle ;
%Shape: Rectangle [id:dp12733374768288575] 
\draw  [color={rgb, 255:red, 0; green, 0; blue, 0 }  ,draw opacity=1 ][fill={rgb, 255:red, 200; green, 43; blue, 43 }  ,fill opacity=0.23 ] (239.6,69) -- (259.6,69) -- (259.6,89) -- (239.6,89) -- cycle ;
%Shape: Rectangle [id:dp8140813955207011] 
\draw  [color={rgb, 255:red, 0; green, 0; blue, 0 }  ,draw opacity=1 ][fill={rgb, 255:red, 200; green, 43; blue, 43 }  ,fill opacity=0.21 ] (259.6,89) -- (279.6,89) -- (279.6,109) -- (259.6,109) -- cycle ;
%Shape: Rectangle [id:dp5441728742071007] 
\draw  [color={rgb, 255:red, 0; green, 0; blue, 0 }  ,draw opacity=1 ][fill={rgb, 255:red, 200; green, 43; blue, 43 }  ,fill opacity=0.25 ] (239.6,89) -- (259.6,89) -- (259.6,109) -- (239.6,109) -- cycle ;
%Shape: Rectangle [id:dp4352090259725204] 
\draw  [color={rgb, 255:red, 0; green, 0; blue, 0 }  ,draw opacity=1 ][fill={rgb, 255:red, 200; green, 43; blue, 43 }  ,fill opacity=0.28 ] (239.6,109) -- (259.6,109) -- (259.6,129) -- (239.6,129) -- cycle ;
%Shape: Rectangle [id:dp8369983389318182] 
\draw  [color={rgb, 255:red, 0; green, 0; blue, 0 }  ,draw opacity=1 ][fill={rgb, 255:red, 200; green, 43; blue, 43 }  ,fill opacity=0.66 ] (259.6,129) -- (279.6,129) -- (279.6,149) -- (259.6,149) -- cycle ;
%Shape: Rectangle [id:dp3622508883600454] 
\draw  [color={rgb, 255:red, 0; green, 0; blue, 0 }  ,draw opacity=1 ][fill={rgb, 255:red, 200; green, 43; blue, 43 }  ,fill opacity=0.77 ] (279.6,149) -- (299.6,149) -- (299.6,169) -- (279.6,169) -- cycle ;
%Shape: Rectangle [id:dp4909680033056917] 
\draw  [color={rgb, 255:red, 0; green, 0; blue, 0 }  ,draw opacity=1 ][fill={rgb, 255:red, 200; green, 43; blue, 43 }  ,fill opacity=0.24 ] (259.6,109) -- (279.6,109) -- (279.6,129) -- (259.6,129) -- cycle ;
%Shape: Rectangle [id:dp32148408980275944] 
\draw  [color={rgb, 255:red, 0; green, 0; blue, 0 }  ,draw opacity=1 ][fill={rgb, 255:red, 200; green, 43; blue, 43 }  ,fill opacity=0.26 ] (259.6,69) -- (279.6,69) -- (279.6,89) -- (259.6,89) -- cycle ;
%Shape: Rectangle [id:dp9565080577380919] 
\draw  [color={rgb, 255:red, 0; green, 0; blue, 0 }  ,draw opacity=1 ][fill={rgb, 255:red, 200; green, 43; blue, 43 }  ,fill opacity=0.22 ] (279.6,89) -- (299.6,89) -- (299.6,109) -- (279.6,109) -- cycle ;
%Shape: Rectangle [id:dp6802012811969649] 
\draw  [color={rgb, 255:red, 0; green, 0; blue, 0 }  ,draw opacity=1 ][fill={rgb, 255:red, 200; green, 43; blue, 43 }  ,fill opacity=0.29 ] (279.6,69) -- (299.6,69) -- (299.6,89) -- (279.6,89) -- cycle ;
%Shape: Rectangle [id:dp3097842072193001] 
\draw  [color={rgb, 255:red, 0; green, 0; blue, 0 }  ,draw opacity=1 ][fill={rgb, 255:red, 200; green, 43; blue, 43 }  ,fill opacity=0.28 ] (299.6,69) -- (319.6,69) -- (319.6,89) -- (299.6,89) -- cycle ;
%Shape: Rectangle [id:dp7093765311745719] 
\draw  [color={rgb, 255:red, 0; green, 0; blue, 0 }  ,draw opacity=1 ][fill={rgb, 255:red, 200; green, 43; blue, 43 }  ,fill opacity=0.65 ] (319.6,109) -- (339.6,109) -- (339.6,129) -- (319.6,129) -- cycle ;
%Shape: Rectangle [id:dp6727225745032497] 
\draw  [color={rgb, 255:red, 0; green, 0; blue, 0 }  ,draw opacity=1 ][fill={rgb, 255:red, 200; green, 43; blue, 43 }  ,fill opacity=0.64 ] (239.6,129) -- (259.6,129) -- (259.6,149) -- (239.6,149) -- cycle ;
%Shape: Rectangle [id:dp2838905896546031] 
\draw  [color={rgb, 255:red, 0; green, 0; blue, 0 }  ,draw opacity=1 ][fill={rgb, 255:red, 200; green, 43; blue, 43 }  ,fill opacity=0.73 ] (239.6,149) -- (259.6,149) -- (259.6,169) -- (239.6,169) -- cycle ;
%Shape: Rectangle [id:dp5249713236193003] 
\draw  [color={rgb, 255:red, 0; green, 0; blue, 0 }  ,draw opacity=1 ][fill={rgb, 255:red, 200; green, 43; blue, 43 }  ,fill opacity=0.74 ] (259.6,149) -- (279.6,149) -- (279.6,169) -- (259.6,169) -- cycle ;
%Shape: Rectangle [id:dp11634068894640182] 
\draw  [color={rgb, 255:red, 0; green, 0; blue, 0 }  ,draw opacity=1 ][fill={rgb, 255:red, 200; green, 43; blue, 43 }  ,fill opacity=0.68 ] (319.6,149) -- (339.6,149) -- (339.6,169) -- (319.6,169) -- cycle ;
%Shape: Rectangle [id:dp11798884947196353] 
\draw  [color={rgb, 255:red, 0; green, 0; blue, 0 }  ,draw opacity=1 ][fill={rgb, 255:red, 200; green, 43; blue, 43 }  ,fill opacity=0.7 ] (319.6,69) -- (339.6,69) -- (339.6,89) -- (319.6,89) -- cycle ;
%Shape: Rectangle [id:dp3086351088617566] 
\draw  [color={rgb, 255:red, 0; green, 0; blue, 0 }  ,draw opacity=1 ][fill={rgb, 255:red, 200; green, 43; blue, 43 }  ,fill opacity=0.63 ] (319.6,89) -- (339.6,89) -- (339.6,109) -- (319.6,109) -- cycle ;
%Shape: Rectangle [id:dp8644140098494498] 
\draw  [color={rgb, 255:red, 30; green, 14; blue, 222 }  ,draw opacity=1 ][fill={rgb, 255:red, 192; green, 177; blue, 255 }  ,fill opacity=0.34 ] (279.6,109) -- (299.6,109) -- (299.6,129) -- (279.6,129) -- cycle ;
%Shape: Rectangle [id:dp37575572566639026] 
\draw  [color={rgb, 255:red, 255; green, 255; blue, 255 }  ,draw opacity=1 ] (209,54.5) -- (451,54.5) -- (451,190.5) -- (209,190.5) -- cycle ;
%Shape: Rectangle [id:dp17946493354937076] 
\draw  [color={rgb, 255:red, 222; green, 195; blue, 14 }  ,draw opacity=1 ][fill={rgb, 255:red, 192; green, 177; blue, 255 }  ,fill opacity=0.34 ] (239.6,109) -- (259.6,109) -- (259.6,129) -- (239.6,129) -- cycle ;
%Shape: Rectangle [id:dp947603637741915] 
\draw  [color={rgb, 255:red, 222; green, 195; blue, 14 }  ,draw opacity=1 ][fill={rgb, 255:red, 192; green, 177; blue, 255 }  ,fill opacity=0.34 ] (259.6,335) -- (279.6,335) -- (279.6,355) -- (259.6,355) -- cycle ;

% Text Node
\draw (352.6,70) node [anchor=north west][inner sep=0.75pt]  [font=\tiny,color={rgb, 255:red, 72; green, 36; blue, 227 }  ,opacity=1 ] [align=left] {pixel considéré};
% Text Node
\draw (304.8,136) node [anchor=north west][inner sep=0.75pt]  [font=\tiny] [align=left] {64};
% Text Node
\draw (324.8,155.2) node [anchor=north west][inner sep=0.75pt]  [font=\tiny] [align=left] {69};
% Text Node
\draw (305.2,115.6) node [anchor=north west][inner sep=0.75pt]  [font=\tiny] [align=left] {65};
% Text Node
\draw (305.2,95.6) node [anchor=north west][inner sep=0.75pt]  [font=\tiny] [align=left] {32};
% Text Node
\draw (285.2,95.6) node [anchor=north west][inner sep=0.75pt]  [font=\tiny] [align=left] {22};
% Text Node
\draw (325.2,115.6) node [anchor=north west][inner sep=0.75pt]  [font=\tiny] [align=left] {65};
% Text Node
\draw (244.4,76) node [anchor=north west][inner sep=0.75pt]  [font=\tiny] [align=left] {23};
% Text Node
\draw (265.2,96) node [anchor=north west][inner sep=0.75pt]  [font=\tiny] [align=left] {21};
% Text Node
\draw (244.8,95.6) node [anchor=north west][inner sep=0.75pt]  [font=\tiny] [align=left] {25};
% Text Node
\draw (264.8,116.4) node [anchor=north west][inner sep=0.75pt]  [font=\tiny] [align=left] {24};
% Text Node
\draw (284.8,135.6) node [anchor=north west][inner sep=0.75pt]  [font=\tiny] [align=left] {64};
% Text Node
\draw (304.8,155.2) node [anchor=north west][inner sep=0.75pt]  [font=\tiny] [align=left] {69};
% Text Node
\draw (264.4,75.6) node [anchor=north west][inner sep=0.75pt]  [font=\tiny] [align=left] {26};
% Text Node
\draw (285.2,75.6) node [anchor=north west][inner sep=0.75pt]  [font=\tiny] [align=left] {29};
% Text Node
\draw (304.8,75.2) node [anchor=north west][inner sep=0.75pt]  [font=\tiny] [align=left] {28};
% Text Node
\draw (324.8,95.2) node [anchor=north west][inner sep=0.75pt]  [font=\tiny] [align=left] {63};
% Text Node
\draw (324.8,76.8) node [anchor=north west][inner sep=0.75pt]  [font=\tiny] [align=left] {70};
% Text Node
\draw (244.8,115.6) node [anchor=north west][inner sep=0.75pt]  [font=\tiny] [align=left] {28};
% Text Node
\draw (264.8,135.6) node [anchor=north west][inner sep=0.75pt]  [font=\tiny] [align=left] {66};
% Text Node
\draw (284.8,155.2) node [anchor=north west][inner sep=0.75pt]  [font=\tiny] [align=left] {77};
% Text Node
\draw (244,136) node [anchor=north west][inner sep=0.75pt]  [font=\tiny] [align=left] {63};
% Text Node
\draw (264,155.2) node [anchor=north west][inner sep=0.75pt]  [font=\tiny] [align=left] {74};
% Text Node
\draw (245.2,155.6) node [anchor=north west][inner sep=0.75pt]  [font=\tiny] [align=left] {73};
% Text Node
\draw (324.8,136.4) node [anchor=north west][inner sep=0.75pt]  [font=\tiny] [align=left] {68};
% Text Node
\draw (285.2,116.4) node [anchor=north west][inner sep=0.75pt]  [font=\tiny] [align=left] {23};
% Text Node
\draw (354.5,86.4) node [anchor=north west][inner sep=0.75pt]  [font=\tiny]  {$w=2$};
% Text Node
\draw (357.3,135.4) node [anchor=north west][inner sep=0.75pt]  [font=\tiny]  {$i=-2$};
% Text Node
\draw (357.1,108.8) node [anchor=north west][inner sep=0.75pt]  [font=\tiny]  {$dx=1,\ dy=0$};
% Text Node
\draw (355.33,125.27) node [anchor=north west][inner sep=0.75pt]  [font=\tiny,color={rgb, 255:red, 72; green, 36; blue, 227 }  ,opacity=1 ] [align=left] {\textcolor[rgb]{0.55,0.64,0.02}{pixel comparé}};
% Text Node
\draw (357,98.13) node [anchor=north west][inner sep=0.75pt]  [font=\tiny,color={rgb, 255:red, 72; green, 36; blue, 227 }  ,opacity=1 ] [align=left] {direction horizontal};
% Text Node
\draw (357.3,154.8) node [anchor=north west][inner sep=0.75pt]  [font=\tiny]  {$S=28$};
% Text Node
\draw (393.3,154.4) node [anchor=north west][inner sep=0.75pt]  [font=\tiny]  {$S^{2} =784$};
%Shape: Rectangle [id:dp7383053127861628] 
\draw  [color={rgb, 255:red, 0; green, 0; blue, 0 }  ,draw opacity=1 ] (209,54.5) -- (451,54.5) -- (451,190.5) -- (209,190.5) -- cycle ;

}
      \only<3>{
\tikzset{every picture/.style={line width=0.75pt}} %set default line width to 0.75pt        



%Shape: Rectangle [id:dp41623934555948805] 
\draw  [color={rgb, 255:red, 0; green, 0; blue, 0 }  ,draw opacity=1 ][fill={rgb, 255:red, 200; green, 43; blue, 43 }  ,fill opacity=0.23 ] (279.6,109) -- (299.6,109) -- (299.6,129) -- (279.6,129) -- cycle ;
%Shape: Rectangle [id:dp15742549259652527] 
\draw  [color={rgb, 255:red, 0; green, 0; blue, 0 }  ,draw opacity=1 ][fill={rgb, 255:red, 200; green, 43; blue, 43 }  ,fill opacity=0.64 ] (299.6,129) -- (319.6,129) -- (319.6,149) -- (299.6,149) -- cycle ;
%Shape: Rectangle [id:dp3863982180716097] 
\draw  [color={rgb, 255:red, 0; green, 0; blue, 0 }  ,draw opacity=1 ][fill={rgb, 255:red, 200; green, 43; blue, 43 }  ,fill opacity=0.64 ] (279.6,129) -- (299.6,129) -- (299.6,149) -- (279.6,149) -- cycle ;
%Shape: Rectangle [id:dp478374436473631] 
\draw  [color={rgb, 255:red, 0; green, 0; blue, 0 }  ,draw opacity=1 ][fill={rgb, 255:red, 200; green, 43; blue, 43 }  ,fill opacity=0.69 ] (299.6,149) -- (319.6,149) -- (319.6,169) -- (299.6,169) -- cycle ;
%Shape: Rectangle [id:dp8911880470737273] 
\draw  [color={rgb, 255:red, 0; green, 0; blue, 0 }  ,draw opacity=1 ][fill={rgb, 255:red, 200; green, 43; blue, 43 }  ,fill opacity=0.68 ] (319.6,129) -- (339.6,129) -- (339.6,149) -- (319.6,149) -- cycle ;
%Shape: Rectangle [id:dp04503138886073499] 
\draw  [color={rgb, 255:red, 0; green, 0; blue, 0 }  ,draw opacity=1 ][fill={rgb, 255:red, 200; green, 43; blue, 43 }  ,fill opacity=0.32 ] (299.6,89) -- (319.6,89) -- (319.6,109) -- (299.6,109) -- cycle ;
%Shape: Rectangle [id:dp9644609206003013] 
\draw  [color={rgb, 255:red, 0; green, 0; blue, 0 }  ,draw opacity=1 ][fill={rgb, 255:red, 200; green, 43; blue, 43 }  ,fill opacity=0.64 ] (299.6,109) -- (319.6,109) -- (319.6,129) -- (299.6,129) -- cycle ;
%Shape: Rectangle [id:dp39507963239061805] 
\draw  [color={rgb, 255:red, 0; green, 0; blue, 0 }  ,draw opacity=1 ][fill={rgb, 255:red, 200; green, 43; blue, 43 }  ,fill opacity=0.23 ] (239.6,69) -- (259.6,69) -- (259.6,89) -- (239.6,89) -- cycle ;
%Shape: Rectangle [id:dp09172875488400045] 
\draw  [color={rgb, 255:red, 0; green, 0; blue, 0 }  ,draw opacity=1 ][fill={rgb, 255:red, 200; green, 43; blue, 43 }  ,fill opacity=0.21 ] (259.6,89) -- (279.6,89) -- (279.6,109) -- (259.6,109) -- cycle ;
%Shape: Rectangle [id:dp3660178952403952] 
\draw  [color={rgb, 255:red, 0; green, 0; blue, 0 }  ,draw opacity=1 ][fill={rgb, 255:red, 200; green, 43; blue, 43 }  ,fill opacity=0.25 ] (239.6,89) -- (259.6,89) -- (259.6,109) -- (239.6,109) -- cycle ;
%Shape: Rectangle [id:dp9520992808429042] 
\draw  [color={rgb, 255:red, 0; green, 0; blue, 0 }  ,draw opacity=1 ][fill={rgb, 255:red, 200; green, 43; blue, 43 }  ,fill opacity=0.28 ] (239.6,109) -- (259.6,109) -- (259.6,129) -- (239.6,129) -- cycle ;
%Shape: Rectangle [id:dp9715922571371791] 
\draw  [color={rgb, 255:red, 0; green, 0; blue, 0 }  ,draw opacity=1 ][fill={rgb, 255:red, 200; green, 43; blue, 43 }  ,fill opacity=0.66 ] (259.6,129) -- (279.6,129) -- (279.6,149) -- (259.6,149) -- cycle ;
%Shape: Rectangle [id:dp3583133966943417] 
\draw  [color={rgb, 255:red, 0; green, 0; blue, 0 }  ,draw opacity=1 ][fill={rgb, 255:red, 200; green, 43; blue, 43 }  ,fill opacity=0.77 ] (279.6,149) -- (299.6,149) -- (299.6,169) -- (279.6,169) -- cycle ;
%Shape: Rectangle [id:dp022168816978197836] 
\draw  [color={rgb, 255:red, 0; green, 0; blue, 0 }  ,draw opacity=1 ][fill={rgb, 255:red, 200; green, 43; blue, 43 }  ,fill opacity=0.24 ] (259.6,109) -- (279.6,109) -- (279.6,129) -- (259.6,129) -- cycle ;
%Shape: Rectangle [id:dp03377422539076902] 
\draw  [color={rgb, 255:red, 0; green, 0; blue, 0 }  ,draw opacity=1 ][fill={rgb, 255:red, 200; green, 43; blue, 43 }  ,fill opacity=0.26 ] (259.6,69) -- (279.6,69) -- (279.6,89) -- (259.6,89) -- cycle ;
%Shape: Rectangle [id:dp5054454472454241] 
\draw  [color={rgb, 255:red, 0; green, 0; blue, 0 }  ,draw opacity=1 ][fill={rgb, 255:red, 200; green, 43; blue, 43 }  ,fill opacity=0.22 ] (279.6,89) -- (299.6,89) -- (299.6,109) -- (279.6,109) -- cycle ;
%Shape: Rectangle [id:dp7342890782294983] 
\draw  [color={rgb, 255:red, 0; green, 0; blue, 0 }  ,draw opacity=1 ][fill={rgb, 255:red, 200; green, 43; blue, 43 }  ,fill opacity=0.29 ] (279.6,69) -- (299.6,69) -- (299.6,89) -- (279.6,89) -- cycle ;
%Shape: Rectangle [id:dp005507899969640073] 
\draw  [color={rgb, 255:red, 0; green, 0; blue, 0 }  ,draw opacity=1 ][fill={rgb, 255:red, 200; green, 43; blue, 43 }  ,fill opacity=0.28 ] (299.6,69) -- (319.6,69) -- (319.6,89) -- (299.6,89) -- cycle ;
%Shape: Rectangle [id:dp017524738654428162] 
\draw  [color={rgb, 255:red, 0; green, 0; blue, 0 }  ,draw opacity=1 ][fill={rgb, 255:red, 200; green, 43; blue, 43 }  ,fill opacity=0.65 ] (319.6,109) -- (339.6,109) -- (339.6,129) -- (319.6,129) -- cycle ;
%Shape: Rectangle [id:dp37178653266438877] 
\draw  [color={rgb, 255:red, 0; green, 0; blue, 0 }  ,draw opacity=1 ][fill={rgb, 255:red, 200; green, 43; blue, 43 }  ,fill opacity=0.64 ] (239.6,129) -- (259.6,129) -- (259.6,149) -- (239.6,149) -- cycle ;
%Shape: Rectangle [id:dp43511000750558937] 
\draw  [color={rgb, 255:red, 0; green, 0; blue, 0 }  ,draw opacity=1 ][fill={rgb, 255:red, 200; green, 43; blue, 43 }  ,fill opacity=0.73 ] (239.6,149) -- (259.6,149) -- (259.6,169) -- (239.6,169) -- cycle ;
%Shape: Rectangle [id:dp7129303286636162] 
\draw  [color={rgb, 255:red, 0; green, 0; blue, 0 }  ,draw opacity=1 ][fill={rgb, 255:red, 200; green, 43; blue, 43 }  ,fill opacity=0.74 ] (259.6,149) -- (279.6,149) -- (279.6,169) -- (259.6,169) -- cycle ;
%Shape: Rectangle [id:dp5549125040121842] 
\draw  [color={rgb, 255:red, 0; green, 0; blue, 0 }  ,draw opacity=1 ][fill={rgb, 255:red, 200; green, 43; blue, 43 }  ,fill opacity=0.68 ] (319.6,149) -- (339.6,149) -- (339.6,169) -- (319.6,169) -- cycle ;
%Shape: Rectangle [id:dp6510207368905343] 
\draw  [color={rgb, 255:red, 0; green, 0; blue, 0 }  ,draw opacity=1 ][fill={rgb, 255:red, 200; green, 43; blue, 43 }  ,fill opacity=0.7 ] (319.6,69) -- (339.6,69) -- (339.6,89) -- (319.6,89) -- cycle ;
%Shape: Rectangle [id:dp5732403849087179] 
\draw  [color={rgb, 255:red, 0; green, 0; blue, 0 }  ,draw opacity=1 ][fill={rgb, 255:red, 200; green, 43; blue, 43 }  ,fill opacity=0.63 ] (319.6,89) -- (339.6,89) -- (339.6,109) -- (319.6,109) -- cycle ;
%Shape: Rectangle [id:dp18147294007765313] 
\draw  [color={rgb, 255:red, 30; green, 14; blue, 222 }  ,draw opacity=1 ][fill={rgb, 255:red, 192; green, 177; blue, 255 }  ,fill opacity=0.34 ] (279.6,109) -- (299.6,109) -- (299.6,129) -- (279.6,129) -- cycle ;
%Shape: Rectangle [id:dp7383053127861628] 
\draw  [color={rgb, 255:red, 0; green, 0; blue, 0 }  ,draw opacity=1 ] (209,54.5) -- (451,54.5) -- (451,190.5) -- (209,190.5) -- cycle ;
%Shape: Rectangle [id:dp28157309764636895] 
\draw  [color={rgb, 255:red, 222; green, 195; blue, 14 }  ,draw opacity=1 ][fill={rgb, 255:red, 192; green, 177; blue, 255 }  ,fill opacity=0.34 ] (259.6,109) -- (279.6,109) -- (279.6,129) -- (259.6,129) -- cycle ;

% Text Node
\draw (352.6,70) node [anchor=north west][inner sep=0.75pt]  [font=\tiny,color={rgb, 255:red, 72; green, 36; blue, 227 }  ,opacity=1 ] [align=left] {pixel considéré};
% Text Node
\draw (304.8,136) node [anchor=north west][inner sep=0.75pt]  [font=\tiny] [align=left] {64};
% Text Node
\draw (324.8,155.2) node [anchor=north west][inner sep=0.75pt]  [font=\tiny] [align=left] {69};
% Text Node
\draw (305.2,115.6) node [anchor=north west][inner sep=0.75pt]  [font=\tiny] [align=left] {65};
% Text Node
\draw (305.2,95.6) node [anchor=north west][inner sep=0.75pt]  [font=\tiny] [align=left] {32};
% Text Node
\draw (285.2,95.6) node [anchor=north west][inner sep=0.75pt]  [font=\tiny] [align=left] {22};
% Text Node
\draw (325.2,115.6) node [anchor=north west][inner sep=0.75pt]  [font=\tiny] [align=left] {65};
% Text Node
\draw (244.4,76) node [anchor=north west][inner sep=0.75pt]  [font=\tiny] [align=left] {23};
% Text Node
\draw (265.2,96) node [anchor=north west][inner sep=0.75pt]  [font=\tiny] [align=left] {21};
% Text Node
\draw (244.8,95.6) node [anchor=north west][inner sep=0.75pt]  [font=\tiny] [align=left] {25};
% Text Node
\draw (264.8,116.4) node [anchor=north west][inner sep=0.75pt]  [font=\tiny] [align=left] {24};
% Text Node
\draw (284.8,135.6) node [anchor=north west][inner sep=0.75pt]  [font=\tiny] [align=left] {64};
% Text Node
\draw (304.8,155.2) node [anchor=north west][inner sep=0.75pt]  [font=\tiny] [align=left] {69};
% Text Node
\draw (264.4,75.6) node [anchor=north west][inner sep=0.75pt]  [font=\tiny] [align=left] {26};
% Text Node
\draw (285.2,75.6) node [anchor=north west][inner sep=0.75pt]  [font=\tiny] [align=left] {29};
% Text Node
\draw (304.8,75.2) node [anchor=north west][inner sep=0.75pt]  [font=\tiny] [align=left] {28};
% Text Node
\draw (324.8,95.2) node [anchor=north west][inner sep=0.75pt]  [font=\tiny] [align=left] {63};
% Text Node
\draw (324.8,76.8) node [anchor=north west][inner sep=0.75pt]  [font=\tiny] [align=left] {70};
% Text Node
\draw (244.8,115.6) node [anchor=north west][inner sep=0.75pt]  [font=\tiny] [align=left] {28};
% Text Node
\draw (264.8,135.6) node [anchor=north west][inner sep=0.75pt]  [font=\tiny] [align=left] {66};
% Text Node
\draw (284.8,155.2) node [anchor=north west][inner sep=0.75pt]  [font=\tiny] [align=left] {77};
% Text Node
\draw (244,136) node [anchor=north west][inner sep=0.75pt]  [font=\tiny] [align=left] {63};
% Text Node
\draw (264,155.2) node [anchor=north west][inner sep=0.75pt]  [font=\tiny] [align=left] {74};
% Text Node
\draw (245.2,155.6) node [anchor=north west][inner sep=0.75pt]  [font=\tiny] [align=left] {73};
% Text Node
\draw (324.8,136.4) node [anchor=north west][inner sep=0.75pt]  [font=\tiny] [align=left] {68};
% Text Node
\draw (285.2,116.4) node [anchor=north west][inner sep=0.75pt]  [font=\tiny] [align=left] {23};
% Text Node
\draw (354.5,86.4) node [anchor=north west][inner sep=0.75pt]  [font=\tiny]  {$w=2$};
% Text Node
\draw (357.3,135.4) node [anchor=north west][inner sep=0.75pt]  [font=\tiny]  {$i=-1$};
% Text Node
\draw (357.1,108.8) node [anchor=north west][inner sep=0.75pt]  [font=\tiny]  {$dx=1,\ dy=0$};
% Text Node
\draw (355.33,125.27) node [anchor=north west][inner sep=0.75pt]  [font=\tiny,color={rgb, 255:red, 72; green, 36; blue, 227 }  ,opacity=1 ] [align=left] {\textcolor[rgb]{0.55,0.64,0.02}{pixel comparé}};
% Text Node
\draw (357,98.13) node [anchor=north west][inner sep=0.75pt]  [font=\tiny,color={rgb, 255:red, 72; green, 36; blue, 227 }  ,opacity=1 ] [align=left] {direction horizontal};
% Text Node
\draw (357.3,154.8) node [anchor=north west][inner sep=0.75pt]  [font=\tiny]  {$S=52$};
% Text Node
\draw (393.3,154.4) node [anchor=north west][inner sep=0.75pt]  [font=\tiny]  {$S^{2} =1248$};

}
      \only<4>{\input{donnees/s1/tikz1_4.tex}}
      \only<5>{
\tikzset{every picture/.style={line width=0.75pt}} %set default line width to 0.75pt        


%Shape: Rectangle [id:dp5375598584667333] 
\draw  [color={rgb, 255:red, 0; green, 0; blue, 0 }  ,draw opacity=1 ][fill={rgb, 255:red, 200; green, 43; blue, 43 }  ,fill opacity=0.23 ] (279.6,109) -- (299.6,109) -- (299.6,129) -- (279.6,129) -- cycle ;
%Shape: Rectangle [id:dp6362028799834906] 
\draw  [color={rgb, 255:red, 0; green, 0; blue, 0 }  ,draw opacity=1 ][fill={rgb, 255:red, 200; green, 43; blue, 43 }  ,fill opacity=0.64 ] (299.6,129) -- (319.6,129) -- (319.6,149) -- (299.6,149) -- cycle ;
%Shape: Rectangle [id:dp17138982287354876] 
\draw  [color={rgb, 255:red, 0; green, 0; blue, 0 }  ,draw opacity=1 ][fill={rgb, 255:red, 200; green, 43; blue, 43 }  ,fill opacity=0.64 ] (279.6,129) -- (299.6,129) -- (299.6,149) -- (279.6,149) -- cycle ;
%Shape: Rectangle [id:dp11168678440472213] 
\draw  [color={rgb, 255:red, 0; green, 0; blue, 0 }  ,draw opacity=1 ][fill={rgb, 255:red, 200; green, 43; blue, 43 }  ,fill opacity=0.69 ] (299.6,149) -- (319.6,149) -- (319.6,169) -- (299.6,169) -- cycle ;
%Shape: Rectangle [id:dp10868115608692452] 
\draw  [color={rgb, 255:red, 0; green, 0; blue, 0 }  ,draw opacity=1 ][fill={rgb, 255:red, 200; green, 43; blue, 43 }  ,fill opacity=0.68 ] (319.6,129) -- (339.6,129) -- (339.6,149) -- (319.6,149) -- cycle ;
%Shape: Rectangle [id:dp768485299456727] 
\draw  [color={rgb, 255:red, 0; green, 0; blue, 0 }  ,draw opacity=1 ][fill={rgb, 255:red, 200; green, 43; blue, 43 }  ,fill opacity=0.32 ] (299.6,89) -- (319.6,89) -- (319.6,109) -- (299.6,109) -- cycle ;
%Shape: Rectangle [id:dp04068165547025682] 
\draw  [color={rgb, 255:red, 0; green, 0; blue, 0 }  ,draw opacity=1 ][fill={rgb, 255:red, 200; green, 43; blue, 43 }  ,fill opacity=0.64 ] (299.6,109) -- (319.6,109) -- (319.6,129) -- (299.6,129) -- cycle ;
%Shape: Rectangle [id:dp4604312220958826] 
\draw  [color={rgb, 255:red, 0; green, 0; blue, 0 }  ,draw opacity=1 ][fill={rgb, 255:red, 200; green, 43; blue, 43 }  ,fill opacity=0.23 ] (239.6,69) -- (259.6,69) -- (259.6,89) -- (239.6,89) -- cycle ;
%Shape: Rectangle [id:dp8858359898164444] 
\draw  [color={rgb, 255:red, 0; green, 0; blue, 0 }  ,draw opacity=1 ][fill={rgb, 255:red, 200; green, 43; blue, 43 }  ,fill opacity=0.21 ] (259.6,89) -- (279.6,89) -- (279.6,109) -- (259.6,109) -- cycle ;
%Shape: Rectangle [id:dp13618530730853728] 
\draw  [color={rgb, 255:red, 0; green, 0; blue, 0 }  ,draw opacity=1 ][fill={rgb, 255:red, 200; green, 43; blue, 43 }  ,fill opacity=0.25 ] (239.6,89) -- (259.6,89) -- (259.6,109) -- (239.6,109) -- cycle ;
%Shape: Rectangle [id:dp1678578319512094] 
\draw  [color={rgb, 255:red, 0; green, 0; blue, 0 }  ,draw opacity=1 ][fill={rgb, 255:red, 200; green, 43; blue, 43 }  ,fill opacity=0.28 ] (239.6,109) -- (259.6,109) -- (259.6,129) -- (239.6,129) -- cycle ;
%Shape: Rectangle [id:dp061963626898036694] 
\draw  [color={rgb, 255:red, 0; green, 0; blue, 0 }  ,draw opacity=1 ][fill={rgb, 255:red, 200; green, 43; blue, 43 }  ,fill opacity=0.66 ] (259.6,129) -- (279.6,129) -- (279.6,149) -- (259.6,149) -- cycle ;
%Shape: Rectangle [id:dp45445804554823566] 
\draw  [color={rgb, 255:red, 0; green, 0; blue, 0 }  ,draw opacity=1 ][fill={rgb, 255:red, 200; green, 43; blue, 43 }  ,fill opacity=0.77 ] (279.6,149) -- (299.6,149) -- (299.6,169) -- (279.6,169) -- cycle ;
%Shape: Rectangle [id:dp8471899987892493] 
\draw  [color={rgb, 255:red, 0; green, 0; blue, 0 }  ,draw opacity=1 ][fill={rgb, 255:red, 200; green, 43; blue, 43 }  ,fill opacity=0.24 ] (259.6,109) -- (279.6,109) -- (279.6,129) -- (259.6,129) -- cycle ;
%Shape: Rectangle [id:dp7454863327917263] 
\draw  [color={rgb, 255:red, 0; green, 0; blue, 0 }  ,draw opacity=1 ][fill={rgb, 255:red, 200; green, 43; blue, 43 }  ,fill opacity=0.26 ] (259.6,69) -- (279.6,69) -- (279.6,89) -- (259.6,89) -- cycle ;
%Shape: Rectangle [id:dp6027156226511302] 
\draw  [color={rgb, 255:red, 0; green, 0; blue, 0 }  ,draw opacity=1 ][fill={rgb, 255:red, 200; green, 43; blue, 43 }  ,fill opacity=0.22 ] (279.6,89) -- (299.6,89) -- (299.6,109) -- (279.6,109) -- cycle ;
%Shape: Rectangle [id:dp2772118388277819] 
\draw  [color={rgb, 255:red, 0; green, 0; blue, 0 }  ,draw opacity=1 ][fill={rgb, 255:red, 200; green, 43; blue, 43 }  ,fill opacity=0.29 ] (279.6,69) -- (299.6,69) -- (299.6,89) -- (279.6,89) -- cycle ;
%Shape: Rectangle [id:dp7719372959811857] 
\draw  [color={rgb, 255:red, 0; green, 0; blue, 0 }  ,draw opacity=1 ][fill={rgb, 255:red, 200; green, 43; blue, 43 }  ,fill opacity=0.28 ] (299.6,69) -- (319.6,69) -- (319.6,89) -- (299.6,89) -- cycle ;
%Shape: Rectangle [id:dp5124780871937471] 
\draw  [color={rgb, 255:red, 0; green, 0; blue, 0 }  ,draw opacity=1 ][fill={rgb, 255:red, 200; green, 43; blue, 43 }  ,fill opacity=0.65 ] (319.6,109) -- (339.6,109) -- (339.6,129) -- (319.6,129) -- cycle ;
%Shape: Rectangle [id:dp3617179862334128] 
\draw  [color={rgb, 255:red, 0; green, 0; blue, 0 }  ,draw opacity=1 ][fill={rgb, 255:red, 200; green, 43; blue, 43 }  ,fill opacity=0.64 ] (239.6,129) -- (259.6,129) -- (259.6,149) -- (239.6,149) -- cycle ;
%Shape: Rectangle [id:dp32793534866540996] 
\draw  [color={rgb, 255:red, 0; green, 0; blue, 0 }  ,draw opacity=1 ][fill={rgb, 255:red, 200; green, 43; blue, 43 }  ,fill opacity=0.73 ] (239.6,149) -- (259.6,149) -- (259.6,169) -- (239.6,169) -- cycle ;
%Shape: Rectangle [id:dp36554937028089873] 
\draw  [color={rgb, 255:red, 0; green, 0; blue, 0 }  ,draw opacity=1 ][fill={rgb, 255:red, 200; green, 43; blue, 43 }  ,fill opacity=0.74 ] (259.6,149) -- (279.6,149) -- (279.6,169) -- (259.6,169) -- cycle ;
%Shape: Rectangle [id:dp17667518825747697] 
\draw  [color={rgb, 255:red, 0; green, 0; blue, 0 }  ,draw opacity=1 ][fill={rgb, 255:red, 200; green, 43; blue, 43 }  ,fill opacity=0.68 ] (319.6,149) -- (339.6,149) -- (339.6,169) -- (319.6,169) -- cycle ;
%Shape: Rectangle [id:dp4256535156440978] 
\draw  [color={rgb, 255:red, 0; green, 0; blue, 0 }  ,draw opacity=1 ][fill={rgb, 255:red, 200; green, 43; blue, 43 }  ,fill opacity=0.7 ] (319.6,69) -- (339.6,69) -- (339.6,89) -- (319.6,89) -- cycle ;
%Shape: Rectangle [id:dp3346814044539125] 
\draw  [color={rgb, 255:red, 0; green, 0; blue, 0 }  ,draw opacity=1 ][fill={rgb, 255:red, 200; green, 43; blue, 43 }  ,fill opacity=0.63 ] (319.6,89) -- (339.6,89) -- (339.6,109) -- (319.6,109) -- cycle ;
%Shape: Rectangle [id:dp39668089199593637] 
\draw  [color={rgb, 255:red, 30; green, 14; blue, 222 }  ,draw opacity=1 ][fill={rgb, 255:red, 192; green, 177; blue, 255 }  ,fill opacity=0.34 ] (279.6,109) -- (299.6,109) -- (299.6,129) -- (279.6,129) -- cycle ;
%Shape: Rectangle [id:dp17425338332474016] 
\draw  [color={rgb, 255:red, 255; green, 255; blue, 255 }  ,draw opacity=1 ] (209,54.5) -- (451,54.5) -- (451,190.5) -- (209,190.5) -- cycle ;
%Shape: Rectangle [id:dp23241515965389126] 
\draw  [color={rgb, 255:red, 222; green, 195; blue, 14 }  ,draw opacity=1 ][fill={rgb, 255:red, 192; green, 177; blue, 255 }  ,fill opacity=0.34 ] (319.6,109) -- (339.6,109) -- (339.6,129) -- (319.6,129) -- cycle ;

% Text Node
\draw (352.6,70) node [anchor=north west][inner sep=0.75pt]  [font=\tiny,color={rgb, 255:red, 72; green, 36; blue, 227 }  ,opacity=1 ] [align=left] {pixel considéré};
% Text Node
\draw (304.8,136) node [anchor=north west][inner sep=0.75pt]  [font=\tiny] [align=left] {64};
% Text Node
\draw (324.8,155.2) node [anchor=north west][inner sep=0.75pt]  [font=\tiny] [align=left] {69};
% Text Node
\draw (305.2,115.6) node [anchor=north west][inner sep=0.75pt]  [font=\tiny] [align=left] {65};
% Text Node
\draw (305.2,95.6) node [anchor=north west][inner sep=0.75pt]  [font=\tiny] [align=left] {32};
% Text Node
\draw (285.2,95.6) node [anchor=north west][inner sep=0.75pt]  [font=\tiny] [align=left] {22};
% Text Node
\draw (325.2,115.6) node [anchor=north west][inner sep=0.75pt]  [font=\tiny] [align=left] {65};
% Text Node
\draw (244.4,76) node [anchor=north west][inner sep=0.75pt]  [font=\tiny] [align=left] {23};
% Text Node
\draw (265.2,96) node [anchor=north west][inner sep=0.75pt]  [font=\tiny] [align=left] {21};
% Text Node
\draw (244.8,95.6) node [anchor=north west][inner sep=0.75pt]  [font=\tiny] [align=left] {25};
% Text Node
\draw (264.8,116.4) node [anchor=north west][inner sep=0.75pt]  [font=\tiny] [align=left] {24};
% Text Node
\draw (284.8,135.6) node [anchor=north west][inner sep=0.75pt]  [font=\tiny] [align=left] {64};
% Text Node
\draw (304.8,155.2) node [anchor=north west][inner sep=0.75pt]  [font=\tiny] [align=left] {69};
% Text Node
\draw (264.4,75.6) node [anchor=north west][inner sep=0.75pt]  [font=\tiny] [align=left] {26};
% Text Node
\draw (285.2,75.6) node [anchor=north west][inner sep=0.75pt]  [font=\tiny] [align=left] {29};
% Text Node
\draw (304.8,75.2) node [anchor=north west][inner sep=0.75pt]  [font=\tiny] [align=left] {28};
% Text Node
\draw (324.8,95.2) node [anchor=north west][inner sep=0.75pt]  [font=\tiny] [align=left] {63};
% Text Node
\draw (324.8,76.8) node [anchor=north west][inner sep=0.75pt]  [font=\tiny] [align=left] {70};
% Text Node
\draw (244.8,115.6) node [anchor=north west][inner sep=0.75pt]  [font=\tiny] [align=left] {28};
% Text Node
\draw (264.8,135.6) node [anchor=north west][inner sep=0.75pt]  [font=\tiny] [align=left] {66};
% Text Node
\draw (284.8,155.2) node [anchor=north west][inner sep=0.75pt]  [font=\tiny] [align=left] {77};
% Text Node
\draw (244,136) node [anchor=north west][inner sep=0.75pt]  [font=\tiny] [align=left] {63};
% Text Node
\draw (264,155.2) node [anchor=north west][inner sep=0.75pt]  [font=\tiny] [align=left] {74};
% Text Node
\draw (245.2,155.6) node [anchor=north west][inner sep=0.75pt]  [font=\tiny] [align=left] {73};
% Text Node
\draw (324.8,136.4) node [anchor=north west][inner sep=0.75pt]  [font=\tiny] [align=left] {68};
% Text Node
\draw (285.2,116.4) node [anchor=north west][inner sep=0.75pt]  [font=\tiny] [align=left] {23};
% Text Node
\draw (354.5,86.4) node [anchor=north west][inner sep=0.75pt]  [font=\tiny]  {$w=2$};
% Text Node
\draw (354.8,114.33) node [anchor=north west][inner sep=0.75pt]  [font=\tiny,color={rgb, 255:red, 72; green, 36; blue, 227 }  ,opacity=1 ] [align=left] {Var(1,0)=386.8};
% Text Node
\draw (355.9,98.2) node [anchor=north west][inner sep=0.75pt]  [font=\tiny]  {$T=300$};
% Text Node
\draw (355.4,124.33) node [anchor=north west][inner sep=0.75pt]  [font=\tiny,color={rgb, 255:red, 72; green, 36; blue, 227 }  ,opacity=1 ] [align=left] {Var(0,1)=526.8};
% Text Node
\draw (354.6,133.73) node [anchor=north west][inner sep=0.75pt]  [font=\tiny,color={rgb, 255:red, 72; green, 36; blue, 227 }  ,opacity=1 ] [align=left] {Var(1,1)=439.7};
% Text Node
\draw (355,143.93) node [anchor=north west][inner sep=0.75pt]  [font=\tiny,color={rgb, 255:red, 72; green, 36; blue, 227 }  ,opacity=1 ] [align=left] {Var(1,-1)=471.2};
% Text Node
\draw (356.23,170.2) node [anchor=north west][inner sep=0.75pt]  [font=\tiny]  {$S >T\Longrightarrow $};
% Text Node
\draw (395.93,170) node [anchor=north west][inner sep=0.75pt]  [font=\tiny,color={rgb, 255:red, 0; green, 0; blue, 0 }  ,opacity=1 ] [align=left] {point d'intérêt};
% Text Node
\draw (356.57,157.53) node [anchor=north west][inner sep=0.75pt]  [font=\tiny]  {$S=386.8$};
%Shape: Rectangle [id:dp7383053127861628] 
\draw  [color={rgb, 255:red, 0; green, 0; blue, 0 }  ,draw opacity=1 ] (209,54.5) -- (451,54.5) -- (451,190.5) -- (209,190.5) -- cycle ;
}
    \end{tikzpicture}
  \end{center}
\end{frame}

%------------------------------------------------

\begin{frame}
	\small
\frametitle{Algorithme type \lin{Moravec}}
\begin{algorithm}[H]
    \caption{\textsf{Moravec (minimum des variances)}}
    \Input{Image d’intensité \texttt{image}}
    \Output{Liste des coins détectés}
    \BlankLine
    \ForEach{pixel $(x, y)$ dans l’image}{
        $scores \gets$ liste vide\;
        \ForEach{direction $(dx, dy)$ parmi : verticale, horizontale, diagonales}{
            Calculer la variance locale autour de $(x, y)$ dans la direction $(dx, dy)$\;
            Ajouter la variance à $scores$\;
        }
        $score \gets \min(scores)$\;
        \If{$score >$ \texttt{SEUIL}}{
            Marquer $(x, y)$ comme coin
        }
    }
    \Return{Liste des points marqués}
\end{algorithm}


\end{frame}


%++++++++++++++++++++++++++++++++++++++++++++++++
\subsection{Appariement}
%------------------------------------------------
\begin{frame}
\frametitle{Titre de la slide sans lettre descendant sous la baseline}
	Pour régler ce problème, utiliser la commande \lin{\esp} à la fin du titre, \textit{Cf.} slide suivante
\end{frame}


