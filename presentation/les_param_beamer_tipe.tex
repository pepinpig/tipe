\usetheme{Montpellier}
\usecolortheme{seahorse}


%--------marges
\setbeamersize{text margin left= 0.7cm}
\setbeamersize{text margin right= 0.7cm}

%--------tête et pieds
\setbeamertemplate{navigation symbols}{}
\setbeamertemplate{footline}[frame number]
\setbeamertemplate{headline}{
  %la premiere ligne
  	\begin{beamercolorbox}[ht=0.42cm, vmode]{section in head/foot}
	\hspace{0.4cm} \insertshorttitle 
	\hspace*{0.1cm}- \initiales - {\insertshortdate}
	\vspace*{0.08cm} 
  	\end{beamercolorbox}
  %la deuxième ligne
	\begin{beamercolorbox}[ht=0.4cm, vmode]{subsection in head/foot}
	%titre de la section si elle est pas 0
		\ifnum\value{section}=0{} 
		\else{ \hspace{0.8cm} \thesection - \insertsectionhead }
		\fi
	%séparateur + titre de la sous-section si elle est pas 0
		\ifnum\value{subsection}=0{} 
		\else{ 
			\hspace*{0.1cm} \couleur{$\bullet$} \hspace*{0.1cm} 
			\thesection.\thesubsection \, \insertsubsectionhead
		}
		\fi
		\vspace*{0.12cm}
\end{beamercolorbox}
%\vspace*{-0.03cm} %pour pas qu'il y ait d'espace avec la ligne de frametitle
 }
%\setbeamertemplate{frametitleheigth}{4cm}
\setbeamertemplate{frametitle}{
	\vspace*{-0.04cm} 
	\begin{beamercolorbox}[ht=0.8cm,wd=\paperwidth, vmode]{frametitle}
		\hspace{0.3cm} \insertframetitle \vspace*{0.1cm}
	\end{beamercolorbox}
}

%commande pour ajuster l'alignement vertical des titres sans lettres descendantes
%\newcommand{\esp}{\\[0.1cm]} %--version qui marche sans le package minted
\newcommand{\esp}{\\[-0.5cm]} %--version qui marche avec le package minted


%--------couleurs
\setbeamercolor{structure}{fg=turquoiseFonce!70!black} 

\setbeamercolor{block title}{fg=turquoiseFonce!70!black,bg=vertdEau}
\setbeamercolor{block body}{bg=vertdEau!10!white}

\setbeamercolor{block title alerted}{bg=vertdEau!85!white,fg=turquoiseFonce!80!black}
\setbeamercolor{block body alerted}{bg=vertdEau!8!white}
%\setbeamercolor{alerted text}{fg=red}

\setbeamercolor{block title example}{bg=vertdEau,fg=turquoiseFonce}
\setbeamercolor{block body example}{bg=vertdEau!10!white}
\setbeamercolor{example text}{fg=blue!20!turquoise}

%-------- TOC
\setbeamertemplate{section in toc}[sections numbered]

%-----------------------------------------------
%Plan qui s'affiche au début de chaque section %|
\AtBeginSection[]{                             %|
\begin{frame}[plain]                           %|
\frametitle{Plan\\[0.1cm]}                     %|
\tableofcontents[                              %|
currentsection,                                %|
hideothersubsections,                          %|
subsubsectionstyle=hide]                       %|
\addtocounter{framenumber}{-1}                 %|
\end{frame}}                                   %|
%-----------------------------------------------




%-------- commande pour les ref sur les slides
\newcommand{\bandeauREF}[1]{
\noindent\makebox[\textwidth][l]{%
\hspace{-\dimexpr\oddsidemargin+1in}%
\colorbox{expli!20!white}{%
\parbox{\dimexpr\paperwidth-2\fboxsep}{
\footnotesize\textcolor{expli!80!black}{#1}
}}}}
% Commandes utilitaires
\tikzset{every picture/.style={line width=0.75pt}} %set default line width to 0.75pt        
\newcommand{\imageFrame}{
  \draw [line width=0.75] (261.98,12.03) -- (262.02,142.84) -- (150.52,236.47) -- (150.47,105.66) -- cycle ;
}
\newcommand{\lin}[1]{\texttt{#1}} % Évite minted si pas nécessaire
\newcommand{\flch}{\item[$\rightarrow$]}
\newcommand{\dc}{{\usebeamercolor[fg]{structure}$\hookrightarrow$}}
\newcommand{\ok}{\textcolor{green}{\checkmark}}
\newcommand{\point}{{\usebeamercolor[fg]{structure}$\bullet\enskip$}}
\newcommand{\Point}{\point}
\newcommand{\couleur}[1]{{\usebeamercolor[fg]{structure}#1}}
\newcommand{\important}[1]{\couleur{\textbf{#1}}}
\newcommand{\remarque}[1]{\textit{\textrm{#1}}}
\newcommand{\cmark}{\ding{51}\xspace} % check ✓
\newcommand{\xmark}{\ding{55}\xspace} % cross ✗