%------------------------------------------
\subsection{Résolution}
%------------------------------------------------
\begin{frame}{Équations sans \(\lambda\) et système matriciel}
  
\[
\resizebox{\textwidth}{!}{$
\left(
\begin{array}{cccccccccccc}
    x_{C}^{(1)} & y_{C}^{(1)} & z_{C}^{(1)} & 1 & 0 & 0 & 0 & 0 & -u^{( 1)} x{C}^{(1)} & -u^{( 1)}y{C}^{(1)} & -u^{( 1)}z_{C}^{(1)} & -u^{(1)}\\
0 & 0 & 0 & 0 & x_{C}^{( 1)} & y_{C}^{( 1)} & z_{C}^{( 1)} & 1 & -v^{( 1)} x_{C}^{( 1)} & -v^{( 1)} y_{C}^{( 1)} & -v^{( 1)} z_{C}^{( 1)} & -v^{( 1)}\\
\vdots  & \vdots  & \vdots  & \vdots  & \vdots  & \vdots  & \vdots  & \vdots  & \vdots  & \vdots  & \vdots  & \vdots \\
x_{C}^{( i)} & y_{C}^{( i)} & z_{C}^{( i)} & 1 & 0 & 0 & 0 & 0 & -u^{( i)} x_{C}^{( i)} & -u^{( i)} y_{C}^{( i)} & -u^{( i)} z_{C}^{( i)} & -u^{( i)}\\
0 & 0 & 0 & 0 & x_{C}^{( i)} & y_{C}^{( i)} & z_{C}^{( i)} & 1 & -v^{( i)} x_{C}^{( i)} & -v^{( i)} y_{C}^{( i)} & -v^{( i)} z_{C}^{( i)} & -v^{( i)}\\
\vdots  & \vdots  & \vdots  & \vdots  & \vdots  & \vdots  & \vdots  & \vdots  & \vdots  & \vdots  & \vdots  & \vdots \\
x_{C}^{( 6)} & y_{C}^{( 6)} & z_{C}^{( 6)} & 1 & 0 & 0 & 0 & 0 & -u^{( 6)} x_{C}^{( 6)} & -u^{( 6)} y_{C}^{( 6)} & -u^{( 6)} z_{C}^{( 6)} & -u^{( 6)}\\
0 & 0 & 0 & 0 & x_{C}^{( 6)} & y_{C}^{( 6)} & z_{C}^{( 6)} & 1 & -v^{( 6)} x_{C}^{( 6)} & -v^{( 6)} y_{C}^{( 6)} & -v^{( 6)} z_{C}^{( 6)} & -v^{( 6)}
\end{array}
\right)
\begin{pmatrix}
p_{11}\\
p_{12}\\
p_{13}\\
p_{14}\\
p_{21}\\
p_{22}\\
p_{23}\\
p_{24}\\
p_{31}\\
p_{32}\\
p_{33}\\
p_{34}
\end{pmatrix} =\begin{pmatrix}
0\\
0\\
0\\
0\\
0\\
0\\
0\\
0\\
0\\
0\\
0\\
0
\end{pmatrix}
$}
\]
\end{frame}

\begin{frame}
\frametitle{Système d’optimisation à contrainte unitaire}

On souhaite résoudre le système en évitant la solution triviale \( P = 0 \).  
\pause

Sachant que la matrice \( P \) ne peut être déterminée qu’à un facteur près, on peut imposer :
\[
\|P\|^2 = 1
\]
et reformuler le système comme un problème d’optimisation :
\pause
\[
\min_{\|p\|^2 = 1} \|Ap\|^2 = \min_{\|p\|^2 = 1} p^T A^T A p
\]
\pause
On introduit les fonctions :
\begin{itemize}
  \item \( f(p) = p^T A^T A p \)
  \item \( g(p) = p^T p - 1 \)
\end{itemize}
\pause
D’après le théorème d’optimisation sous contrainte (Lagrange), au point optimal \( P^* \), il existe \( \lambda \in \mathbb{R} \) tel que :
\[
\nabla f(P^*) = \lambda \nabla g(P^*)
\]
\end{frame}


\begin{frame}{Lien avec les valeurs propres}

Posons \( M = A^T A \).  
Alors :
\[
f(p) = \sum_{i=1}^n \sum_{j=1}^n p_i M_{ij} p_j
\]
\pause

Comme \( M \) est symétrique :
\[
\frac{\partial f}{\partial p} = 2Mp
\quad \text{et} \quad
\frac{\partial g}{\partial p} = 2p
\]
\pause

On a donc :
\[
\frac{\partial f}{\partial p} = \lambda \frac{\partial g}{\partial p}
\quad \Rightarrow \quad
\boxed{A^T A p = \lambda p}
\]
\pause

C’est une équation aux valeurs propres :
\begin{itemize}
  \item \( p \) est un vecteur propre de \( A^T A \)
  \item \( \lambda \) est la valeur propre associée
\end{itemize}
\end{frame}

\begin{frame}{Minimisation par SVD}

\begin{itemize}
  \item<1-> Pour minimiser \( \|AP\|^2 = p^T A^T A p \),  
  il faut choisir \( p \) comme vecteur propre associé à la plus petite valeur propre de \( A^T A \).
  
  \item<2-> Ce problème se résout efficacement par la décomposition en valeurs singulières (SVD) de la matrice \( A \).

  \item<3-> Une solution consiste à calculer :
  \[
  A = U \Sigma V^T
  \]
  où :
  \begin{itemize}
    \item<4-> \( U \in \mathbb{R}^{m \times m} \) est une matrice orthogonale (vecteurs singuliers à gauche),
    \item<5-> \( \Sigma \in \mathbb{R}^{m \times n} \) est une matrice diagonale contenant les valeurs singulières \( \sigma_i \),
    \item<6-> \( V \in \mathbb{R}^{n \times n} \) est une matrice orthogonale (vecteurs singuliers à droite).
  \end{itemize}
  
  \item<7-> Le vecteur \( p \) recherché correspond à la dernière colonne de \( V \), associée à la plus petite valeur singulière.
\end{itemize}
\end{frame}

%++++++++++++++++++++++++++++++++++++++++++++++++
\subsection{Implémentation}

\begin{frame}{Décomposition SVD : idée générale}
  \begin{minipage}[t][0.8\textheight][t]{0.45\textwidth}
    \vspace*{\fill}
    \textbf{Étapes principales :}
    \begin{itemize}
      \item<1-> \textbf{1.} Calculer \( A^T A \)
      \item<2-> \textbf{2.} Appliquer QR à \( A^T A \)
      \item<3-> \textbf{3.} Extraire valeurs propres \( \sigma_i^2 \)
      \item<4-> \textbf{4.} Déduire \( V \) puis \( U = \frac{1}{\sigma} A v \)
    \end{itemize}
    \vspace*{\fill}
  \end{minipage}
  \hfill
  \begin{minipage}[t]{0.52\textwidth}
    \vspace*{\fill}
    \only<1>{
      \textbf{1. Produit symétrique :}
      \[
        A^T A \text{ est symétrique, de taille } n \times n
      \]
      \vspace{1em}
      On peut chercher ses valeurs propres via QR.
    }

    \only<2>{
      \textbf{2. Algorithme QR :}
      \[
        A_k = R_k Q_k \Rightarrow A_{k+1} = Q_k^T A_k Q_k
      \]
      \vspace{1em}
      Converge vers matrice diagonale contenant \( \sigma_i^2 \).
    }

    \only<3>{
      \textbf{3. Valeurs singulières :}
      \[
        \sigma_i = \sqrt{\lambda_i}, \quad \text{avec } \lambda_i \text{ valeur propre}
      \]
      \vspace{1em}
      On range les \( \sigma_i \) du plus grand au plus petit.
    }

    \only<4>{
      \textbf{4. Vecteurs singuliers :}
   \[
  v_i \text{ propre de } A^T A \quad \Rightarrow \quad u_i = \textstyle\frac{1}{\sigma_i} A v_i
\]

      \vspace{1em}
      On normalise chaque \( u_i \) pour obtenir \( U \).
    }
    \vspace*{\fill}
  \end{minipage}
\end{frame}

\begin{frame}
    \scriptsize
    \begin{algorithm}[H]
\DontPrintSemicolon
\Input{$A \in \mathbb{R}^{m \times n}$}
\Output{$Q \in \mathbb{R}^{m \times n}$, $R \in \mathbb{R}^{n \times n}$ tels que $A = QR$}
\BlankLine

\For{$j \gets 1$ \KwTo $n$}{
  $v_j \gets A_{:,j}$ \tcc*[r]{Copie de la $j^{\text{\`e}me}$ colonne de $A$}
  \For{$i \gets 1$ \KwTo $j-1$}{
    $R_{i,j} \gets \langle Q_{:,i}, A_{:,j} \rangle$ \;
    $v_j \gets v_j - R_{i,j} Q_{:,i}$ \;
  }
  $R_{j,j} \gets \|v_j\|$ \;
  \Si{$R_{j,j} > \varepsilon$}{
    $Q_{:,j} \gets \frac{v_j}{R_{j,j}}$
  }
  \Sinon{
    $Q_{:,j} \gets 0$
  }
}
\Return{$Q, R$}
\caption{Décomposition QR via Gram-Schmidt}
\end{algorithm}
\end{frame}

%------------------------

\begin{frame}
\scriptsize
\begin{algorithm}[H]
\DontPrintSemicolon
\Input{$B \in \mathbb{R}^{n \times n}$ symétrique}
\Output{$\Sigma^2$, $V$ tels que $B = V \Sigma^2 V^T$}

\BlankLine
$Q_{\text{acc}} \gets I_n$ \tcc*[r]{Accumule les produits de $Q$}
\BlankLine
%$\varepsilon \gets 10^{-12}$ \;
$\delta \gets 1$, $k_{\text{max}} \gets 1000$, $k \gets 0$\;

\Tq{$\delta > 10^{-9}$ et $k < k_{\text{max}}$}{
  $Q, R \gets$ décomposition QR de $B$\;
  $B_{\text{nouveau}} \gets R \cdot Q$\;
  $Q_{\text{acc}} \gets Q_{\text{acc}} \cdot Q$\;

  $\delta \gets \sum_i |\text{diag}(B_{\text{nouveau}})_i - \text{diag}(B)_i|$\;

  $A \gets B_{\text{nouveau}}$\;
  $k \gets k + 1$\;
}

\BlankLine
\For{$i=1$ à $n$}{
  \Si{$1[i, i] > \varepsilon$}{
    $\Sigma^2[i, i] \gets V[i, i]$
  }
  \Sinon{
    $\Sigma^2[i, i] \gets 0$
  }
}

\Return{$\Sigma^2$,$Q_{\text{acc}}$ }

\caption{algorithme QR}
\end{algorithm}
\end{frame}

\begin{frame}
\scriptsize  
\begin{algorithm}[H]
\DontPrintSemicolon
\Input{$A \in \mathbb{R}^{m \times n}$}
\Output{$U, \Sigma, V$ tels que $A \approx U \Sigma V^T$}

\BlankLine
$A^T \gets$ transposée de $A$\;
$A^T A \gets A^T \cdot A$ \tcc*[r]{Symétrique et définie positive}

\BlankLine
\texttt{algorithme\_QR}($A^T A$, $\Sigma^2$, $V$) \tcc*[r]{$\Sigma^2$ diagonale, $V$ orthogonale}

\BlankLine
\For{$i \gets 1$ \KwTo $n$}{
    $\sigma^2 \gets \Sigma^2[i, i]$\;
    \Si{$\sigma^2 < 10^{-12}$}{
        \textbf{continuer} \tcc*[r]{Ignorer valeur singulière nulle}
    }

    $\sigma \gets \sqrt{\sigma^2}$\;
    $\Sigma[i, i] \gets \sigma$ \tcc*[r]{Met à jour la vraie valeur singulière}

    $v_i \gets$ $i^{\text{e}}$ colonne de $V$\;
    $u_i \gets A \cdot v_i$ \tcc*[r]{$u_i$ non normalisé}
    $u_i \gets u_i / \sigma$\;
    normaliser $u_i$\;
    insérer $u_i$ comme $i^{\text{e}}$ colonne de $U$\;
}

\caption{SVD via algorithme QR sur \( A^T A \)}
\end{algorithm}
\end{frame}

