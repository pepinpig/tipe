%------------------------------------------
\subsection{Résolution système}
%------------------------------------------------

\begin{frame}{Résolution système}
  
\[
\resizebox{\textwidth}{!}{$
\left(
\begin{array}{cccccccccccc}
    x_{C}^{(1)} & y_{C}^{(1)} & z_{C}^{(1)} & 1 & 0 & 0 & 0 & 0 & -u^{( 1)} x{C}^{(1)} & -u^{( 1)}y{C}^{(1)} & -u^{( 1)}z_{C}^{(1)} & -u^{(1)}\\
0 & 0 & 0 & 0 & x_{C}^{( 1)} & y_{C}^{( 1)} & z_{C}^{( 1)} & 1 & -v^{( 1)} x_{C}^{( 1)} & -v^{( 1)} y_{C}^{( 1)} & -v^{( 1)} z_{C}^{( 1)} & -v^{( 1)}\\
\vdots  & \vdots  & \vdots  & \vdots  & \vdots  & \vdots  & \vdots  & \vdots  & \vdots  & \vdots  & \vdots  & \vdots \\
x_{C}^{( i)} & y_{C}^{( i)} & z_{C}^{( i)} & 1 & 0 & 0 & 0 & 0 & -u^{( i)} x_{C}^{( i)} & -u^{( i)} y_{C}^{( i)} & -u^{( i)} z_{C}^{( i)} & -u^{( i)}\\
0 & 0 & 0 & 0 & x_{C}^{( i)} & y_{C}^{( i)} & z_{C}^{( i)} & 1 & -v^{( i)} x_{C}^{( i)} & -v^{( i)} y_{C}^{( i)} & -v^{( i)} z_{C}^{( i)} & -v^{( i)}\\
\vdots  & \vdots  & \vdots  & \vdots  & \vdots  & \vdots  & \vdots  & \vdots  & \vdots  & \vdots  & \vdots  & \vdots \\
x_{C}^{( 6)} & y_{C}^{( 6)} & z_{C}^{( 6)} & 1 & 0 & 0 & 0 & 0 & -u^{( 6)} x_{C}^{( 6)} & -u^{( 6)} y_{C}^{( 6)} & -u^{( 6)} z_{C}^{( 6)} & -u^{( 6)}\\
0 & 0 & 0 & 0 & x_{C}^{( 6)} & y_{C}^{( 6)} & z_{C}^{( 6)} & 1 & -v^{( 6)} x_{C}^{( 6)} & -v^{( 6)} y_{C}^{( 6)} & -v^{( 6)} z_{C}^{( 6)} & -v^{( 6)}
\end{array}
\right)
\begin{pmatrix}
p_{11}\\
p_{12}\\
p_{13}\\
p_{14}\\
p_{21}\\
p_{22}\\
p_{23}\\
p_{24}\\
p_{31}\\
p_{32}\\
p_{33}\\
p_{34}
\end{pmatrix} =\begin{pmatrix}
0\\
0\\
0\\
0\\
0\\
0\\
0\\
0\\
0\\
0\\
0\\
0
\end{pmatrix}
$}
\]
\end{frame}

\begin{frame}{Problème d’optimisation sous contrainte}
\note{
Résolution classique -> solution triviale nulle 
Le système est homogène -> on peut fixé la norme du vecteur $p$ à 1.
On transforme le problème en une minimisation sous contrainte.

En appliquant les multiplicateurs de Lagrange, on obtient une condition d’optimalité : le gradient de la fonction est proportionnel au gradient de la contrainte, ce qui donne une équation aux valeurs propres.
}
Solution triviale \( P = 0 \).  
On impose : \( \|p\|^2 = 1 \)

\vspace{0.8em}
On reformule le problème comme une minimisation sous contrainte :
\[
\min_{\|p\| = 1} \|Ap\|^2
\quad \Leftrightarrow \quad
\min_{\|p\| = 1} p^T A^T A p
\]
\pause
\textbf{Propriété clé :} au minimum, \( p \) vérifie :
\[
\boxed{A^T A p = \lambda p}
\]
  \hyperlink{optimisation-appendix}{
    \beamerbutton{Compléments calcul}
  }
\vspace{1em}
\pause
\[
\boxed{\min \|Ap\| \Rightarrow p = \text{un v.p. associé à la plus petite va.p. de } A^T A}
\]

\end{frame}

\begin{frame}{Décomposition en valeurs singulières (SVD)}
\begin{itemize}
  \item On factorise \( A \) : 
  \[
  A = U \Sigma V^T
  \]
  \pause
  \item \( U \) : matrice orthogonale de \( \mathbb{R}^{m \times m} \)
  \pause
  \item \( \Sigma \) : matrice diagonale \( \mathbb{R}^{m \times n} \) contenant les valeurs singulières \( \sigma_1 \geq \sigma_2 \geq \dots \geq \sigma_n \geq 0 \)
  \pause
  \item \( V \) : matrice orthogonale de \( \mathbb{R}^{n \times n} \), ses colonnes sont les vecteurs propres de \( A^T A \), tels que :
  \[
  A^T A v_i = \sigma_i^2 v_i
  \]
\end{itemize}

\note{
La décomposition en valeurs singulières, ou SVD, permet de factoriser n’importe quelle matrice \( A \) en trois matrices : deux orthogonales \( U \) et \( V \), et une diagonale \( \Sigma \) contenant les valeurs singulières.

Ces valeurs sont les racines carrées des valeurs propres de \( A^T A \), classées par ordre décroissant.

\( U \) contient les vecteurs propres de \( AA^T \), \( V \) ceux de \( A^T A \). Autrement dit, chaque colonne \( v_i \) de \( V \) vérifie : \( A^T A v_i = \sigma_i^2 v_i \).

La dernière colonne de \( V \), associée à la plus petite valeur singulière, donne donc directement le vecteur \( p \) recherché.

C’est une méthode très stable numériquement.
}
\end{frame}


%++++++++++++++++++++++++++++++++++++++++++++++++
\subsection{Implémentation}

\begin{frame}{Algorithme QR : principe de convergence}
\scriptsize
\begin{minipage}[t][0.8\textheight][t]{\textwidth}
\vspace*{\fill}
\textbf{Décomposition QR pour l’itération :}
\begin{itemize}
  \item Si \( A = QR \), avec :
  \begin{itemize}
    \scriptsize
    \item \( Q \) une matrice orthogonale (\( Q^T Q = I \)),
    \item \( R \) une matrice triangulaire supérieure,
  \end{itemize}
  \pause
  alors on définit :
  \[
    A' = RQ
  \]
  \pause
  \vspace{-1em}
  \item \( A' \) est \textbf{semblable} à \( A \), car :
  \[
    Q^T A Q = Q^T Q R Q = R Q = A'
  \]
\end{itemize}
\pause
\vspace{-0.3em}
\textbf{Construction de la suite :}
\[
\begin{cases}
A_0 = A \\
\text{À chaque itération : } A_k = Q_k R_k \\
A_{k+1} = R_k Q_k
\end{cases}
\]
\vspace{-0.3em}
\pause
\begin{itemize}
  \item Cette suite \( (A_k) \) converge vers une matrice triangulaire.
  \item Les valeurs propres de \( A_k \) sont identiques à celles de \( A \).
\end{itemize}

\vspace*{\fill}
\end{minipage}

\end{frame}


\begin{frame}{Résolution avec SVD}
\hyperlink{SVD-appendix}{\beamerbutton{Pseudo-code détaillé}}
\begin{itemize}
  \item Produit symétrique : \( A^T A \in \mathbb{R}^{n \times n} \)
  \pause
  \item Décomposition QR : 
  \[
    A^TA = Q_{k-1} \cdots Q_0 \, A_k \, Q_0^T \cdots Q_{k-1}^T
  \]
  \pause
  \item Matrice des vecteurs propres : 
  \[
    V = Q_0 Q_1 \cdots Q_{k-1}
  \]
  \pause
  \item Extraction du vecteur propre associé à la plus petite valeur propre non nulle :
  \[
    v_{i_0} \in V
  \]
\end{itemize}

\end{frame}
