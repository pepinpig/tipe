\section*[Appendix]{Appendix}
%===============================================
\begin{frame}{Exemple Moravec}
  \label{moravec-appendix}
  \centering
  \begin{tikzpicture}
    \node[anchor=south west,inner sep=0] (image) at (0,0) {\includegraphics[width=0.5\textwidth]{capture/rub_rouge.jpg}};
    \only<2>{\draw[red, thick] (1,4.5) rectangle (2,5.5);}     % rectangle cible
  \end{tikzpicture}
\end{frame}

\begin{frame}
  \note{
    Voici une animation de l’évolution de la détection avec l’algorithme de Moravec.

    On visualise étape par étape comment les coins apparaissent au fur et à mesure.

    Ce genre de visualisation aide à comprendre la sensibilité de l’algorithme selon l’image et les paramètres.
  }
  \centering
  \begin{overlayarea}{0.9\linewidth}{4cm}
    \vspace*{-1cm}
    \hspace*{-1cm}
    \begin{tikzpicture}[x=0.75pt,y=0.75pt,yscale=-1,xscale=1, scale=2]
      \only<1>{
%Shape: Rectangle [id:dp855744465122586] 
\draw  [color={rgb, 255:red, 0; green, 0; blue, 0 }  ,draw opacity=1 ][fill={rgb, 255:red, 200; green, 43; blue, 43 }  ,fill opacity=0.23 ] (279.6,109) -- (299.6,109) -- (299.6,129) -- (279.6,129) -- cycle ;
%Shape: Rectangle [id:dp8201664290630494] 
\draw  [color={rgb, 255:red, 0; green, 0; blue, 0 }  ,draw opacity=1 ][fill={rgb, 255:red, 200; green, 43; blue, 43 }  ,fill opacity=0.64 ] (299.6,129) -- (319.6,129) -- (319.6,149) -- (299.6,149) -- cycle ;
%Shape: Rectangle [id:dp8273756646818722] 
\draw  [color={rgb, 255:red, 0; green, 0; blue, 0 }  ,draw opacity=1 ][fill={rgb, 255:red, 200; green, 43; blue, 43 }  ,fill opacity=0.64 ] (279.6,129) -- (299.6,129) -- (299.6,149) -- (279.6,149) -- cycle ;
%Shape: Rectangle [id:dp6720526871451368] 
\draw  [color={rgb, 255:red, 0; green, 0; blue, 0 }  ,draw opacity=1 ][fill={rgb, 255:red, 200; green, 43; blue, 43 }  ,fill opacity=0.69 ] (299.6,149) -- (319.6,149) -- (319.6,169) -- (299.6,169) -- cycle ;
%Shape: Rectangle [id:dp21990750654974778] 
\draw  [color={rgb, 255:red, 0; green, 0; blue, 0 }  ,draw opacity=1 ][fill={rgb, 255:red, 200; green, 43; blue, 43 }  ,fill opacity=0.68 ] (319.6,129) -- (339.6,129) -- (339.6,149) -- (319.6,149) -- cycle ;
%Shape: Rectangle [id:dp07109720325756996] 
\draw  [color={rgb, 255:red, 0; green, 0; blue, 0 }  ,draw opacity=1 ][fill={rgb, 255:red, 200; green, 43; blue, 43 }  ,fill opacity=0.32 ] (299.6,89) -- (319.6,89) -- (319.6,109) -- (299.6,109) -- cycle ;
%Shape: Rectangle [id:dp9795028573738183] 
\draw  [color={rgb, 255:red, 0; green, 0; blue, 0 }  ,draw opacity=1 ][fill={rgb, 255:red, 200; green, 43; blue, 43 }  ,fill opacity=0.64 ] (299.6,109) -- (319.6,109) -- (319.6,129) -- (299.6,129) -- cycle ;
%Shape: Rectangle [id:dp5174606270935156] 
\draw  [color={rgb, 255:red, 0; green, 0; blue, 0 }  ,draw opacity=1 ][fill={rgb, 255:red, 200; green, 43; blue, 43 }  ,fill opacity=0.23 ] (239.6,69) -- (259.6,69) -- (259.6,89) -- (239.6,89) -- cycle ;
%Shape: Rectangle [id:dp003251435331488195] 
\draw  [color={rgb, 255:red, 0; green, 0; blue, 0 }  ,draw opacity=1 ][fill={rgb, 255:red, 200; green, 43; blue, 43 }  ,fill opacity=0.21 ] (259.6,89) -- (279.6,89) -- (279.6,109) -- (259.6,109) -- cycle ;
%Shape: Rectangle [id:dp12978711061445025] 
\draw  [color={rgb, 255:red, 0; green, 0; blue, 0 }  ,draw opacity=1 ][fill={rgb, 255:red, 200; green, 43; blue, 43 }  ,fill opacity=0.25 ] (239.6,89) -- (259.6,89) -- (259.6,109) -- (239.6,109) -- cycle ;
%Shape: Rectangle [id:dp11386610015932308] 
\draw  [color={rgb, 255:red, 0; green, 0; blue, 0 }  ,draw opacity=1 ][fill={rgb, 255:red, 200; green, 43; blue, 43 }  ,fill opacity=0.28 ] (239.6,109) -- (259.6,109) -- (259.6,129) -- (239.6,129) -- cycle ;
%Shape: Rectangle [id:dp31730737510449536] 
\draw  [color={rgb, 255:red, 0; green, 0; blue, 0 }  ,draw opacity=1 ][fill={rgb, 255:red, 200; green, 43; blue, 43 }  ,fill opacity=0.66 ] (259.6,129) -- (279.6,129) -- (279.6,149) -- (259.6,149) -- cycle ;
%Shape: Rectangle [id:dp0039535950874963754] 
\draw  [color={rgb, 255:red, 0; green, 0; blue, 0 }  ,draw opacity=1 ][fill={rgb, 255:red, 200; green, 43; blue, 43 }  ,fill opacity=0.77 ] (279.6,149) -- (299.6,149) -- (299.6,169) -- (279.6,169) -- cycle ;
%Shape: Rectangle [id:dp5148350575823291] 
\draw  [color={rgb, 255:red, 0; green, 0; blue, 0 }  ,draw opacity=1 ][fill={rgb, 255:red, 200; green, 43; blue, 43 }  ,fill opacity=0.24 ] (259.6,109) -- (279.6,109) -- (279.6,129) -- (259.6,129) -- cycle ;
%Shape: Rectangle [id:dp36043744827539115] 
\draw  [color={rgb, 255:red, 0; green, 0; blue, 0 }  ,draw opacity=1 ][fill={rgb, 255:red, 200; green, 43; blue, 43 }  ,fill opacity=0.26 ] (259.6,69) -- (279.6,69) -- (279.6,89) -- (259.6,89) -- cycle ;
%Shape: Rectangle [id:dp3332412201524424] 
\draw  [color={rgb, 255:red, 0; green, 0; blue, 0 }  ,draw opacity=1 ][fill={rgb, 255:red, 200; green, 43; blue, 43 }  ,fill opacity=0.22 ] (279.6,89) -- (299.6,89) -- (299.6,109) -- (279.6,109) -- cycle ;
%Shape: Rectangle [id:dp49757845918426946] 
\draw  [color={rgb, 255:red, 0; green, 0; blue, 0 }  ,draw opacity=1 ][fill={rgb, 255:red, 200; green, 43; blue, 43 }  ,fill opacity=0.29 ] (279.6,69) -- (299.6,69) -- (299.6,89) -- (279.6,89) -- cycle ;
%Shape: Rectangle [id:dp1948054383738922] 
\draw  [color={rgb, 255:red, 0; green, 0; blue, 0 }  ,draw opacity=1 ][fill={rgb, 255:red, 200; green, 43; blue, 43 }  ,fill opacity=0.28 ] (299.6,69) -- (319.6,69) -- (319.6,89) -- (299.6,89) -- cycle ;
%Shape: Rectangle [id:dp5908482157161481] 
\draw  [color={rgb, 255:red, 0; green, 0; blue, 0 }  ,draw opacity=1 ][fill={rgb, 255:red, 200; green, 43; blue, 43 }  ,fill opacity=0.65 ] (319.6,109) -- (339.6,109) -- (339.6,129) -- (319.6,129) -- cycle ;
%Shape: Rectangle [id:dp8244027566618196] 
\draw  [color={rgb, 255:red, 0; green, 0; blue, 0 }  ,draw opacity=1 ][fill={rgb, 255:red, 200; green, 43; blue, 43 }  ,fill opacity=0.64 ] (239.6,129) -- (259.6,129) -- (259.6,149) -- (239.6,149) -- cycle ;
%Shape: Rectangle [id:dp29418903051156065] 
\draw  [color={rgb, 255:red, 0; green, 0; blue, 0 }  ,draw opacity=1 ][fill={rgb, 255:red, 200; green, 43; blue, 43 }  ,fill opacity=0.73 ] (239.6,149) -- (259.6,149) -- (259.6,169) -- (239.6,169) -- cycle ;
%Shape: Rectangle [id:dp03059484266780088] 
\draw  [color={rgb, 255:red, 0; green, 0; blue, 0 }  ,draw opacity=1 ][fill={rgb, 255:red, 200; green, 43; blue, 43 }  ,fill opacity=0.74 ] (259.6,149) -- (279.6,149) -- (279.6,169) -- (259.6,169) -- cycle ;
%Shape: Rectangle [id:dp15824841679231605] 
\draw  [color={rgb, 255:red, 0; green, 0; blue, 0 }  ,draw opacity=1 ][fill={rgb, 255:red, 200; green, 43; blue, 43 }  ,fill opacity=0.68 ] (319.6,149) -- (339.6,149) -- (339.6,169) -- (319.6,169) -- cycle ;
%Shape: Rectangle [id:dp5219374564422655] 
\draw  [color={rgb, 255:red, 0; green, 0; blue, 0 }  ,draw opacity=1 ][fill={rgb, 255:red, 200; green, 43; blue, 43 }  ,fill opacity=0.7 ] (319.6,69) -- (339.6,69) -- (339.6,89) -- (319.6,89) -- cycle ;
%Shape: Rectangle [id:dp41832573855226485] 
\draw  [color={rgb, 255:red, 0; green, 0; blue, 0 }  ,draw opacity=1 ][fill={rgb, 255:red, 200; green, 43; blue, 43 }  ,fill opacity=0.63 ] (319.6,89) -- (339.6,89) -- (339.6,109) -- (319.6,109) -- cycle ;
%Shape: Rectangle [id:dp4512036699306804] 
\draw  [color={rgb, 255:red, 255; green, 255; blue, 255 }  ,draw opacity=1 ] (209,54.5) -- (451,54.5) -- (451,190.5) -- (209,190.5) -- cycle ;
%Shape: Rectangle [id:dp30277857918883444] 
\draw  [color={rgb, 255:red, 222; green, 195; blue, 14 }  ,draw opacity=1 ][fill={rgb, 255:red, 192; green, 177; blue, 255 }  ,fill opacity=0.34 ] (259.6,335) -- (279.6,335) -- (279.6,355) -- (259.6,355) -- cycle ;

% Text Node
\draw (304.8,136) node [anchor=north west][inner sep=0.75pt]  [font=\footnotesize] [align=left] {64};
% Text Node
\draw (324.8,155.2) node [anchor=north west][inner sep=0.75pt]  [font=\footnotesize] [align=left] {69};
% Text Node
\draw (305.2,115.6) node [anchor=north west][inner sep=0.75pt]  [font=\footnotesize] [align=left] {65};
% Text Node
\draw (305.2,95.6) node [anchor=north west][inner sep=0.75pt]  [font=\footnotesize] [align=left] {32};
% Text Node
\draw (285.2,95.6) node [anchor=north west][inner sep=0.75pt]  [font=\footnotesize] [align=left] {22};
% Text Node
\draw (325.2,115.6) node [anchor=north west][inner sep=0.75pt]  [font=\footnotesize] [align=left] {65};
% Text Node
\draw (244.4,76) node [anchor=north west][inner sep=0.75pt]  [font=\footnotesize] [align=left] {23};
% Text Node
\draw (265.2,96) node [anchor=north west][inner sep=0.75pt]  [font=\footnotesize] [align=left] {21};
% Text Node
\draw (244.8,95.6) node [anchor=north west][inner sep=0.75pt]  [font=\footnotesize] [align=left] {25};
% Text Node
\draw (264.8,116.4) node [anchor=north west][inner sep=0.75pt]  [font=\footnotesize] [align=left] {24};
% Text Node
\draw (284.8,135.6) node [anchor=north west][inner sep=0.75pt]  [font=\footnotesize] [align=left] {64};
% Text Node
\draw (304.8,155.2) node [anchor=north west][inner sep=0.75pt]  [font=\footnotesize] [align=left] {69};
% Text Node
\draw (264.4,75.6) node [anchor=north west][inner sep=0.75pt]  [font=\footnotesize] [align=left] {26};
% Text Node
\draw (285.2,75.6) node [anchor=north west][inner sep=0.75pt]  [font=\footnotesize] [align=left] {29};
% Text Node
\draw (304.8,75.2) node [anchor=north west][inner sep=0.75pt]  [font=\footnotesize] [align=left] {28};
% Text Node
\draw (324.8,95.2) node [anchor=north west][inner sep=0.75pt]  [font=\footnotesize] [align=left] {63};
% Text Node
\draw (324.8,76.8) node [anchor=north west][inner sep=0.75pt]  [font=\footnotesize] [align=left] {70};
% Text Node
\draw (244.8,115.6) node [anchor=north west][inner sep=0.75pt]  [font=\footnotesize] [align=left] {28};
% Text Node
\draw (264.8,135.6) node [anchor=north west][inner sep=0.75pt]  [font=\footnotesize] [align=left] {66};
% Text Node
\draw (284.8,155.2) node [anchor=north west][inner sep=0.75pt]  [font=\footnotesize] [align=left] {77};
% Text Node
\draw (244,136) node [anchor=north west][inner sep=0.75pt]  [font=\footnotesize] [align=left] {63};
% Text Node
\draw (264,155.2) node [anchor=north west][inner sep=0.75pt]  [font=\footnotesize] [align=left] {74};
% Text Node
\draw (245.2,155.6) node [anchor=north west][inner sep=0.75pt]  [font=\footnotesize] [align=left] {73};
% Text Node
\draw (324.8,136.4) node [anchor=north west][inner sep=0.75pt]  [font=\footnotesize] [align=left] {68};
% Text Node
\draw (285.2,116.4) node [anchor=north west][inner sep=0.75pt]  [font=\footnotesize] [align=left] {23};

\draw (355,80) node [anchor=north west][inner sep=0.75pt]  [font=\footnotesize]  {$w=2$};}
      \only<2>{
%Shape: Rectangle [id:dp34895762482355186] 
\draw  [color={rgb, 255:red, 0; green, 0; blue, 0 }  ,draw opacity=1 ][fill={rgb, 255:red, 200; green, 43; blue, 43 }  ,fill opacity=0.23 ] (279.6,109) -- (299.6,109) -- (299.6,129) -- (279.6,129) -- cycle ;
%Shape: Rectangle [id:dp3706129514955345] 
\draw  [color={rgb, 255:red, 0; green, 0; blue, 0 }  ,draw opacity=1 ][fill={rgb, 255:red, 200; green, 43; blue, 43 }  ,fill opacity=0.64 ] (299.6,129) -- (319.6,129) -- (319.6,149) -- (299.6,149) -- cycle ;
%Shape: Rectangle [id:dp7937037744108977] 
\draw  [color={rgb, 255:red, 0; green, 0; blue, 0 }  ,draw opacity=1 ][fill={rgb, 255:red, 200; green, 43; blue, 43 }  ,fill opacity=0.64 ] (279.6,129) -- (299.6,129) -- (299.6,149) -- (279.6,149) -- cycle ;
%Shape: Rectangle [id:dp20259457131157055] 
\draw  [color={rgb, 255:red, 0; green, 0; blue, 0 }  ,draw opacity=1 ][fill={rgb, 255:red, 200; green, 43; blue, 43 }  ,fill opacity=0.69 ] (299.6,149) -- (319.6,149) -- (319.6,169) -- (299.6,169) -- cycle ;
%Shape: Rectangle [id:dp4433057251733791] 
\draw  [color={rgb, 255:red, 0; green, 0; blue, 0 }  ,draw opacity=1 ][fill={rgb, 255:red, 200; green, 43; blue, 43 }  ,fill opacity=0.68 ] (319.6,129) -- (339.6,129) -- (339.6,149) -- (319.6,149) -- cycle ;
%Shape: Rectangle [id:dp6669398936433204] 
\draw  [color={rgb, 255:red, 0; green, 0; blue, 0 }  ,draw opacity=1 ][fill={rgb, 255:red, 200; green, 43; blue, 43 }  ,fill opacity=0.32 ] (299.6,89) -- (319.6,89) -- (319.6,109) -- (299.6,109) -- cycle ;
%Shape: Rectangle [id:dp21171515185264544] 
\draw  [color={rgb, 255:red, 0; green, 0; blue, 0 }  ,draw opacity=1 ][fill={rgb, 255:red, 200; green, 43; blue, 43 }  ,fill opacity=0.64 ] (299.6,109) -- (319.6,109) -- (319.6,129) -- (299.6,129) -- cycle ;
%Shape: Rectangle [id:dp06872955891721255] 
\draw  [color={rgb, 255:red, 0; green, 0; blue, 0 }  ,draw opacity=1 ][fill={rgb, 255:red, 200; green, 43; blue, 43 }  ,fill opacity=0.23 ] (239.6,69) -- (259.6,69) -- (259.6,89) -- (239.6,89) -- cycle ;
%Shape: Rectangle [id:dp31874857825881] 
\draw  [color={rgb, 255:red, 0; green, 0; blue, 0 }  ,draw opacity=1 ][fill={rgb, 255:red, 200; green, 43; blue, 43 }  ,fill opacity=0.21 ] (259.6,89) -- (279.6,89) -- (279.6,109) -- (259.6,109) -- cycle ;
%Shape: Rectangle [id:dp6824558327787323] 
\draw  [color={rgb, 255:red, 0; green, 0; blue, 0 }  ,draw opacity=1 ][fill={rgb, 255:red, 200; green, 43; blue, 43 }  ,fill opacity=0.25 ] (239.6,89) -- (259.6,89) -- (259.6,109) -- (239.6,109) -- cycle ;
%Shape: Rectangle [id:dp7485420074379594] 
\draw  [color={rgb, 255:red, 0; green, 0; blue, 0 }  ,draw opacity=1 ][fill={rgb, 255:red, 200; green, 43; blue, 43 }  ,fill opacity=0.28 ] (239.6,109) -- (259.6,109) -- (259.6,129) -- (239.6,129) -- cycle ;
%Shape: Rectangle [id:dp518076998549063] 
\draw  [color={rgb, 255:red, 0; green, 0; blue, 0 }  ,draw opacity=1 ][fill={rgb, 255:red, 200; green, 43; blue, 43 }  ,fill opacity=0.66 ] (259.6,129) -- (279.6,129) -- (279.6,149) -- (259.6,149) -- cycle ;
%Shape: Rectangle [id:dp30451200783199606] 
\draw  [color={rgb, 255:red, 0; green, 0; blue, 0 }  ,draw opacity=1 ][fill={rgb, 255:red, 200; green, 43; blue, 43 }  ,fill opacity=0.77 ] (279.6,149) -- (299.6,149) -- (299.6,169) -- (279.6,169) -- cycle ;
%Shape: Rectangle [id:dp1908549710583085] 
\draw  [color={rgb, 255:red, 0; green, 0; blue, 0 }  ,draw opacity=1 ][fill={rgb, 255:red, 200; green, 43; blue, 43 }  ,fill opacity=0.24 ] (259.6,109) -- (279.6,109) -- (279.6,129) -- (259.6,129) -- cycle ;
%Shape: Rectangle [id:dp4988144165606093] 
\draw  [color={rgb, 255:red, 0; green, 0; blue, 0 }  ,draw opacity=1 ][fill={rgb, 255:red, 200; green, 43; blue, 43 }  ,fill opacity=0.26 ] (259.6,69) -- (279.6,69) -- (279.6,89) -- (259.6,89) -- cycle ;
%Shape: Rectangle [id:dp7655450055950659] 
\draw  [color={rgb, 255:red, 0; green, 0; blue, 0 }  ,draw opacity=1 ][fill={rgb, 255:red, 200; green, 43; blue, 43 }  ,fill opacity=0.22 ] (279.6,89) -- (299.6,89) -- (299.6,109) -- (279.6,109) -- cycle ;
%Shape: Rectangle [id:dp13960350908642583] 
\draw  [color={rgb, 255:red, 0; green, 0; blue, 0 }  ,draw opacity=1 ][fill={rgb, 255:red, 200; green, 43; blue, 43 }  ,fill opacity=0.29 ] (279.6,69) -- (299.6,69) -- (299.6,89) -- (279.6,89) -- cycle ;
%Shape: Rectangle [id:dp4182608599365719] 
\draw  [color={rgb, 255:red, 0; green, 0; blue, 0 }  ,draw opacity=1 ][fill={rgb, 255:red, 200; green, 43; blue, 43 }  ,fill opacity=0.28 ] (299.6,69) -- (319.6,69) -- (319.6,89) -- (299.6,89) -- cycle ;
%Shape: Rectangle [id:dp3569567884999012] 
\draw  [color={rgb, 255:red, 0; green, 0; blue, 0 }  ,draw opacity=1 ][fill={rgb, 255:red, 200; green, 43; blue, 43 }  ,fill opacity=0.65 ] (319.6,109) -- (339.6,109) -- (339.6,129) -- (319.6,129) -- cycle ;
%Shape: Rectangle [id:dp28676610010827] 
\draw  [color={rgb, 255:red, 0; green, 0; blue, 0 }  ,draw opacity=1 ][fill={rgb, 255:red, 200; green, 43; blue, 43 }  ,fill opacity=0.64 ] (239.6,129) -- (259.6,129) -- (259.6,149) -- (239.6,149) -- cycle ;
%Shape: Rectangle [id:dp45600868944371487] 
\draw  [color={rgb, 255:red, 0; green, 0; blue, 0 }  ,draw opacity=1 ][fill={rgb, 255:red, 200; green, 43; blue, 43 }  ,fill opacity=0.73 ] (239.6,149) -- (259.6,149) -- (259.6,169) -- (239.6,169) -- cycle ;
%Shape: Rectangle [id:dp8145428819349263] 
\draw  [color={rgb, 255:red, 0; green, 0; blue, 0 }  ,draw opacity=1 ][fill={rgb, 255:red, 200; green, 43; blue, 43 }  ,fill opacity=0.74 ] (259.6,149) -- (279.6,149) -- (279.6,169) -- (259.6,169) -- cycle ;
%Shape: Rectangle [id:dp7078836921241948] 
\draw  [color={rgb, 255:red, 0; green, 0; blue, 0 }  ,draw opacity=1 ][fill={rgb, 255:red, 200; green, 43; blue, 43 }  ,fill opacity=0.68 ] (319.6,149) -- (339.6,149) -- (339.6,169) -- (319.6,169) -- cycle ;
%Shape: Rectangle [id:dp8396418073196056] 
\draw  [color={rgb, 255:red, 0; green, 0; blue, 0 }  ,draw opacity=1 ][fill={rgb, 255:red, 200; green, 43; blue, 43 }  ,fill opacity=0.7 ] (319.6,69) -- (339.6,69) -- (339.6,89) -- (319.6,89) -- cycle ;
%Shape: Rectangle [id:dp9348793553111198] 
\draw  [color={rgb, 255:red, 0; green, 0; blue, 0 }  ,draw opacity=1 ][fill={rgb, 255:red, 200; green, 43; blue, 43 }  ,fill opacity=0.63 ] (319.6,89) -- (339.6,89) -- (339.6,109) -- (319.6,109) -- cycle ;
%Shape: Rectangle [id:dp10953260736468606] 
\draw  [color={rgb, 255:red, 30; green, 14; blue, 222 }  ,draw opacity=1 ][fill={rgb, 255:red, 192; green, 177; blue, 255 }  ,fill opacity=0.34 ] (279.6,109) -- (299.6,109) -- (299.6,129) -- (279.6,129) -- cycle ;
%Shape: Rectangle [id:dp7457156202929361] 
\draw  [color={rgb, 255:red, 255; green, 255; blue, 255 }  ,draw opacity=1 ] (209,54.5) -- (451,54.5) -- (451,190.5) -- (209,190.5) -- cycle ;
%Shape: Rectangle [id:dp4886950260713504] 
\draw  [color={rgb, 255:red, 222; green, 195; blue, 14 }  ,draw opacity=1 ][fill={rgb, 255:red, 192; green, 177; blue, 255 }  ,fill opacity=0.34 ] (239.6,109) -- (259.6,109) -- (259.6,129) -- (239.6,129) -- cycle ;
%Shape: Rectangle [id:dp7097656602066365] 
\draw  [color={rgb, 255:red, 222; green, 195; blue, 14 }  ,draw opacity=1 ][fill={rgb, 255:red, 192; green, 177; blue, 255 }  ,fill opacity=0.34 ] (259.6,335) -- (279.6,335) -- (279.6,355) -- (259.6,355) -- cycle ;


\draw (304.8,136) node [anchor=north west][inner sep=0.75pt]  [font=\footnotesize] [align=left] {64};
% Text Node
\draw (324.8,155.2) node [anchor=north west][inner sep=0.75pt]  [font=\footnotesize] [align=left] {69};
% Text Node
\draw (305.2,115.6) node [anchor=north west][inner sep=0.75pt]  [font=\footnotesize] [align=left] {65};
% Text Node
\draw (305.2,95.6) node [anchor=north west][inner sep=0.75pt]  [font=\footnotesize] [align=left] {32};
% Text Node
\draw (285.2,95.6) node [anchor=north west][inner sep=0.75pt]  [font=\footnotesize] [align=left] {22};
% Text Node
\draw (325.2,115.6) node [anchor=north west][inner sep=0.75pt]  [font=\footnotesize] [align=left] {65};
% Text Node
\draw (244.4,76) node [anchor=north west][inner sep=0.75pt]  [font=\footnotesize] [align=left] {23};
% Text Node
\draw (265.2,96) node [anchor=north west][inner sep=0.75pt]  [font=\footnotesize] [align=left] {21};
% Text Node
\draw (244.8,95.6) node [anchor=north west][inner sep=0.75pt]  [font=\footnotesize] [align=left] {25};
% Text Node
\draw (264.8,116.4) node [anchor=north west][inner sep=0.75pt]  [font=\footnotesize] [align=left] {24};
% Text Node
\draw (284.8,135.6) node [anchor=north west][inner sep=0.75pt]  [font=\footnotesize] [align=left] {64};
% Text Node
\draw (304.8,155.2) node [anchor=north west][inner sep=0.75pt]  [font=\footnotesize] [align=left] {69};
% Text Node
\draw (264.4,75.6) node [anchor=north west][inner sep=0.75pt]  [font=\footnotesize] [align=left] {26};
% Text Node
\draw (285.2,75.6) node [anchor=north west][inner sep=0.75pt]  [font=\footnotesize] [align=left] {29};
% Text Node
\draw (304.8,75.2) node [anchor=north west][inner sep=0.75pt]  [font=\footnotesize] [align=left] {28};
% Text Node
\draw (324.8,95.2) node [anchor=north west][inner sep=0.75pt]  [font=\footnotesize] [align=left] {63};
% Text Node
\draw (324.8,76.8) node [anchor=north west][inner sep=0.75pt]  [font=\footnotesize] [align=left] {70};
% Text Node
\draw (244.8,115.6) node [anchor=north west][inner sep=0.75pt]  [font=\footnotesize] [align=left] {28};
% Text Node
\draw (264.8,135.6) node [anchor=north west][inner sep=0.75pt]  [font=\footnotesize] [align=left] {66};
% Text Node
\draw (284.8,155.2) node [anchor=north west][inner sep=0.75pt]  [font=\footnotesize] [align=left] {77};
% Text Node
\draw (244,136) node [anchor=north west][inner sep=0.75pt]  [font=\footnotesize] [align=left] {63};
% Text Node
\draw (264,155.2) node [anchor=north west][inner sep=0.75pt]  [font=\footnotesize] [align=left] {74};
% Text Node
\draw (245.2,155.6) node [anchor=north west][inner sep=0.75pt]  [font=\footnotesize] [align=left] {73};
% Text Node
\draw (324.8,136.4) node [anchor=north west][inner sep=0.75pt]  [font=\footnotesize] [align=left] {68};
% Text Node
\draw (285.2,116.4) node [anchor=north west][inner sep=0.75pt]  [font=\footnotesize] [align=left] {23};
% Text Node


\draw (355,80) node [anchor=north west][inner sep=0.75pt]  [font=\footnotesize]  {$w=2$};
% Text Node
\draw (355,135.4) node [anchor=north west][inner sep=0.75pt]  [font=\footnotesize]  {$i=-2$};
% Text Node
\draw (355,108.8) node [anchor=north west][inner sep=0.75pt]  [font=\footnotesize]  {$dx=1,\ dy=0$};
% Text Node
\draw (355,125.27) node [anchor=north west][inner sep=0.75pt]  [font=\footnotesize,color={rgb, 255:red, 72; green, 36; blue, 227 }  ,opacity=1 ] [align=left] {\textcolor[rgb]{0.55,0.64,0.02}{pixel comparé}};
% Text Node
\draw (355,98.13) node [anchor=north west][inner sep=0.75pt]  [font=\footnotesize,opacity=1 ] [align=left] {direction horizontal};
% Text Node
\draw (355.3,154.4) node [anchor=north west][inner sep=0.75pt]  [font=\footnotesize]  {$S=28$};
% Text Node
\draw (383.3,154.0) node [anchor=north west][inner sep=0.75pt]  [font=\footnotesize]  {$S^{2} =784$};

% Text Node
\draw (355,117) node [anchor=north west][inner sep=0.75pt]  [font=\footnotesize,color={rgb, 255:red, 72; green, 36; blue, 227 }  ,opacity=1 ] [align=left] {pixel considéré};
% Text Node}
      \only<3>{
%Shape: Rectangle [id:dp4812352510414377] 
\draw  [color={rgb, 255:red, 0; green, 0; blue, 0 }  ,draw opacity=1 ][fill={rgb, 255:red, 200; green, 43; blue, 43 }  ,fill opacity=0.23 ] (279.6,109) -- (299.6,109) -- (299.6,129) -- (279.6,129) -- cycle ;
%Shape: Rectangle [id:dp8369925573319851] 
\draw  [color={rgb, 255:red, 0; green, 0; blue, 0 }  ,draw opacity=1 ][fill={rgb, 255:red, 200; green, 43; blue, 43 }  ,fill opacity=0.64 ] (299.6,129) -- (319.6,129) -- (319.6,149) -- (299.6,149) -- cycle ;
%Shape: Rectangle [id:dp22732928696285393] 
\draw  [color={rgb, 255:red, 0; green, 0; blue, 0 }  ,draw opacity=1 ][fill={rgb, 255:red, 200; green, 43; blue, 43 }  ,fill opacity=0.64 ] (279.6,129) -- (299.6,129) -- (299.6,149) -- (279.6,149) -- cycle ;
%Shape: Rectangle [id:dp8704034364799831] 
\draw  [color={rgb, 255:red, 0; green, 0; blue, 0 }  ,draw opacity=1 ][fill={rgb, 255:red, 200; green, 43; blue, 43 }  ,fill opacity=0.69 ] (299.6,149) -- (319.6,149) -- (319.6,169) -- (299.6,169) -- cycle ;
%Shape: Rectangle [id:dp37274322264135495] 
\draw  [color={rgb, 255:red, 0; green, 0; blue, 0 }  ,draw opacity=1 ][fill={rgb, 255:red, 200; green, 43; blue, 43 }  ,fill opacity=0.68 ] (319.6,129) -- (339.6,129) -- (339.6,149) -- (319.6,149) -- cycle ;
%Shape: Rectangle [id:dp2642638953436689] 
\draw  [color={rgb, 255:red, 0; green, 0; blue, 0 }  ,draw opacity=1 ][fill={rgb, 255:red, 200; green, 43; blue, 43 }  ,fill opacity=0.32 ] (299.6,89) -- (319.6,89) -- (319.6,109) -- (299.6,109) -- cycle ;
%Shape: Rectangle [id:dp5440385706105905] 
\draw  [color={rgb, 255:red, 0; green, 0; blue, 0 }  ,draw opacity=1 ][fill={rgb, 255:red, 200; green, 43; blue, 43 }  ,fill opacity=0.64 ] (299.6,109) -- (319.6,109) -- (319.6,129) -- (299.6,129) -- cycle ;
%Shape: Rectangle [id:dp02554658774016616] 
\draw  [color={rgb, 255:red, 0; green, 0; blue, 0 }  ,draw opacity=1 ][fill={rgb, 255:red, 200; green, 43; blue, 43 }  ,fill opacity=0.23 ] (239.6,69) -- (259.6,69) -- (259.6,89) -- (239.6,89) -- cycle ;
%Shape: Rectangle [id:dp2866912157681306] 
\draw  [color={rgb, 255:red, 0; green, 0; blue, 0 }  ,draw opacity=1 ][fill={rgb, 255:red, 200; green, 43; blue, 43 }  ,fill opacity=0.21 ] (259.6,89) -- (279.6,89) -- (279.6,109) -- (259.6,109) -- cycle ;
%Shape: Rectangle [id:dp5714893317994543] 
\draw  [color={rgb, 255:red, 0; green, 0; blue, 0 }  ,draw opacity=1 ][fill={rgb, 255:red, 200; green, 43; blue, 43 }  ,fill opacity=0.25 ] (239.6,89) -- (259.6,89) -- (259.6,109) -- (239.6,109) -- cycle ;
%Shape: Rectangle [id:dp19250474380221405] 
\draw  [color={rgb, 255:red, 0; green, 0; blue, 0 }  ,draw opacity=1 ][fill={rgb, 255:red, 200; green, 43; blue, 43 }  ,fill opacity=0.28 ] (239.6,109) -- (259.6,109) -- (259.6,129) -- (239.6,129) -- cycle ;
%Shape: Rectangle [id:dp5221422669665428] 
\draw  [color={rgb, 255:red, 0; green, 0; blue, 0 }  ,draw opacity=1 ][fill={rgb, 255:red, 200; green, 43; blue, 43 }  ,fill opacity=0.66 ] (259.6,129) -- (279.6,129) -- (279.6,149) -- (259.6,149) -- cycle ;
%Shape: Rectangle [id:dp6823240417832277] 
\draw  [color={rgb, 255:red, 0; green, 0; blue, 0 }  ,draw opacity=1 ][fill={rgb, 255:red, 200; green, 43; blue, 43 }  ,fill opacity=0.77 ] (279.6,149) -- (299.6,149) -- (299.6,169) -- (279.6,169) -- cycle ;
%Shape: Rectangle [id:dp9885272513506412] 
\draw  [color={rgb, 255:red, 0; green, 0; blue, 0 }  ,draw opacity=1 ][fill={rgb, 255:red, 200; green, 43; blue, 43 }  ,fill opacity=0.24 ] (259.6,109) -- (279.6,109) -- (279.6,129) -- (259.6,129) -- cycle ;
%Shape: Rectangle [id:dp21382036721315567] 
\draw  [color={rgb, 255:red, 0; green, 0; blue, 0 }  ,draw opacity=1 ][fill={rgb, 255:red, 200; green, 43; blue, 43 }  ,fill opacity=0.26 ] (259.6,69) -- (279.6,69) -- (279.6,89) -- (259.6,89) -- cycle ;
%Shape: Rectangle [id:dp8913435985913778] 
\draw  [color={rgb, 255:red, 0; green, 0; blue, 0 }  ,draw opacity=1 ][fill={rgb, 255:red, 200; green, 43; blue, 43 }  ,fill opacity=0.22 ] (279.6,89) -- (299.6,89) -- (299.6,109) -- (279.6,109) -- cycle ;
%Shape: Rectangle [id:dp4260618206954214] 
\draw  [color={rgb, 255:red, 0; green, 0; blue, 0 }  ,draw opacity=1 ][fill={rgb, 255:red, 200; green, 43; blue, 43 }  ,fill opacity=0.29 ] (279.6,69) -- (299.6,69) -- (299.6,89) -- (279.6,89) -- cycle ;
%Shape: Rectangle [id:dp6202390296160146] 
\draw  [color={rgb, 255:red, 0; green, 0; blue, 0 }  ,draw opacity=1 ][fill={rgb, 255:red, 200; green, 43; blue, 43 }  ,fill opacity=0.28 ] (299.6,69) -- (319.6,69) -- (319.6,89) -- (299.6,89) -- cycle ;
%Shape: Rectangle [id:dp3342486595721411] 
\draw  [color={rgb, 255:red, 0; green, 0; blue, 0 }  ,draw opacity=1 ][fill={rgb, 255:red, 200; green, 43; blue, 43 }  ,fill opacity=0.65 ] (319.6,109) -- (339.6,109) -- (339.6,129) -- (319.6,129) -- cycle ;
%Shape: Rectangle [id:dp9920584697146176] 
\draw  [color={rgb, 255:red, 0; green, 0; blue, 0 }  ,draw opacity=1 ][fill={rgb, 255:red, 200; green, 43; blue, 43 }  ,fill opacity=0.64 ] (239.6,129) -- (259.6,129) -- (259.6,149) -- (239.6,149) -- cycle ;
%Shape: Rectangle [id:dp7634742015665422] 
\draw  [color={rgb, 255:red, 0; green, 0; blue, 0 }  ,draw opacity=1 ][fill={rgb, 255:red, 200; green, 43; blue, 43 }  ,fill opacity=0.73 ] (239.6,149) -- (259.6,149) -- (259.6,169) -- (239.6,169) -- cycle ;
%Shape: Rectangle [id:dp1475849290481438] 
\draw  [color={rgb, 255:red, 0; green, 0; blue, 0 }  ,draw opacity=1 ][fill={rgb, 255:red, 200; green, 43; blue, 43 }  ,fill opacity=0.74 ] (259.6,149) -- (279.6,149) -- (279.6,169) -- (259.6,169) -- cycle ;
%Shape: Rectangle [id:dp08516043078024949] 
\draw  [color={rgb, 255:red, 0; green, 0; blue, 0 }  ,draw opacity=1 ][fill={rgb, 255:red, 200; green, 43; blue, 43 }  ,fill opacity=0.68 ] (319.6,149) -- (339.6,149) -- (339.6,169) -- (319.6,169) -- cycle ;
%Shape: Rectangle [id:dp07859591766270757] 
\draw  [color={rgb, 255:red, 0; green, 0; blue, 0 }  ,draw opacity=1 ][fill={rgb, 255:red, 200; green, 43; blue, 43 }  ,fill opacity=0.7 ] (319.6,69) -- (339.6,69) -- (339.6,89) -- (319.6,89) -- cycle ;
%Shape: Rectangle [id:dp5177098587849153] 
\draw  [color={rgb, 255:red, 0; green, 0; blue, 0 }  ,draw opacity=1 ][fill={rgb, 255:red, 200; green, 43; blue, 43 }  ,fill opacity=0.63 ] (319.6,89) -- (339.6,89) -- (339.6,109) -- (319.6,109) -- cycle ;
%Shape: Rectangle [id:dp08372666390563621] 
\draw  [color={rgb, 255:red, 30; green, 14; blue, 222 }  ,draw opacity=1 ][fill={rgb, 255:red, 192; green, 177; blue, 255 }  ,fill opacity=0.34 ] (279.6,109) -- (299.6,109) -- (299.6,129) -- (279.6,129) -- cycle ;
%Shape: Rectangle [id:dp3807570709257173] 
\draw  [color={rgb, 255:red, 255; green, 255; blue, 255 }  ,draw opacity=1 ] (209,54.5) -- (451,54.5) -- (451,190.5) -- (209,190.5) -- cycle ;
%Shape: Rectangle [id:dp44333620508990423] 
\draw  [color={rgb, 255:red, 222; green, 195; blue, 14 }  ,draw opacity=1 ][fill={rgb, 255:red, 192; green, 177; blue, 255 }  ,fill opacity=0.34 ] (259.6,109) -- (279.6,109) -- (279.6,129) -- (259.6,129) -- cycle ;

\draw (304.8,136) node [anchor=north west][inner sep=0.75pt]  [font=\footnotesize] [align=left] {64};
% Text Node
\draw (324.8,155.2) node [anchor=north west][inner sep=0.75pt]  [font=\footnotesize] [align=left] {69};
% Text Node
\draw (305.2,115.6) node [anchor=north west][inner sep=0.75pt]  [font=\footnotesize] [align=left] {65};
% Text Node
\draw (305.2,95.6) node [anchor=north west][inner sep=0.75pt]  [font=\footnotesize] [align=left] {32};
% Text Node
\draw (285.2,95.6) node [anchor=north west][inner sep=0.75pt]  [font=\footnotesize] [align=left] {22};
% Text Node
\draw (325.2,115.6) node [anchor=north west][inner sep=0.75pt]  [font=\footnotesize] [align=left] {65};
% Text Node
\draw (244.4,76) node [anchor=north west][inner sep=0.75pt]  [font=\footnotesize] [align=left] {23};
% Text Node
\draw (265.2,96) node [anchor=north west][inner sep=0.75pt]  [font=\footnotesize] [align=left] {21};
% Text Node
\draw (244.8,95.6) node [anchor=north west][inner sep=0.75pt]  [font=\footnotesize] [align=left] {25};
% Text Node
\draw (264.8,116.4) node [anchor=north west][inner sep=0.75pt]  [font=\footnotesize] [align=left] {24};
% Text Node
\draw (284.8,135.6) node [anchor=north west][inner sep=0.75pt]  [font=\footnotesize] [align=left] {64};
% Text Node
\draw (304.8,155.2) node [anchor=north west][inner sep=0.75pt]  [font=\footnotesize] [align=left] {69};
% Text Node
\draw (264.4,75.6) node [anchor=north west][inner sep=0.75pt]  [font=\footnotesize] [align=left] {26};
% Text Node
\draw (285.2,75.6) node [anchor=north west][inner sep=0.75pt]  [font=\footnotesize] [align=left] {29};
% Text Node
\draw (304.8,75.2) node [anchor=north west][inner sep=0.75pt]  [font=\footnotesize] [align=left] {28};
% Text Node
\draw (324.8,95.2) node [anchor=north west][inner sep=0.75pt]  [font=\footnotesize] [align=left] {63};
% Text Node
\draw (324.8,76.8) node [anchor=north west][inner sep=0.75pt]  [font=\footnotesize] [align=left] {70};
% Text Node
\draw (244.8,115.6) node [anchor=north west][inner sep=0.75pt]  [font=\footnotesize] [align=left] {28};
% Text Node
\draw (264.8,135.6) node [anchor=north west][inner sep=0.75pt]  [font=\footnotesize] [align=left] {66};
% Text Node
\draw (284.8,155.2) node [anchor=north west][inner sep=0.75pt]  [font=\footnotesize] [align=left] {77};
% Text Node
\draw (244,136) node [anchor=north west][inner sep=0.75pt]  [font=\footnotesize] [align=left] {63};
% Text Node
\draw (264,155.2) node [anchor=north west][inner sep=0.75pt]  [font=\footnotesize] [align=left] {74};
% Text Node
\draw (245.2,155.6) node [anchor=north west][inner sep=0.75pt]  [font=\footnotesize] [align=left] {73};
% Text Node
\draw (324.8,136.4) node [anchor=north west][inner sep=0.75pt]  [font=\footnotesize] [align=left] {68};
% Text Node
\draw (285.2,116.4) node [anchor=north west][inner sep=0.75pt]  [font=\footnotesize] [align=left] {23};
% Text Node
\draw (355,80) node [anchor=north west][inner sep=0.75pt]  [font=\footnotesize]  {$w=2$};
% Text Node
\draw (355,135.4) node [anchor=north west][inner sep=0.75pt]  [font=\footnotesize]  {$i=-1$};
% Text Node
\draw (355,108.8) node [anchor=north west][inner sep=0.75pt]  [font=\footnotesize]  {$dx=1,\ dy=0$};
% Text Node
\draw (355,125.27) node [anchor=north west][inner sep=0.75pt]  [font=\footnotesize,color={rgb, 255:red, 72; green, 36; blue, 227 }  ,opacity=1 ] [align=left] {\textcolor[rgb]{0.55,0.64,0.02}{pixel comparé}};
% Text Node
\draw (355,98.13) node [anchor=north west][inner sep=0.75pt]  [font=\footnotesize,opacity=1 ] [align=left] {direction horizontal};
% Text Node
\draw (355.3,154.4) node [anchor=north west][inner sep=0.75pt]  [font=\footnotesize]  {$S=52$};
% Text Node
\draw (383.3,154.0) node [anchor=north west][inner sep=0.75pt]  [font=\footnotesize]  {$S^{2} =1248$};

% Text Node
\draw (355,117) node [anchor=north west][inner sep=0.75pt]  [font=\footnotesize,color={rgb, 255:red, 72; green, 36; blue, 227 }  ,opacity=1 ] [align=left] {pixel considéré};
% Text Node}
      \only<4>{
%Shape: Rectangle [id:dp5192071024908617] 
\draw  [color={rgb, 255:red, 0; green, 0; blue, 0 }  ,draw opacity=1 ][fill={rgb, 255:red, 200; green, 43; blue, 43 }  ,fill opacity=0.23 ] (279.6,109) -- (299.6,109) -- (299.6,129) -- (279.6,129) -- cycle ;
%Shape: Rectangle [id:dp9236036776180179] 
\draw  [color={rgb, 255:red, 0; green, 0; blue, 0 }  ,draw opacity=1 ][fill={rgb, 255:red, 200; green, 43; blue, 43 }  ,fill opacity=0.64 ] (299.6,129) -- (319.6,129) -- (319.6,149) -- (299.6,149) -- cycle ;
%Shape: Rectangle [id:dp2517405814578214] 
\draw  [color={rgb, 255:red, 0; green, 0; blue, 0 }  ,draw opacity=1 ][fill={rgb, 255:red, 200; green, 43; blue, 43 }  ,fill opacity=0.64 ] (279.6,129) -- (299.6,129) -- (299.6,149) -- (279.6,149) -- cycle ;
%Shape: Rectangle [id:dp6487739812614729] 
\draw  [color={rgb, 255:red, 0; green, 0; blue, 0 }  ,draw opacity=1 ][fill={rgb, 255:red, 200; green, 43; blue, 43 }  ,fill opacity=0.69 ] (299.6,149) -- (319.6,149) -- (319.6,169) -- (299.6,169) -- cycle ;
%Shape: Rectangle [id:dp56790215710131] 
\draw  [color={rgb, 255:red, 0; green, 0; blue, 0 }  ,draw opacity=1 ][fill={rgb, 255:red, 200; green, 43; blue, 43 }  ,fill opacity=0.68 ] (319.6,129) -- (339.6,129) -- (339.6,149) -- (319.6,149) -- cycle ;
%Shape: Rectangle [id:dp3160175285872746] 
\draw  [color={rgb, 255:red, 0; green, 0; blue, 0 }  ,draw opacity=1 ][fill={rgb, 255:red, 200; green, 43; blue, 43 }  ,fill opacity=0.32 ] (299.6,89) -- (319.6,89) -- (319.6,109) -- (299.6,109) -- cycle ;
%Shape: Rectangle [id:dp29940151559187067] 
\draw  [color={rgb, 255:red, 0; green, 0; blue, 0 }  ,draw opacity=1 ][fill={rgb, 255:red, 200; green, 43; blue, 43 }  ,fill opacity=0.64 ] (299.6,109) -- (319.6,109) -- (319.6,129) -- (299.6,129) -- cycle ;
%Shape: Rectangle [id:dp34533507939410724] 
\draw  [color={rgb, 255:red, 0; green, 0; blue, 0 }  ,draw opacity=1 ][fill={rgb, 255:red, 200; green, 43; blue, 43 }  ,fill opacity=0.23 ] (239.6,69) -- (259.6,69) -- (259.6,89) -- (239.6,89) -- cycle ;
%Shape: Rectangle [id:dp09193846882501755] 
\draw  [color={rgb, 255:red, 0; green, 0; blue, 0 }  ,draw opacity=1 ][fill={rgb, 255:red, 200; green, 43; blue, 43 }  ,fill opacity=0.21 ] (259.6,89) -- (279.6,89) -- (279.6,109) -- (259.6,109) -- cycle ;
%Shape: Rectangle [id:dp32652928317529517] 
\draw  [color={rgb, 255:red, 0; green, 0; blue, 0 }  ,draw opacity=1 ][fill={rgb, 255:red, 200; green, 43; blue, 43 }  ,fill opacity=0.25 ] (239.6,89) -- (259.6,89) -- (259.6,109) -- (239.6,109) -- cycle ;
%Shape: Rectangle [id:dp8857799255732063] 
\draw  [color={rgb, 255:red, 0; green, 0; blue, 0 }  ,draw opacity=1 ][fill={rgb, 255:red, 200; green, 43; blue, 43 }  ,fill opacity=0.28 ] (239.6,109) -- (259.6,109) -- (259.6,129) -- (239.6,129) -- cycle ;
%Shape: Rectangle [id:dp46673758434330925] 
\draw  [color={rgb, 255:red, 0; green, 0; blue, 0 }  ,draw opacity=1 ][fill={rgb, 255:red, 200; green, 43; blue, 43 }  ,fill opacity=0.66 ] (259.6,129) -- (279.6,129) -- (279.6,149) -- (259.6,149) -- cycle ;
%Shape: Rectangle [id:dp016678671468070227] 
\draw  [color={rgb, 255:red, 0; green, 0; blue, 0 }  ,draw opacity=1 ][fill={rgb, 255:red, 200; green, 43; blue, 43 }  ,fill opacity=0.77 ] (279.6,149) -- (299.6,149) -- (299.6,169) -- (279.6,169) -- cycle ;
%Shape: Rectangle [id:dp9127282378074322] 
\draw  [color={rgb, 255:red, 0; green, 0; blue, 0 }  ,draw opacity=1 ][fill={rgb, 255:red, 200; green, 43; blue, 43 }  ,fill opacity=0.24 ] (259.6,109) -- (279.6,109) -- (279.6,129) -- (259.6,129) -- cycle ;
%Shape: Rectangle [id:dp3547099592131202] 
\draw  [color={rgb, 255:red, 0; green, 0; blue, 0 }  ,draw opacity=1 ][fill={rgb, 255:red, 200; green, 43; blue, 43 }  ,fill opacity=0.26 ] (259.6,69) -- (279.6,69) -- (279.6,89) -- (259.6,89) -- cycle ;
%Shape: Rectangle [id:dp5783170288231004] 
\draw  [color={rgb, 255:red, 0; green, 0; blue, 0 }  ,draw opacity=1 ][fill={rgb, 255:red, 200; green, 43; blue, 43 }  ,fill opacity=0.22 ] (279.6,89) -- (299.6,89) -- (299.6,109) -- (279.6,109) -- cycle ;
%Shape: Rectangle [id:dp5807823124816117] 
\draw  [color={rgb, 255:red, 0; green, 0; blue, 0 }  ,draw opacity=1 ][fill={rgb, 255:red, 200; green, 43; blue, 43 }  ,fill opacity=0.29 ] (279.6,69) -- (299.6,69) -- (299.6,89) -- (279.6,89) -- cycle ;
%Shape: Rectangle [id:dp18187210922169417] 
\draw  [color={rgb, 255:red, 0; green, 0; blue, 0 }  ,draw opacity=1 ][fill={rgb, 255:red, 200; green, 43; blue, 43 }  ,fill opacity=0.28 ] (299.6,69) -- (319.6,69) -- (319.6,89) -- (299.6,89) -- cycle ;
%Shape: Rectangle [id:dp3840923530349337] 
\draw  [color={rgb, 255:red, 0; green, 0; blue, 0 }  ,draw opacity=1 ][fill={rgb, 255:red, 200; green, 43; blue, 43 }  ,fill opacity=0.65 ] (319.6,109) -- (339.6,109) -- (339.6,129) -- (319.6,129) -- cycle ;
%Shape: Rectangle [id:dp12505696536988953] 
\draw  [color={rgb, 255:red, 0; green, 0; blue, 0 }  ,draw opacity=1 ][fill={rgb, 255:red, 200; green, 43; blue, 43 }  ,fill opacity=0.64 ] (239.6,129) -- (259.6,129) -- (259.6,149) -- (239.6,149) -- cycle ;
%Shape: Rectangle [id:dp8093623572954336] 
\draw  [color={rgb, 255:red, 0; green, 0; blue, 0 }  ,draw opacity=1 ][fill={rgb, 255:red, 200; green, 43; blue, 43 }  ,fill opacity=0.73 ] (239.6,149) -- (259.6,149) -- (259.6,169) -- (239.6,169) -- cycle ;
%Shape: Rectangle [id:dp4501480289911729] 
\draw  [color={rgb, 255:red, 0; green, 0; blue, 0 }  ,draw opacity=1 ][fill={rgb, 255:red, 200; green, 43; blue, 43 }  ,fill opacity=0.74 ] (259.6,149) -- (279.6,149) -- (279.6,169) -- (259.6,169) -- cycle ;
%Shape: Rectangle [id:dp1842703104934752] 
\draw  [color={rgb, 255:red, 0; green, 0; blue, 0 }  ,draw opacity=1 ][fill={rgb, 255:red, 200; green, 43; blue, 43 }  ,fill opacity=0.68 ] (319.6,149) -- (339.6,149) -- (339.6,169) -- (319.6,169) -- cycle ;
%Shape: Rectangle [id:dp993771752487645] 
\draw  [color={rgb, 255:red, 0; green, 0; blue, 0 }  ,draw opacity=1 ][fill={rgb, 255:red, 200; green, 43; blue, 43 }  ,fill opacity=0.7 ] (319.6,69) -- (339.6,69) -- (339.6,89) -- (319.6,89) -- cycle ;
%Shape: Rectangle [id:dp4196239051406302] 
\draw  [color={rgb, 255:red, 0; green, 0; blue, 0 }  ,draw opacity=1 ][fill={rgb, 255:red, 200; green, 43; blue, 43 }  ,fill opacity=0.63 ] (319.6,89) -- (339.6,89) -- (339.6,109) -- (319.6,109) -- cycle ;
%Shape: Rectangle [id:dp9089119523474927] 
\draw  [color={rgb, 255:red, 255; green, 255; blue, 255 }  ,draw opacity=1 ] (209,54.5) -- (451,54.5) -- (451,190.5) -- (209,190.5) -- cycle ;
%Shape: Rectangle [id:dp9656794727290822] 
\draw  [color={rgb, 255:red, 30; green, 14; blue, 222 }  ,draw opacity=1 ][fill={rgb, 255:red, 192; green, 177; blue, 255 }  ,fill opacity=0.34 ] (279.6,109) -- (299.6,109) -- (299.6,129) -- (279.6,129) -- cycle ;
%Shape: Rectangle [id:dp2872429599526535] 
\draw  [color={rgb, 255:red, 222; green, 195; blue, 14 }  ,draw opacity=1 ][fill={rgb, 255:red, 192; green, 177; blue, 255 }  ,fill opacity=0.34 ] (319.6,109) -- (339.6,109) -- (339.6,129) -- (319.6,129) -- cycle ;

% Text Node
\draw (304.8,136) node [anchor=north west][inner sep=0.75pt]  [font=\footnotesize] [align=left] {64};
% Text Node
\draw (324.8,155.2) node [anchor=north west][inner sep=0.75pt]  [font=\footnotesize] [align=left] {69};
% Text Node
\draw (305.2,115.6) node [anchor=north west][inner sep=0.75pt]  [font=\footnotesize] [align=left] {65};
% Text Node
\draw (305.2,95.6) node [anchor=north west][inner sep=0.75pt]  [font=\footnotesize] [align=left] {32};
% Text Node
\draw (285.2,95.6) node [anchor=north west][inner sep=0.75pt]  [font=\footnotesize] [align=left] {22};
% Text Node
\draw (325.2,115.6) node [anchor=north west][inner sep=0.75pt]  [font=\footnotesize] [align=left] {65};
% Text Node
\draw (244.4,76) node [anchor=north west][inner sep=0.75pt]  [font=\footnotesize] [align=left] {23};
% Text Node
\draw (265.2,96) node [anchor=north west][inner sep=0.75pt]  [font=\footnotesize] [align=left] {21};
% Text Node
\draw (244.8,95.6) node [anchor=north west][inner sep=0.75pt]  [font=\footnotesize] [align=left] {25};
% Text Node
\draw (264.8,116.4) node [anchor=north west][inner sep=0.75pt]  [font=\footnotesize] [align=left] {24};
% Text Node
\draw (284.8,135.6) node [anchor=north west][inner sep=0.75pt]  [font=\footnotesize] [align=left] {64};
% Text Node
\draw (304.8,155.2) node [anchor=north west][inner sep=0.75pt]  [font=\footnotesize] [align=left] {69};
% Text Node
\draw (264.4,75.6) node [anchor=north west][inner sep=0.75pt]  [font=\footnotesize] [align=left] {26};
% Text Node
\draw (285.2,75.6) node [anchor=north west][inner sep=0.75pt]  [font=\footnotesize] [align=left] {29};
% Text Node
\draw (304.8,75.2) node [anchor=north west][inner sep=0.75pt]  [font=\footnotesize] [align=left] {28};
% Text Node
\draw (324.8,95.2) node [anchor=north west][inner sep=0.75pt]  [font=\footnotesize] [align=left] {63};
% Text Node
\draw (324.8,76.8) node [anchor=north west][inner sep=0.75pt]  [font=\footnotesize] [align=left] {70};
% Text Node
\draw (244.8,115.6) node [anchor=north west][inner sep=0.75pt]  [font=\footnotesize] [align=left] {28};
% Text Node
\draw (264.8,135.6) node [anchor=north west][inner sep=0.75pt]  [font=\footnotesize] [align=left] {66};
% Text Node
\draw (284.8,155.2) node [anchor=north west][inner sep=0.75pt]  [font=\footnotesize] [align=left] {77};
% Text Node
\draw (244,136) node [anchor=north west][inner sep=0.75pt]  [font=\footnotesize] [align=left] {63};
% Text Node
\draw (264,155.2) node [anchor=north west][inner sep=0.75pt]  [font=\footnotesize] [align=left] {74};
% Text Node
\draw (245.2,155.6) node [anchor=north west][inner sep=0.75pt]  [font=\footnotesize] [align=left] {73};
% Text Node
\draw (324.8,136.4) node [anchor=north west][inner sep=0.75pt]  [font=\footnotesize] [align=left] {68};
% Text Node
\draw (285.2,116.4) node [anchor=north west][inner sep=0.75pt]  [font=\footnotesize] [align=left] {23};
% Text Node
\draw (358.17,167.4) node [anchor=north west][inner sep=0.75pt]  [font=\footnotesize]  {$Var( 1,0) =386.8$};

\draw (355,80) node [anchor=north west][inner sep=0.75pt]  [font=\footnotesize]  {$w=2$};
% Text Node
\draw (355,135.4) node [anchor=north west][inner sep=0.75pt]  [font=\footnotesize]  {$i=2$};
% Text Node
\draw (355,108.8) node [anchor=north west][inner sep=0.75pt]  [font=\footnotesize]  {$dx=1,\ dy=0$};
% Text Node
\draw (355,125.27) node [anchor=north west][inner sep=0.75pt]  [font=\footnotesize,color={rgb, 255:red, 72; green, 36; blue, 227 }  ,opacity=1 ] [align=left] {\textcolor[rgb]{0.55,0.64,0.02}{pixel comparé}};
% Text Node
\draw (355,98.13) node [anchor=north west][inner sep=0.75pt]  [font=\footnotesize,opacity=1 ] [align=left] {direction horizontal};
% Text Node
\draw (355.3,154.4) node [anchor=north west][inner sep=0.75pt]  [font=\footnotesize]  {$S=205$};
% Text Node
\draw (383.3,154.0) node [anchor=north west][inner sep=0.75pt]  [font=\footnotesize]  {$S^{2} =10339$};

% Text Node
\draw (355,117) node [anchor=north west][inner sep=0.75pt]  [font=\footnotesize,color={rgb, 255:red, 72; green, 36; blue, 227 }  ,opacity=1 ] [align=left] {pixel considéré};
% Text Node}
      \only<5>{
%Shape: Rectangle [id:dp3673988175672026] 
\draw  [color={rgb, 255:red, 0; green, 0; blue, 0 }  ,draw opacity=1 ][fill={rgb, 255:red, 200; green, 43; blue, 43 }  ,fill opacity=0.23 ] (279.6,109) -- (299.6,109) -- (299.6,129) -- (279.6,129) -- cycle ;
%Shape: Rectangle [id:dp06911017740122116] 
\draw  [color={rgb, 255:red, 0; green, 0; blue, 0 }  ,draw opacity=1 ][fill={rgb, 255:red, 200; green, 43; blue, 43 }  ,fill opacity=0.64 ] (299.6,129) -- (319.6,129) -- (319.6,149) -- (299.6,149) -- cycle ;
%Shape: Rectangle [id:dp9183287545259993] 
\draw  [color={rgb, 255:red, 0; green, 0; blue, 0 }  ,draw opacity=1 ][fill={rgb, 255:red, 200; green, 43; blue, 43 }  ,fill opacity=0.64 ] (279.6,129) -- (299.6,129) -- (299.6,149) -- (279.6,149) -- cycle ;
%Shape: Rectangle [id:dp6562694088472094] 
\draw  [color={rgb, 255:red, 0; green, 0; blue, 0 }  ,draw opacity=1 ][fill={rgb, 255:red, 200; green, 43; blue, 43 }  ,fill opacity=0.69 ] (299.6,149) -- (319.6,149) -- (319.6,169) -- (299.6,169) -- cycle ;
%Shape: Rectangle [id:dp7985021018689166] 
\draw  [color={rgb, 255:red, 0; green, 0; blue, 0 }  ,draw opacity=1 ][fill={rgb, 255:red, 200; green, 43; blue, 43 }  ,fill opacity=0.68 ] (319.6,129) -- (339.6,129) -- (339.6,149) -- (319.6,149) -- cycle ;
%Shape: Rectangle [id:dp7031718926325354] 
\draw  [color={rgb, 255:red, 0; green, 0; blue, 0 }  ,draw opacity=1 ][fill={rgb, 255:red, 200; green, 43; blue, 43 }  ,fill opacity=0.32 ] (299.6,89) -- (319.6,89) -- (319.6,109) -- (299.6,109) -- cycle ;
%Shape: Rectangle [id:dp783587759994216] 
\draw  [color={rgb, 255:red, 0; green, 0; blue, 0 }  ,draw opacity=1 ][fill={rgb, 255:red, 200; green, 43; blue, 43 }  ,fill opacity=0.64 ] (299.6,109) -- (319.6,109) -- (319.6,129) -- (299.6,129) -- cycle ;
%Shape: Rectangle [id:dp9662939252630034] 
\draw  [color={rgb, 255:red, 0; green, 0; blue, 0 }  ,draw opacity=1 ][fill={rgb, 255:red, 200; green, 43; blue, 43 }  ,fill opacity=0.23 ] (239.6,69) -- (259.6,69) -- (259.6,89) -- (239.6,89) -- cycle ;
%Shape: Rectangle [id:dp48575293023469235] 
\draw  [color={rgb, 255:red, 0; green, 0; blue, 0 }  ,draw opacity=1 ][fill={rgb, 255:red, 200; green, 43; blue, 43 }  ,fill opacity=0.21 ] (259.6,89) -- (279.6,89) -- (279.6,109) -- (259.6,109) -- cycle ;
%Shape: Rectangle [id:dp4857509184179015] 
\draw  [color={rgb, 255:red, 0; green, 0; blue, 0 }  ,draw opacity=1 ][fill={rgb, 255:red, 200; green, 43; blue, 43 }  ,fill opacity=0.25 ] (239.6,89) -- (259.6,89) -- (259.6,109) -- (239.6,109) -- cycle ;
%Shape: Rectangle [id:dp18106676258474197] 
\draw  [color={rgb, 255:red, 0; green, 0; blue, 0 }  ,draw opacity=1 ][fill={rgb, 255:red, 200; green, 43; blue, 43 }  ,fill opacity=0.28 ] (239.6,109) -- (259.6,109) -- (259.6,129) -- (239.6,129) -- cycle ;
%Shape: Rectangle [id:dp7398331420310125] 
\draw  [color={rgb, 255:red, 0; green, 0; blue, 0 }  ,draw opacity=1 ][fill={rgb, 255:red, 200; green, 43; blue, 43 }  ,fill opacity=0.66 ] (259.6,129) -- (279.6,129) -- (279.6,149) -- (259.6,149) -- cycle ;
%Shape: Rectangle [id:dp6664500511746738] 
\draw  [color={rgb, 255:red, 0; green, 0; blue, 0 }  ,draw opacity=1 ][fill={rgb, 255:red, 200; green, 43; blue, 43 }  ,fill opacity=0.77 ] (279.6,149) -- (299.6,149) -- (299.6,169) -- (279.6,169) -- cycle ;
%Shape: Rectangle [id:dp7761037756805319] 
\draw  [color={rgb, 255:red, 0; green, 0; blue, 0 }  ,draw opacity=1 ][fill={rgb, 255:red, 200; green, 43; blue, 43 }  ,fill opacity=0.24 ] (259.6,109) -- (279.6,109) -- (279.6,129) -- (259.6,129) -- cycle ;
%Shape: Rectangle [id:dp47615480173576585] 
\draw  [color={rgb, 255:red, 0; green, 0; blue, 0 }  ,draw opacity=1 ][fill={rgb, 255:red, 200; green, 43; blue, 43 }  ,fill opacity=0.26 ] (259.6,69) -- (279.6,69) -- (279.6,89) -- (259.6,89) -- cycle ;
%Shape: Rectangle [id:dp5511363480961518] 
\draw  [color={rgb, 255:red, 0; green, 0; blue, 0 }  ,draw opacity=1 ][fill={rgb, 255:red, 200; green, 43; blue, 43 }  ,fill opacity=0.22 ] (279.6,89) -- (299.6,89) -- (299.6,109) -- (279.6,109) -- cycle ;
%Shape: Rectangle [id:dp0682147004306336] 
\draw  [color={rgb, 255:red, 0; green, 0; blue, 0 }  ,draw opacity=1 ][fill={rgb, 255:red, 200; green, 43; blue, 43 }  ,fill opacity=0.29 ] (279.6,69) -- (299.6,69) -- (299.6,89) -- (279.6,89) -- cycle ;
%Shape: Rectangle [id:dp8691993765448085] 
\draw  [color={rgb, 255:red, 0; green, 0; blue, 0 }  ,draw opacity=1 ][fill={rgb, 255:red, 200; green, 43; blue, 43 }  ,fill opacity=0.28 ] (299.6,69) -- (319.6,69) -- (319.6,89) -- (299.6,89) -- cycle ;
%Shape: Rectangle [id:dp17404541200070656] 
\draw  [color={rgb, 255:red, 0; green, 0; blue, 0 }  ,draw opacity=1 ][fill={rgb, 255:red, 200; green, 43; blue, 43 }  ,fill opacity=0.65 ] (319.6,109) -- (339.6,109) -- (339.6,129) -- (319.6,129) -- cycle ;
%Shape: Rectangle [id:dp5923320394720909] 
\draw  [color={rgb, 255:red, 0; green, 0; blue, 0 }  ,draw opacity=1 ][fill={rgb, 255:red, 200; green, 43; blue, 43 }  ,fill opacity=0.64 ] (239.6,129) -- (259.6,129) -- (259.6,149) -- (239.6,149) -- cycle ;
%Shape: Rectangle [id:dp7479336276737337] 
\draw  [color={rgb, 255:red, 0; green, 0; blue, 0 }  ,draw opacity=1 ][fill={rgb, 255:red, 200; green, 43; blue, 43 }  ,fill opacity=0.73 ] (239.6,149) -- (259.6,149) -- (259.6,169) -- (239.6,169) -- cycle ;
%Shape: Rectangle [id:dp3768471165379389] 
\draw  [color={rgb, 255:red, 0; green, 0; blue, 0 }  ,draw opacity=1 ][fill={rgb, 255:red, 200; green, 43; blue, 43 }  ,fill opacity=0.74 ] (259.6,149) -- (279.6,149) -- (279.6,169) -- (259.6,169) -- cycle ;
%Shape: Rectangle [id:dp6110539098309046] 
\draw  [color={rgb, 255:red, 0; green, 0; blue, 0 }  ,draw opacity=1 ][fill={rgb, 255:red, 200; green, 43; blue, 43 }  ,fill opacity=0.68 ] (319.6,149) -- (339.6,149) -- (339.6,169) -- (319.6,169) -- cycle ;
%Shape: Rectangle [id:dp964874955359474] 
\draw  [color={rgb, 255:red, 0; green, 0; blue, 0 }  ,draw opacity=1 ][fill={rgb, 255:red, 200; green, 43; blue, 43 }  ,fill opacity=0.7 ] (319.6,69) -- (339.6,69) -- (339.6,89) -- (319.6,89) -- cycle ;
%Shape: Rectangle [id:dp5795440959245807] 
\draw  [color={rgb, 255:red, 0; green, 0; blue, 0 }  ,draw opacity=1 ][fill={rgb, 255:red, 200; green, 43; blue, 43 }  ,fill opacity=0.63 ] (319.6,89) -- (339.6,89) -- (339.6,109) -- (319.6,109) -- cycle ;
%Shape: Rectangle [id:dp5219399610706349] 
\draw  [color={rgb, 255:red, 30; green, 14; blue, 222 }  ,draw opacity=1 ][fill={rgb, 255:red, 192; green, 177; blue, 255 }  ,fill opacity=0.34 ] (279.6,109) -- (299.6,109) -- (299.6,129) -- (279.6,129) -- cycle ;
%Shape: Rectangle [id:dp6941838183488424] 
\draw  [color={rgb, 255:red, 255; green, 255; blue, 255 }  ,draw opacity=1 ] (209,54.5) -- (451,54.5) -- (451,190.5) -- (209,190.5) -- cycle ;

\draw (304.8,136) node [anchor=north west][inner sep=0.75pt]  [font=\footnotesize] [align=left] {64};
% Text Node
\draw (324.8,155.2) node [anchor=north west][inner sep=0.75pt]  [font=\footnotesize] [align=left] {69};
% Text Node
\draw (305.2,115.6) node [anchor=north west][inner sep=0.75pt]  [font=\footnotesize] [align=left] {65};
% Text Node
\draw (305.2,95.6) node [anchor=north west][inner sep=0.75pt]  [font=\footnotesize] [align=left] {32};
% Text Node
\draw (285.2,95.6) node [anchor=north west][inner sep=0.75pt]  [font=\footnotesize] [align=left] {22};
% Text Node
\draw (325.2,115.6) node [anchor=north west][inner sep=0.75pt]  [font=\footnotesize] [align=left] {65};
% Text Node
\draw (244.4,76) node [anchor=north west][inner sep=0.75pt]  [font=\footnotesize] [align=left] {23};
% Text Node
\draw (265.2,96) node [anchor=north west][inner sep=0.75pt]  [font=\footnotesize] [align=left] {21};
% Text Node
\draw (244.8,95.6) node [anchor=north west][inner sep=0.75pt]  [font=\footnotesize] [align=left] {25};
% Text Node
\draw (264.8,116.4) node [anchor=north west][inner sep=0.75pt]  [font=\footnotesize] [align=left] {24};
% Text Node
\draw (284.8,135.6) node [anchor=north west][inner sep=0.75pt]  [font=\footnotesize] [align=left] {64};
% Text Node
\draw (304.8,155.2) node [anchor=north west][inner sep=0.75pt]  [font=\footnotesize] [align=left] {69};
% Text Node
\draw (264.4,75.6) node [anchor=north west][inner sep=0.75pt]  [font=\footnotesize] [align=left] {26};
% Text Node
\draw (285.2,75.6) node [anchor=north west][inner sep=0.75pt]  [font=\footnotesize] [align=left] {29};
% Text Node
\draw (304.8,75.2) node [anchor=north west][inner sep=0.75pt]  [font=\footnotesize] [align=left] {28};
% Text Node
\draw (324.8,95.2) node [anchor=north west][inner sep=0.75pt]  [font=\footnotesize] [align=left] {63};
% Text Node
\draw (324.8,76.8) node [anchor=north west][inner sep=0.75pt]  [font=\footnotesize] [align=left] {70};
% Text Node
\draw (244.8,115.6) node [anchor=north west][inner sep=0.75pt]  [font=\footnotesize] [align=left] {28};
% Text Node
\draw (264.8,135.6) node [anchor=north west][inner sep=0.75pt]  [font=\footnotesize] [align=left] {66};
% Text Node
\draw (284.8,155.2) node [anchor=north west][inner sep=0.75pt]  [font=\footnotesize] [align=left] {77};
% Text Node
\draw (244,136) node [anchor=north west][inner sep=0.75pt]  [font=\footnotesize] [align=left] {63};
% Text Node
\draw (264,155.2) node [anchor=north west][inner sep=0.75pt]  [font=\footnotesize] [align=left] {74};
% Text Node
\draw (245.2,155.6) node [anchor=north west][inner sep=0.75pt]  [font=\footnotesize] [align=left] {73};
% Text Node
\draw (324.8,136.4) node [anchor=north west][inner sep=0.75pt]  [font=\footnotesize] [align=left] {68};
% Text Node
\draw (285.2,116.4) node [anchor=north west][inner sep=0.75pt]  [font=\footnotesize] [align=left] {23};
% Text Node
\draw (355,80) node [anchor=north west][inner sep=0.75pt]  [font=\footnotesize]  {$w=2$};
% Text Node
\draw (355,114.33) node [anchor=north west][inner sep=0.75pt]  [font=\footnotesize,color={rgb, 255:red, 72; green, 36; blue, 227 }  ,opacity=1 ] [align=left] {Var(1,0)=386.8};
% Text Node
\draw (355,98.2) node [anchor=north west][inner sep=0.75pt]  [font=\footnotesize]  {$T=300$};
% Text Node
\draw (355,124.33) node [anchor=north west][inner sep=0.75pt]  [font=\footnotesize,color={rgb, 255:red, 72; green, 36; blue, 227 }  ,opacity=1 ] [align=left] {Var(0,1)=526.8};
% Text Node
\draw (355,133.73) node [anchor=north west][inner sep=0.75pt]  [font=\footnotesize,color={rgb, 255:red, 72; green, 36; blue, 227 }  ,opacity=1 ] [align=left] {Var(1,1)=439.7};
% Text Node
\draw (355,143.93) node [anchor=north west][inner sep=0.75pt]  [font=\footnotesize,color={rgb, 255:red, 72; green, 36; blue, 227 }  ,opacity=1 ] [align=left] {Var(1,-1)=471.2};
% Text Node
\draw (356.23,157.5) node[anchor=north west, inner sep=0.75pt, font=\footnotesize\bfseries] {$S = 386.8$};
\draw (356.23,170.2) node[anchor=north west, inner sep=0.75pt, font=\footnotesize\bfseries] {$S > T \Longrightarrow$};
\draw (388.93,170.2) node[anchor=north west, inner sep=0.75pt, font=\footnotesize\bfseries, align=left] {point d’intérêt};
}
    \end{tikzpicture}
  \end{overlayarea}
\end{frame}

\begin{frame}
      \hyperlink{algo-moravec}{
    \beamerbutton{Voir le schéma détaillé}
  }

\end{frame}

\begin{frame}
    \label{moravec-code}
\end{frame}


\begin{frame}
  \label{projection-appendix}
\frametitle{Les différents repères}

\begin{minipage}[c]{0.48\linewidth}
  \centering
  \begin{overlayarea}{0.9\linewidth}{4cm}
    \hspace*{-1cm}
    \begin{tikzpicture}[x=0.75pt,y=0.75pt,yscale=-1,xscale=1, scale=0.6]
    \only<1>{


\tikzset{every picture/.style={line width=0.75pt}} %set default line width to 0.75pt        

\begin{tikzpicture}[x=0.75pt,y=0.75pt,yscale=-1,xscale=1]
%uncomment if require: \path (0,300); %set diagram left start at 0, and has height of 300

%Shape: Rectangle [id:dp9723727103024109] 
\draw  [line width=0.75]  (273.11,2.68) -- (273.16,155.68) -- (150.53,258.66) -- (150.47,105.66) -- cycle ;
%Shape: Circle [id:dp9239162914529644] 
\draw   (170.87,132.32) .. controls (172.23,131.96) and (173.33,132.77) .. (173.33,134.13) .. controls (173.33,135.49) and (172.23,136.89) .. (170.87,137.26) .. controls (169.5,137.62) and (168.4,136.81) .. (168.4,135.45) .. controls (168.4,134.09) and (169.5,132.69) .. (170.87,132.32) -- cycle ;
%Straight Lines [id:da4507743094695521] 
\draw  [dash pattern={on 0.84pt off 2.51pt}]  (211.82,130.67) -- (45,134.7) ;
%Shape: Circle [id:dp9335082203579057] 
\draw   (499.62,131.92) .. controls (500.98,131.55) and (502.08,132.36) .. (502.08,133.72) .. controls (502.08,135.09) and (500.98,136.49) .. (499.62,136.85) .. controls (498.25,137.22) and (497.15,136.41) .. (497.15,135.05) .. controls (497.15,133.68) and (498.25,132.28) .. (499.62,131.92) -- cycle ;
%Shape: Circle [id:dp5162114026007054] 
\draw  [color={rgb, 255:red, 241; green, 53; blue, 53 }  ,draw opacity=1 ] (512.82,105.92) .. controls (514.18,105.55) and (515.28,106.36) .. (515.28,107.72) .. controls (515.28,109.09) and (514.18,110.49) .. (512.82,110.85) .. controls (511.45,111.22) and (510.35,110.41) .. (510.35,109.05) .. controls (510.35,107.68) and (511.45,106.28) .. (512.82,105.92) -- cycle ;
%Shape: Circle [id:dp4319786220460511] 
\draw  [fill={rgb, 255:red, 0; green, 0; blue, 0 }  ,fill opacity=1 ] (321.12,132.42) .. controls (322.48,132.42) and (323.58,133.52) .. (323.58,134.88) .. controls (323.58,136.25) and (322.48,137.35) .. (321.12,137.35) .. controls (319.75,137.35) and (318.65,136.25) .. (318.65,134.88) .. controls (318.65,133.52) and (319.75,132.42) .. (321.12,132.42) -- cycle ;
%Straight Lines [id:da5164589906920809] 
\draw    (321.12,134.88) -- (321.97,70.75) ;
\draw [shift={(322,68.75)}, rotate = 90.77] [color={rgb, 255:red, 0; green, 0; blue, 0 }  ][line width=0.75]    (10.93,-3.29) .. controls (6.95,-1.4) and (3.31,-0.3) .. (0,0) .. controls (3.31,0.3) and (6.95,1.4) .. (10.93,3.29)   ;
%Straight Lines [id:da7854412907443913] 
\draw    (321.12,134.88) -- (253.5,134.75) ;
\draw [shift={(251.5,134.75)}, rotate = 0.11] [color={rgb, 255:red, 0; green, 0; blue, 0 }  ][line width=0.75]    (10.93,-3.29) .. controls (6.95,-1.4) and (3.31,-0.3) .. (0,0) .. controls (3.31,0.3) and (6.95,1.4) .. (10.93,3.29)   ;
%Straight Lines [id:da3058610918194634] 
\draw    (321.12,134.88) -- (272.48,176.16) ;
\draw [shift={(270.95,177.45)}, rotate = 319.68] [color={rgb, 255:red, 0; green, 0; blue, 0 }  ][line width=0.75]    (10.93,-3.29) .. controls (6.95,-1.4) and (3.31,-0.3) .. (0,0) .. controls (3.31,0.3) and (6.95,1.4) .. (10.93,3.29)   ;
%Straight Lines [id:da5453100201962917] 
\draw  [dash pattern={on 4.5pt off 4.5pt}]  (149.63,172.62) -- (150.47,105.66) ;
\draw [shift={(149.61,174.62)}, rotate = 270.72] [color={rgb, 255:red, 0; green, 0; blue, 0 }  ][line width=0.75]    (10.93,-3.29) .. controls (6.95,-1.4) and (3.31,-0.3) .. (0,0) .. controls (3.31,0.3) and (6.95,1.4) .. (10.93,3.29)   ;
%Straight Lines [id:da4691798453437107] 
\draw  [dash pattern={on 4.5pt off 4.5pt}]  (199.11,64.39) -- (150.47,105.66) ;
\draw [shift={(200.64,63.1)}, rotate = 139.68] [color={rgb, 255:red, 0; green, 0; blue, 0 }  ][line width=0.75]    (10.93,-3.29) .. controls (6.95,-1.4) and (3.31,-0.3) .. (0,0) .. controls (3.31,0.3) and (6.95,1.4) .. (10.93,3.29)   ;
%Straight Lines [id:da9224741393814412] 
\draw [color={rgb, 255:red, 241; green, 45; blue, 45 }  ,draw opacity=1 ]   (512.82,108.38) -- (192.48,152.97) ;
\draw [shift={(190.5,153.25)}, rotate = 352.08] [color={rgb, 255:red, 241; green, 45; blue, 45 }  ,draw opacity=1 ][line width=0.75]    (10.93,-3.29) .. controls (6.95,-1.4) and (3.31,-0.3) .. (0,0) .. controls (3.31,0.3) and (6.95,1.4) .. (10.93,3.29)   ;
%Shape: Circle [id:dp7585208638434278] 
\draw  [color={rgb, 255:red, 241; green, 53; blue, 53 }  ,draw opacity=1 ] (190.5,150.78) .. controls (191.86,150.42) and (192.97,151.23) .. (192.97,152.59) .. controls (192.97,153.95) and (191.86,155.35) .. (190.5,155.72) .. controls (189.14,156.08) and (188.03,155.27) .. (188.03,153.91) .. controls (188.03,152.55) and (189.14,151.15) .. (190.5,150.78) -- cycle ;
%Shape: Cube [id:dp7714367112041981] 
\draw   (499.62,131.92) -- (520.32,111.22) -- (568.62,111.22) -- (568.62,160.52) -- (547.92,181.22) -- (499.62,181.22) -- cycle ; \draw   (568.62,111.22) -- (547.92,131.92) -- (499.62,131.92) ; \draw   (547.92,131.92) -- (547.92,181.22) ;

% Text Node
\draw (517,97.9) node [anchor=north west][inner sep=0.75pt]  [font=\footnotesize,color={rgb, 255:red, 167; green, 17; blue, 17 }  ,opacity=1 ]  {$P$};
% Text Node
\draw (324,59.9) node [anchor=north west][inner sep=0.75pt]  [font=\footnotesize]  {$\vec{j}$};
% Text Node
\draw (280.5,178.9) node [anchor=north west][inner sep=0.75pt]  [font=\footnotesize]  {$\vec{i}$};
% Text Node
\draw (324.43,139.84) node [anchor=north west][inner sep=0.75pt]  [font=\footnotesize]  {$O$};
% Text Node
\draw (252,112.9) node [anchor=north west][inner sep=0.75pt]  [font=\footnotesize]  {$\vec{k}$};
% Text Node
\draw (158.65,118.28) node [anchor=north west][inner sep=0.75pt]  [font=\footnotesize]  {$C$};
% Text Node
\draw (156,62.9) node [anchor=north west][inner sep=0.75pt]  [font=\footnotesize]  {$\vec{u}$};
% Text Node
\draw (125.5,114.4) node [anchor=north west][inner sep=0.75pt]  [font=\footnotesize]  {$\vec{v}$};
% Text Node
\draw (184.5,155.9) node [anchor=north west][inner sep=0.75pt]  [font=\footnotesize,color={rgb, 255:red, 159; green, 16; blue, 16 }  ,opacity=1 ]  {$P'$};


\end{tikzpicture}}
    \only<2>{
%Shape: Rectangle [id:dp4318964674614302] 
\draw  [line width=0.75]  (261.98,12.03) -- (262.02,142.84) -- (150.52,236.47) -- (150.47,105.66) -- cycle ;
%Shape: Circle [id:dp8619322586517022] 
\draw  [color={rgb, 255:red, 16; green, 18; blue, 125 }  ,draw opacity=1 ][fill={rgb, 255:red, 16; green, 18; blue, 125 }  ,fill opacity=1 ] (350.47,93.07) .. controls (351.83,92.71) and (352.93,93.52) .. (352.93,94.88) .. controls (352.93,96.24) and (351.83,97.64) .. (350.47,98.01) .. controls (349.1,98.37) and (348,97.56) .. (348,96.2) .. controls (348,94.84) and (349.1,93.44) .. (350.47,93.07) -- cycle ;
%Shape: Cube [id:dp5712797611314612] 
\draw   (325.13,88.8) -- (342.27,71.67) -- (382.25,71.67) -- (382.25,112.75) -- (365.12,129.88) -- (325.13,129.88) -- cycle ; \draw   (382.25,71.67) -- (365.12,88.8) -- (325.13,88.8) ; \draw   (365.12,88.8) -- (365.12,129.88) ;
%Shape: Rectangle [id:dp6697517945708218] 
\draw  [color={rgb, 255:red, 255; green, 255; blue, 255 }  ,draw opacity=1 ] (41.02,5.34) -- (469,5.34) -- (469,280.34) -- (41.02,280.34) -- cycle ;

% Text Node
\draw (358,50.9) node [anchor=north west][inner sep=0.75pt]  [font=\footnotesize,color={rgb, 255:red, 16; green, 18; blue, 125 }  ,opacity=1 ]  {$M$};}
    \only<3>{\input{donnees/s2_3.tex}}
    \only<4>{\input{donnees/s2_4.tex}}
    \only<5>{
\draw  [line width=0.75]  (261.98,12.03) -- (262.02,142.84) -- (150.52,236.47) -- (150.47,105.66) -- cycle ;
%Shape: Circle [id:dp0391503239902703] 
\draw  [color={rgb, 255:red, 16; green, 18; blue, 125 }  ,draw opacity=1 ][fill={rgb, 255:red, 16; green, 18; blue, 125 }  ,fill opacity=1 ] (350.47,93.07) .. controls (351.83,92.71) and (352.93,93.52) .. (352.93,94.88) .. controls (352.93,96.24) and (351.83,97.64) .. (350.47,98.01) .. controls (349.1,98.37) and (348,97.56) .. (348,96.2) .. controls (348,94.84) and (349.1,93.44) .. (350.47,93.07) -- cycle ;
%Shape: Circle [id:dp8654606077158083] 
\draw  [fill={rgb, 255:red, 0; green, 0; blue, 0 }  ,fill opacity=1 ] (365.12,130.42) .. controls (366.48,130.42) and (367.58,131.52) .. (367.58,132.88) .. controls (367.58,134.25) and (366.48,135.35) .. (365.12,135.35) .. controls (363.75,135.35) and (362.65,134.25) .. (362.65,132.88) .. controls (362.65,131.52) and (363.75,130.42) .. (365.12,130.42) -- cycle ;
%Straight Lines [id:da3675337375445349] 
\draw  [dash pattern={on 4.5pt off 4.5pt}]  (150.02,143.52) -- (150.47,105.66) ;
\draw [shift={(150,145.52)}, rotate = 270.68] [color={rgb, 255:red, 0; green, 0; blue, 0 }  ][line width=0.75]    (10.93,-3.29) .. controls (6.95,-1.4) and (3.31,-0.3) .. (0,0) .. controls (3.31,0.3) and (6.95,1.4) .. (10.93,3.29)   ;
%Straight Lines [id:da2135082456633841] 
\draw  [dash pattern={on 4.5pt off 4.5pt}]  (182.47,78.8) -- (150.47,105.66) ;
\draw [shift={(184,77.52)}, rotate = 139.99] [color={rgb, 255:red, 0; green, 0; blue, 0 }  ][line width=0.75]    (10.93,-3.29) .. controls (6.95,-1.4) and (3.31,-0.3) .. (0,0) .. controls (3.31,0.3) and (6.95,1.4) .. (10.93,3.29)   ;
%Shape: Circle [id:dp3919067214179274] 
\draw  [color={rgb, 255:red, 16; green, 18; blue, 125 }  ,draw opacity=1 ] (209.53,118.71) .. controls (210.9,118.35) and (212,119.15) .. (212,120.52) .. controls (212,121.88) and (210.9,123.28) .. (209.53,123.64) .. controls (208.17,124.01) and (207.07,123.2) .. (207.07,121.84) .. controls (207.07,120.48) and (208.17,119.08) .. (209.53,118.71) -- cycle ;
%Straight Lines [id:da13961882146045834] 
\draw [color={rgb, 255:red, 16; green, 18; blue, 125 }  ,draw opacity=1 ][fill={rgb, 255:red, 16; green, 18; blue, 125 }  ,fill opacity=1 ]   (350.47,95.54) -- (212,120.52) ;
%Straight Lines [id:da6009765896576962] 
\draw    (365.12,130.42) -- (331,129.27) ;
\draw [shift={(329,129.2)}, rotate = 1.93] [color={rgb, 255:red, 0; green, 0; blue, 0 }  ][line width=0.75]    (10.93,-3.29) .. controls (6.95,-1.4) and (3.31,-0.3) .. (0,0) .. controls (3.31,0.3) and (6.95,1.4) .. (10.93,3.29)   ;
%Straight Lines [id:da13460515919081129] 
\draw    (365.12,130.42) -- (380.86,114.19) ;
\draw [shift={(382.25,112.75)}, rotate = 134.12] [color={rgb, 255:red, 0; green, 0; blue, 0 }  ][line width=0.75]    (10.93,-3.29) .. controls (6.95,-1.4) and (3.31,-0.3) .. (0,0) .. controls (3.31,0.3) and (6.95,1.4) .. (10.93,3.29)   ;
%Straight Lines [id:da5115010168409273] 
\draw    (364.65,132.88) -- (364.04,100.2) ;
\draw [shift={(364,98.2)}, rotate = 88.93] [color={rgb, 255:red, 0; green, 0; blue, 0 }  ][line width=0.75]    (10.93,-3.29) .. controls (6.95,-1.4) and (3.31,-0.3) .. (0,0) .. controls (3.31,0.3) and (6.95,1.4) .. (10.93,3.29)   ;
%Shape: Cube [id:dp47417096557677685] 
\draw   (325.13,88.8) -- (342.27,71.67) -- (382.25,71.67) -- (382.25,112.75) -- (365.12,129.88) -- (325.13,129.88) -- cycle ; \draw   (382.25,71.67) -- (365.12,88.8) -- (325.13,88.8) ; \draw   (365.12,88.8) -- (365.12,129.88) ;
%Shape: Rectangle [id:dp12317704811349806] 
\draw  [color={rgb, 255:red, 255; green, 255; blue, 255 }  ,draw opacity=1 ] (41.02,5.34) -- (469,5.34) -- (469,280.34) -- (41.02,280.34) -- cycle ;

% Text Node
\draw (358,50.9) node [anchor=north west][inner sep=0.75pt]  [font=\footnotesize,color={rgb, 255:red, 16; green, 18; blue, 125 }  ,opacity=1 ]  {$M$};
% Text Node
\draw (364.65,136.28) node [anchor=north west][inner sep=0.75pt]  [font=\footnotesize]  {$W$};
% Text Node
\draw (154,71.9) node [anchor=north west][inner sep=0.75pt]  [font=\footnotesize]  {$u'$};
% Text Node
\draw (129.5,109.4) node [anchor=north west][inner sep=0.75pt]  [font=\footnotesize]  {$v'$};
% Text Node
\draw (210,99.9) node [anchor=north west][inner sep=0.75pt]  [font=\footnotesize,color={rgb, 255:red, 167; green, 17; blue, 17 }  ,opacity=1 ]  {$\textcolor[rgb]{0.06,0.07,0.49}{m}$};
}
    \only<6>{
%Shape: Rectangle [id:dp22091086801975413] 
\draw  [line width=0.75]  (261.98,12.03) -- (262.02,142.84) -- (150.52,236.47) -- (150.47,105.66) -- cycle ;
%Shape: Circle [id:dp3960031807548443] 
\draw  [color={rgb, 255:red, 16; green, 18; blue, 125 }  ,draw opacity=1 ][fill={rgb, 255:red, 16; green, 18; blue, 125 }  ,fill opacity=1 ] (350.47,93.07) .. controls (351.83,92.71) and (352.93,93.52) .. (352.93,94.88) .. controls (352.93,96.24) and (351.83,97.64) .. (350.47,98.01) .. controls (349.1,98.37) and (348,97.56) .. (348,96.2) .. controls (348,94.84) and (349.1,93.44) .. (350.47,93.07) -- cycle ;
%Shape: Circle [id:dp7170017338469822] 
\draw  [fill={rgb, 255:red, 0; green, 0; blue, 0 }  ,fill opacity=1 ] (365.12,130.42) .. controls (366.48,130.42) and (367.58,131.52) .. (367.58,132.88) .. controls (367.58,134.25) and (366.48,135.35) .. (365.12,135.35) .. controls (363.75,135.35) and (362.65,134.25) .. (362.65,132.88) .. controls (362.65,131.52) and (363.75,130.42) .. (365.12,130.42) -- cycle ;
%Straight Lines [id:da925005109202037] 
\draw  [dash pattern={on 4.5pt off 4.5pt}]  (150.02,143.52) -- (150.47,105.66) ;
\draw [shift={(150,145.52)}, rotate = 270.68] [color={rgb, 255:red, 0; green, 0; blue, 0 }  ][line width=0.75]    (10.93,-3.29) .. controls (6.95,-1.4) and (3.31,-0.3) .. (0,0) .. controls (3.31,0.3) and (6.95,1.4) .. (10.93,3.29)   ;
%Straight Lines [id:da7341104591578826] 
\draw  [dash pattern={on 4.5pt off 4.5pt}]  (182.47,78.8) -- (150.47,105.66) ;
\draw [shift={(184,77.52)}, rotate = 139.99] [color={rgb, 255:red, 0; green, 0; blue, 0 }  ][line width=0.75]    (10.93,-3.29) .. controls (6.95,-1.4) and (3.31,-0.3) .. (0,0) .. controls (3.31,0.3) and (6.95,1.4) .. (10.93,3.29)   ;
%Shape: Circle [id:dp054443755291766927] 
\draw  [color={rgb, 255:red, 16; green, 18; blue, 125 }  ,draw opacity=1 ] (209.53,118.71) .. controls (210.9,118.35) and (212,119.15) .. (212,120.52) .. controls (212,121.88) and (210.9,123.28) .. (209.53,123.64) .. controls (208.17,124.01) and (207.07,123.2) .. (207.07,121.84) .. controls (207.07,120.48) and (208.17,119.08) .. (209.53,118.71) -- cycle ;
%Straight Lines [id:da5500031803707174] 
\draw [color={rgb, 255:red, 16; green, 18; blue, 125 }  ,draw opacity=1 ][fill={rgb, 255:red, 16; green, 18; blue, 125 }  ,fill opacity=1 ]   (350.47,95.54) -- (212,120.52) ;
%Straight Lines [id:da19418058541638872] 
\draw    (365.12,130.42) -- (331,129.27) ;
\draw [shift={(329,129.2)}, rotate = 1.93] [color={rgb, 255:red, 0; green, 0; blue, 0 }  ][line width=0.75]    (10.93,-3.29) .. controls (6.95,-1.4) and (3.31,-0.3) .. (0,0) .. controls (3.31,0.3) and (6.95,1.4) .. (10.93,3.29)   ;
%Straight Lines [id:da06456179839803622] 
\draw    (365.12,130.42) -- (380.86,114.19) ;
\draw [shift={(382.25,112.75)}, rotate = 134.12] [color={rgb, 255:red, 0; green, 0; blue, 0 }  ][line width=0.75]    (10.93,-3.29) .. controls (6.95,-1.4) and (3.31,-0.3) .. (0,0) .. controls (3.31,0.3) and (6.95,1.4) .. (10.93,3.29)   ;
%Straight Lines [id:da6878817592231475] 
\draw    (364.65,132.88) -- (364.04,100.2) ;
\draw [shift={(364,98.2)}, rotate = 88.93] [color={rgb, 255:red, 0; green, 0; blue, 0 }  ][line width=0.75]    (10.93,-3.29) .. controls (6.95,-1.4) and (3.31,-0.3) .. (0,0) .. controls (3.31,0.3) and (6.95,1.4) .. (10.93,3.29)   ;
%Shape: Cube [id:dp39297068054343187] 
\draw   (325.13,88.8) -- (342.27,71.67) -- (382.25,71.67) -- (382.25,112.75) -- (365.12,129.88) -- (325.13,129.88) -- cycle ; \draw   (382.25,71.67) -- (365.12,88.8) -- (325.13,88.8) ; \draw   (365.12,88.8) -- (365.12,129.88) ;
%Straight Lines [id:da3701505179737167] 
\draw [color={rgb, 255:red, 0; green, 0; blue, 0 }  ,draw opacity=1 ]   (75.12,154.88) -- (75.97,90.75) ;
\draw [shift={(76,88.75)}, rotate = 90.77] [color={rgb, 255:red, 0; green, 0; blue, 0 }  ,draw opacity=1 ][line width=0.75]    (10.93,-3.29) .. controls (6.95,-1.4) and (3.31,-0.3) .. (0,0) .. controls (3.31,0.3) and (6.95,1.4) .. (10.93,3.29)   ;
%Straight Lines [id:da9863793705895071] 
\draw [color={rgb, 255:red, 0; green, 0; blue, 0 }  ,draw opacity=1 ]   (75.12,154.88) -- (116.31,128.59) ;
\draw [shift={(118,127.52)}, rotate = 147.45] [color={rgb, 255:red, 0; green, 0; blue, 0 }  ,draw opacity=1 ][line width=0.75]    (10.93,-3.29) .. controls (6.95,-1.4) and (3.31,-0.3) .. (0,0) .. controls (3.31,0.3) and (6.95,1.4) .. (10.93,3.29)   ;
%Straight Lines [id:da21691112687890424] 
\draw [color={rgb, 255:red, 0; green, 0; blue, 0 }  ,draw opacity=1 ]   (75.12,154.88) -- (128,153.57) ;
\draw [shift={(130,153.52)}, rotate = 178.57] [color={rgb, 255:red, 0; green, 0; blue, 0 }  ,draw opacity=1 ][line width=0.75]    (10.93,-3.29) .. controls (6.95,-1.4) and (3.31,-0.3) .. (0,0) .. controls (3.31,0.3) and (6.95,1.4) .. (10.93,3.29)   ;
%Straight Lines [id:da5973484313893659] 
\draw [color={rgb, 255:red, 16; green, 18; blue, 125 }  ,draw opacity=1 ]   (157,133.52) -- (75.12,154.88) ;
%Straight Lines [id:da9607014603660214] 
\draw [color={rgb, 255:red, 0; green, 0; blue, 0 }  ,draw opacity=1 ] [dash pattern={on 0.84pt off 2.51pt}]  (211.07,120.84) -- (157,133.52) ;
%Shape: Rectangle [id:dp47930549648846155] 
\draw  [color={rgb, 255:red, 255; green, 255; blue, 255 }  ,draw opacity=1 ] (41.02,5.34) -- (469,5.34) -- (469,280.34) -- (41.02,280.34) -- cycle ;

% Text Node
\draw (358,50.9) node [anchor=north west][inner sep=0.75pt]  [font=\footnotesize,color={rgb, 255:red, 16; green, 18; blue, 125 }  ,opacity=1 ]  {$M$};
% Text Node
\draw (364.65,136.28) node [anchor=north west][inner sep=0.75pt]  [font=\footnotesize]  {$W$};
% Text Node
\draw (154,71.9) node [anchor=north west][inner sep=0.75pt]  [font=\footnotesize]  {$u'$};
% Text Node
\draw (129.5,109.4) node [anchor=north west][inner sep=0.75pt]  [font=\footnotesize]  {$v'$};
% Text Node
\draw (210,99.9) node [anchor=north west][inner sep=0.75pt]  [font=\footnotesize,color={rgb, 255:red, 167; green, 17; blue, 17 }  ,opacity=1 ]  {$\textcolor[rgb]{0.06,0.07,0.49}{m}$};
% Text Node
\draw (56.65,159.28) node [anchor=north west][inner sep=0.75pt]  [font=\footnotesize]  {$C$};
% Text Node
\draw (61,70.9) node [anchor=north west][inner sep=0.75pt]  [font=\footnotesize]  {$y$};
}
    \only<7>{\input{donnees/s2_7.tex}}
    \only<8>{
%Shape: Rectangle [id:dp20518050829306933] 
\draw  [line width=0.75]  (261.98,12.03) -- (262.02,142.84) -- (150.52,236.47) -- (150.47,105.66) -- cycle ;
%Shape: Circle [id:dp48036249842364387] 
\draw  [color={rgb, 255:red, 16; green, 18; blue, 125 }  ,draw opacity=1 ][fill={rgb, 255:red, 16; green, 18; blue, 125 }  ,fill opacity=1 ] (350.47,93.07) .. controls (351.83,92.71) and (352.93,93.52) .. (352.93,94.88) .. controls (352.93,96.24) and (351.83,97.64) .. (350.47,98.01) .. controls (349.1,98.37) and (348,97.56) .. (348,96.2) .. controls (348,94.84) and (349.1,93.44) .. (350.47,93.07) -- cycle ;
%Shape: Circle [id:dp15975553735025072] 
\draw  [fill={rgb, 255:red, 0; green, 0; blue, 0 }  ,fill opacity=1 ] (365.12,130.42) .. controls (366.48,130.42) and (367.58,131.52) .. (367.58,132.88) .. controls (367.58,134.25) and (366.48,135.35) .. (365.12,135.35) .. controls (363.75,135.35) and (362.65,134.25) .. (362.65,132.88) .. controls (362.65,131.52) and (363.75,130.42) .. (365.12,130.42) -- cycle ;
%Straight Lines [id:da04569264629085701] 
\draw  [dash pattern={on 4.5pt off 4.5pt}]  (150.02,143.52) -- (150.47,105.66) ;
\draw [shift={(150,145.52)}, rotate = 270.68] [color={rgb, 255:red, 0; green, 0; blue, 0 }  ][line width=0.75]    (10.93,-3.29) .. controls (6.95,-1.4) and (3.31,-0.3) .. (0,0) .. controls (3.31,0.3) and (6.95,1.4) .. (10.93,3.29)   ;
%Straight Lines [id:da34135028023903646] 
\draw  [dash pattern={on 4.5pt off 4.5pt}]  (182.47,78.8) -- (150.47,105.66) ;
\draw [shift={(184,77.52)}, rotate = 139.99] [color={rgb, 255:red, 0; green, 0; blue, 0 }  ][line width=0.75]    (10.93,-3.29) .. controls (6.95,-1.4) and (3.31,-0.3) .. (0,0) .. controls (3.31,0.3) and (6.95,1.4) .. (10.93,3.29)   ;
%Shape: Circle [id:dp9804788379329312] 
\draw  [color={rgb, 255:red, 16; green, 18; blue, 125 }  ,draw opacity=1 ] (209.53,118.71) .. controls (210.9,118.35) and (212,119.15) .. (212,120.52) .. controls (212,121.88) and (210.9,123.28) .. (209.53,123.64) .. controls (208.17,124.01) and (207.07,123.2) .. (207.07,121.84) .. controls (207.07,120.48) and (208.17,119.08) .. (209.53,118.71) -- cycle ;
%Straight Lines [id:da9492695065393072] 
\draw [color={rgb, 255:red, 16; green, 18; blue, 125 }  ,draw opacity=1 ][fill={rgb, 255:red, 16; green, 18; blue, 125 }  ,fill opacity=1 ]   (350.47,95.54) -- (212,120.52) ;
%Straight Lines [id:da3349950281049371] 
\draw    (365.12,130.42) -- (331,129.27) ;
\draw [shift={(329,129.2)}, rotate = 1.93] [color={rgb, 255:red, 0; green, 0; blue, 0 }  ][line width=0.75]    (10.93,-3.29) .. controls (6.95,-1.4) and (3.31,-0.3) .. (0,0) .. controls (3.31,0.3) and (6.95,1.4) .. (10.93,3.29)   ;
%Straight Lines [id:da036984224071377914] 
\draw    (365.12,130.42) -- (380.86,114.19) ;
\draw [shift={(382.25,112.75)}, rotate = 134.12] [color={rgb, 255:red, 0; green, 0; blue, 0 }  ][line width=0.75]    (10.93,-3.29) .. controls (6.95,-1.4) and (3.31,-0.3) .. (0,0) .. controls (3.31,0.3) and (6.95,1.4) .. (10.93,3.29)   ;
%Straight Lines [id:da6787819066580656] 
\draw    (364.65,132.88) -- (364.04,100.2) ;
\draw [shift={(364,98.2)}, rotate = 88.93] [color={rgb, 255:red, 0; green, 0; blue, 0 }  ][line width=0.75]    (10.93,-3.29) .. controls (6.95,-1.4) and (3.31,-0.3) .. (0,0) .. controls (3.31,0.3) and (6.95,1.4) .. (10.93,3.29)   ;
%Shape: Cube [id:dp8344554965778904] 
\draw   (325.13,88.8) -- (342.27,71.67) -- (382.25,71.67) -- (382.25,112.75) -- (365.12,129.88) -- (325.13,129.88) -- cycle ; \draw   (382.25,71.67) -- (365.12,88.8) -- (325.13,88.8) ; \draw   (365.12,88.8) -- (365.12,129.88) ;
%Straight Lines [id:da9940442974254676] 
\draw [color={rgb, 255:red, 0; green, 0; blue, 0 }  ,draw opacity=1 ] [dash pattern={on 0.84pt off 2.51pt}]  (338,148.2) -- (75.12,154.88) ;
%Straight Lines [id:da09913909765003492] 
\draw [color={rgb, 255:red, 0; green, 0; blue, 0 }  ,draw opacity=1 ]   (75.12,154.88) -- (75.97,90.75) ;
\draw [shift={(76,88.75)}, rotate = 90.77] [color={rgb, 255:red, 0; green, 0; blue, 0 }  ,draw opacity=1 ][line width=0.75]    (10.93,-3.29) .. controls (6.95,-1.4) and (3.31,-0.3) .. (0,0) .. controls (3.31,0.3) and (6.95,1.4) .. (10.93,3.29)   ;
%Straight Lines [id:da954541250044646] 
\draw [color={rgb, 255:red, 0; green, 0; blue, 0 }  ,draw opacity=1 ]   (75.12,154.88) -- (116.31,128.59) ;
\draw [shift={(118,127.52)}, rotate = 147.45] [color={rgb, 255:red, 0; green, 0; blue, 0 }  ,draw opacity=1 ][line width=0.75]    (10.93,-3.29) .. controls (6.95,-1.4) and (3.31,-0.3) .. (0,0) .. controls (3.31,0.3) and (6.95,1.4) .. (10.93,3.29)   ;
%Straight Lines [id:da8173811873321303] 
\draw [color={rgb, 255:red, 0; green, 0; blue, 0 }  ,draw opacity=1 ]   (75.12,154.88) -- (128,153.57) ;
\draw [shift={(130,153.52)}, rotate = 178.57] [color={rgb, 255:red, 0; green, 0; blue, 0 }  ,draw opacity=1 ][line width=0.75]    (10.93,-3.29) .. controls (6.95,-1.4) and (3.31,-0.3) .. (0,0) .. controls (3.31,0.3) and (6.95,1.4) .. (10.93,3.29)   ;
%Straight Lines [id:da6590312752180065] 
\draw [color={rgb, 255:red, 16; green, 18; blue, 125 }  ,draw opacity=1 ]   (157,133.52) -- (75.12,154.88) ;
%Straight Lines [id:da9534946249718639] 
\draw [color={rgb, 255:red, 0; green, 0; blue, 0 }  ,draw opacity=1 ]   (86,179.5) -- (192,178.53) ;
\draw [shift={(194,178.52)}, rotate = 179.48] [color={rgb, 255:red, 0; green, 0; blue, 0 }  ,draw opacity=1 ][line width=0.75]    (10.93,-3.29) .. controls (6.95,-1.4) and (3.31,-0.3) .. (0,0) .. controls (3.31,0.3) and (6.95,1.4) .. (10.93,3.29)   ;
\draw [shift={(84,179.52)}, rotate = 359.48] [color={rgb, 255:red, 0; green, 0; blue, 0 }  ,draw opacity=1 ][line width=0.75]    (10.93,-3.29) .. controls (6.95,-1.4) and (3.31,-0.3) .. (0,0) .. controls (3.31,0.3) and (6.95,1.4) .. (10.93,3.29)   ;
%Straight Lines [id:da2603521647004613] 
\draw [color={rgb, 255:red, 0; green, 0; blue, 0 }  ,draw opacity=1 ] [dash pattern={on 0.84pt off 2.51pt}]  (211.07,120.84) -- (157,133.52) ;
%Shape: Rectangle [id:dp15198164582240803] 
\draw  [color={rgb, 255:red, 255; green, 255; blue, 255 }  ,draw opacity=1 ] (41.02,5.34) -- (469,5.34) -- (469,280.34) -- (41.02,280.34) -- cycle ;
%Straight Lines [id:da6760200769193868] 
\draw    (202.02,152.01) -- (201.04,102.52) ;
\draw [shift={(201,100.52)}, rotate = 88.86] [color={rgb, 255:red, 0; green, 0; blue, 0 }  ][line width=0.75]    (10.93,-3.29) .. controls (6.95,-1.4) and (3.31,-0.3) .. (0,0) .. controls (3.31,0.3) and (6.95,1.4) .. (10.93,3.29)   ;
%Straight Lines [id:da9313328046524404] 
\draw    (202.56,151.13) -- (235.26,132.51) ;
\draw [shift={(237,131.52)}, rotate = 150.34] [color={rgb, 255:red, 0; green, 0; blue, 0 }  ][line width=0.75]    (10.93,-3.29) .. controls (6.95,-1.4) and (3.31,-0.3) .. (0,0) .. controls (3.31,0.3) and (6.95,1.4) .. (10.93,3.29)   ;

% Text Node
\draw (358,50.9) node [anchor=north west][inner sep=0.75pt]  [font=\footnotesize,color={rgb, 255:red, 16; green, 18; blue, 125 }  ,opacity=1 ]  {$M$};
% Text Node
\draw (364.65,136.28) node [anchor=north west][inner sep=0.75pt]  [font=\footnotesize]  {$W$};
% Text Node
\draw (154,71.9) node [anchor=north west][inner sep=0.75pt]  [font=\footnotesize]  {$u'$};
% Text Node
\draw (129.5,109.4) node [anchor=north west][inner sep=0.75pt]  [font=\footnotesize]  {$v'$};
% Text Node
% Text Node
\draw (210,99.9) node [anchor=north west][inner sep=0.75pt]  [font=\footnotesize,color={rgb, 255:red, 167; green, 17; blue, 17 }  ,opacity=1 ]  {$\textcolor[rgb]{0.06,0.07,0.49}{m}$};
% Text Node
\draw (294,159) node [anchor=north west][inner sep=0.75pt]   [align=left] {axe optique};
% Text Node
\draw (126,192.9) node [anchor=north west][inner sep=0.75pt]  [font=\footnotesize]  {$f$};
% Text Node
\draw (56.65,159.28) node [anchor=north west][inner sep=0.75pt]  [font=\footnotesize]  {$C$};
% Text Node
\draw (61,70.9) node [anchor=north west][inner sep=0.75pt]  [font=\footnotesize]  {$y$};
% Text Node
\draw (101,112.9) node [anchor=north west][inner sep=0.75pt]  [font=\footnotesize]  {$x$};
% Text Node
\draw (184.65,156.6) node [anchor=north west][inner sep=0.75pt]  [font=\footnotesize]  {$C'$};
% Text Node
\draw (241,123.9) node [anchor=north west][inner sep=0.75pt]  [font=\footnotesize]  {$v$};
% Text Node
\draw (197,79.92) node [anchor=north west][inner sep=0.75pt]  [font=\footnotesize]  {$u$};

}
    \end{tikzpicture}
  \end{overlayarea}
\end{minipage}
\hfill
\begin{minipage}[c]{0.48\linewidth}
  \vspace*{\fill}
  \begin{itemize}
    \item<2-> $M$ : point réel
    \item<3-> $W$ : origine du repère du monde
    \item<4-> $(u', v')$ : coordonnées dans le plan image en pixels
    \item<5-> $m$ : projection de $M$ dans le plan image
    \item<6-> $C$ : origine du repère de la caméra
    \item<8-> $C'$ : origine du repère de l'image par projection de C
  \end{itemize}
  \vspace*{\fill}
\end{minipage}
\end{frame}

\begin{frame}{Projection d’un point 3D sur le plan image}
\begin{minipage}[c]{0.48\linewidth}
  \begin{overlayarea}{\linewidth}{5cm}
    \only<1>{
      Par le théorème de Thalès (projection perspective) \\[0.3em]
      $\begin{array}{rcl}
      u &=& f x_c \\
      v &=& f y_c \\
      w &=& z_c
      \end{array}$
    }

    \only<2>{
      Par le théorème de Thalès (projection perspective) \\[0.3em]
      $\begin{array}{rcl}
      u &=& f x_c \\
      v &=& f y_c \\
      w &=& z_c
      \end{array}$
      \\[0.5em]
      $\displaystyle
      \begin{bmatrix}
      u \\ v \\ w
      \end{bmatrix}
      =
      \begin{bmatrix}
      f & 0 & 0 & 0 \\
      0 & f & 0 & 0 \\
      0 & 0 & 1 & 0
      \end{bmatrix}
      \begin{bmatrix}
      x_c \\ y_c \\ z_c \\ 1
      \end{bmatrix}
      $
    }

    \only<3>{
      Le changement de repère s’écrit avec une transformation homogène :\\[0.5em]
      $\displaystyle
      \begin{bmatrix}
      x_c \\ y_c \\ z_c \\ 1
      \end{bmatrix}
      =
      \begin{bmatrix}
      R & T \\
      0 & 1
      \end{bmatrix}
      \begin{bmatrix}
      X_w \\ Y_w \\ Z_w \\ 1
      \end{bmatrix}
      $
  
      Où $R \in \mathbb{R}^{3\times3}$ est une rotation, $T \in \mathbb{R}^3$ une translation.
    }
  \end{overlayarea}
\end{minipage}
\hfill
\begin{minipage}[c]{0.48\linewidth}
  \centering
  \begin{overlayarea}{0.9\linewidth}{4cm}
    \hspace*{-1cm}
    \begin{tikzpicture}[x=0.75pt,y=0.75pt,yscale=-1,xscale=1, scale=0.6]
      \input{donnees/s2_9.tex}
    \end{tikzpicture}
  \end{overlayarea}
\end{minipage}
\end{frame}


\begin{frame}{Projection d’un point 3D sur le plan image}
  \centering
  \[
    \begin{bmatrix}
    u \\ v \\ w
    \end{bmatrix}
    =
    \underbrace{
    \begin{bmatrix}
    f & 0 & 0 & 0 \\
    0 & f & 0 & 0 \\
    0 & 0 & 1 & 0
    \end{bmatrix}
    \begin{bmatrix}
    R & T \\
    0 & 1
    \end{bmatrix}
    }_{\text{chaîne de projection}}
    \begin{bmatrix}
    X_w \\ Y_w \\ Z_w \\ 1
    \end{bmatrix}
  \]
  
  \pause
  \[
    \begin{bmatrix}
    u \\ v \\ w
    \end{bmatrix}
    =
    P
    \begin{bmatrix}
    X_w \\ Y_w \\ Z_w \\ 1
    \end{bmatrix}
    \quad \text{avec} \quad
    P \in \mathcal{M}_{3 \times 4}(\mathbb{R})
  \]
\end{frame}



%-----------------------------------------------
\begin{frame}
\frametitle{Les différents repères}

\[
\lambda_i 
\begin{pmatrix}
u^{(i)} \\
v^{(i)} \\
1
\end{pmatrix}
=
\begin{pmatrix}
p_{11} & p_{12} & p_{13} & p_{14} \\
p_{21} & p_{22} & p_{23} & p_{24} \\
p_{31} & p_{32} & p_{33} & p_{34}
\end{pmatrix}
\begin{pmatrix}
x_C^{(i)} \\
y_C^{(i)} \\
z_C^{(i)} \\
1
\end{pmatrix}
\]
\end{frame}

\begin{frame}
\frametitle{Système d’optimisation à contrainte unitaire}
\label{optimisation-appendix}
On souhaite résoudre le système en évitant la solution triviale \( P = 0 \).  
\pause

Sachant que la matrice \( P \) ne peut être déterminée qu’à un facteur près, on peut imposer :
\[
\|P\|^2 = 1
\]
et reformuler le système comme un problème d’optimisation :
\pause
\[
\min_{\|p\|^2 = 1} \|Ap\|^2 = \min_{\|p\|^2 = 1} p^T A^T A p
\]
\pause
On introduit les fonctions :
\begin{itemize}
  \item \( f(p) = p^T A^T A p \)
  \item \( g(p) = p^T p - 1 \)
\end{itemize}
\pause
D’après le théorème d’optimisation sous contrainte (Lagrange), au point optimal \( P^* \), il existe \( \lambda \in \mathbb{R} \) tel que :
\[
\nabla f(P^*) = \lambda \nabla g(P^*)
\]
\end{frame}



\begin{frame}{Lien avec les valeurs propres}

Posons \( M = A^T A \).  
Alors :
\[
f(p) = \sum_{i=1}^n \sum_{j=1}^n p_i M_{ij} p_j
\]
\pause

Comme \( M \) est symétrique :
\[
\frac{\partial f}{\partial p} = 2Mp
\quad \text{et} \quad
\frac{\partial g}{\partial p} = 2p
\]
\pause

On a donc :
\[
\frac{\partial f}{\partial p} = \lambda \frac{\partial g}{\partial p}
\quad \Rightarrow \quad
\boxed{A^T A p = \lambda p}
\]
\pause

C’est une équation aux valeurs propres :
\begin{itemize}
  \item \( p \) est un vecteur propre de \( A^T A \)
  \item \( \lambda \) est la valeur propre associée
\end{itemize}
\end{frame}

\begin{frame}{Triangulation : formulation du système}
\vspace*{-0.5em}
\begin{itemize}
  \item<1-> \( P_1 \) et \( P_2 \) déterminées
  \item<2-> On cherche les coordonnées \( X = (x_C, y_C, z_C, 1)^T \)
  \item<3-> Pour chaque paire \( (x_1, x_2) \) de projections
  \item<4-> On élimine \( \lambda_1, \lambda_2 \) et on écrit un système homogène
  \item<5-> Système sous la forme \( A X = 0 \)
\end{itemize}

\vspace{1em}

\pause
\pause
\pause
\pause
\begin{center}
\scriptsize
\[
A =
\left(
\begin{array}{cccc}
p_{31}^{1} u_1 - p_{11}^{1} & p_{32}^{1} u_1 - p_{12}^{1} & p_{33}^{1} u_1 - p_{13}^{1} & p_{34}^{1} u_1 - p_{14}^{1} \\
p_{31}^{1} v_1 - p_{21}^{1} & p_{32}^{1} v_1 - p_{22}^{1} & p_{33}^{1} v_1 - p_{23}^{1} & p_{34}^{1} v_1 - p_{24}^{1} \\
p_{31}^{2} u_2 - p_{11}^{2} & p_{32}^{2} u_2 - p_{12}^{2} & p_{33}^{2} u_2 - p_{13}^{2} & p_{34}^{2} u_2 - p_{14}^{2} \\
p_{31}^{2} v_2 - p_{21}^{2} & p_{32}^{2} v_2 - p_{22}^{2} & p_{33}^{2} v_2 - p_{23}^{2} & p_{34}^{2} v_2 - p_{24}^{2}
\end{array}
\right)
\]
\vspace*{1cm}
\end{center}
\end{frame}



\begin{frame}
    \scriptsize
    \begin{algorithm}[H]
\DontPrintSemicolon
\Input{$A \in \mathbb{R}^{m \times n}$}
\Output{$Q \in \mathbb{R}^{m \times n}$, $R \in \mathbb{R}^{n \times n}$ tels que $A = QR$}
\BlankLine

\For{$j \gets 1$ \KwTo $n$}{
  $v_j \gets A_{:,j}$ \tcc*[r]{Copie de la $j^{\text{\`e}me}$ colonne de $A$}
  \For{$i \gets 1$ \KwTo $j-1$}{
    $R_{i,j} \gets \langle Q_{:,i}, A_{:,j} \rangle$ \;
    $v_j \gets v_j - R_{i,j} Q_{:,i}$ \;
  }
  $R_{j,j} \gets \|v_j\|$ \;
  \Si{$R_{j,j} > \varepsilon$}{
    $Q_{:,j} \gets \frac{v_j}{R_{j,j}}$
  }
  \Sinon{
    $Q_{:,j} \gets 0$
  }
}
\Return{$Q, R$}
\caption{Décomposition QR via Gram-Schmidt}
\end{algorithm}
\end{frame}

%------------------------

\begin{frame}
  \label{SVD-appendix}
\scriptsize
\begin{algorithm}[H]
\DontPrintSemicolon
\Input{$B \in \mathbb{R}^{n \times n}$ symétrique}
\Output{$\Sigma^2$, $V$ tels que $B = V \Sigma^2 V^T$}

\BlankLine
$Q_{\text{acc}} \gets I_n$ \tcc*[r]{Accumule les produits de $Q$}
\BlankLine
%$\varepsilon \gets 10^{-12}$ \;
$\delta \gets 1$, $k_{\text{max}} \gets 1000$, $k \gets 0$\;

\Tq{$\delta > 10^{-9}$ et $k < k_{\text{max}}$}{
  $Q, R \gets$ décomposition QR de $B$\;
  $B_{\text{nouveau}} \gets R \cdot Q$\;
  $Q_{\text{acc}} \gets Q_{\text{acc}} \cdot Q$\;

  $\delta \gets \sum_i |\text{diag}(B_{\text{nouveau}})_i - \text{diag}(B)_i|$\;

  $A \gets B_{\text{nouveau}}$\;
  $k \gets k + 1$\;
}

\BlankLine
\For{$i=1$ à $n$}{
  \Si{$1[i, i] > \varepsilon$}{
    $\Sigma^2[i, i] \gets V[i, i]$
  }
  \Sinon{
    $\Sigma^2[i, i] \gets 0$
  }
}

\Return{$\Sigma^2$,$Q_{\text{acc}}$ }

\caption{algorithme QR}
\end{algorithm}
\end{frame}

\begin{frame}
\scriptsize  
\begin{algorithm}[H]
\DontPrintSemicolon
\Input{$A \in \mathbb{R}^{m \times n}$}
\Output{$U, \Sigma, V$ tels que $A \approx U \Sigma V^T$}

\BlankLine
$A^T \gets$ transposée de $A$\;
$A^T A \gets A^T \cdot A$ \tcc*[r]{Symétrique et définie positive}

\BlankLine
\texttt{algorithme\_QR}($A^T A$, $\Sigma^2$, $V$) \tcc*[r]{$\Sigma^2$ diagonale, $V$ orthogonale}

\BlankLine
\For{$i \gets 1$ \KwTo $n$}{
    $\sigma^2 \gets \Sigma^2[i, i]$\;
    \Si{$\sigma^2 < 10^{-12}$}{
        \textbf{continuer} \tcc*[r]{Ignorer valeur singulière nulle}
    }

    $\sigma \gets \sqrt{\sigma^2}$\;
    $\Sigma[i, i] \gets \sigma$ \tcc*[r]{Met à jour la vraie valeur singulière}

    $v_i \gets$ $i^{\text{e}}$ colonne de $V$\;
    $u_i \gets A \cdot v_i$ \tcc*[r]{$u_i$ non normalisé}
    $u_i \gets u_i / \sigma$\;
    normaliser $u_i$\;
    insérer $u_i$ comme $i^{\text{e}}$ colonne de $U$\;
}

\caption{SVD via algorithme QR sur \( A^T A \)}
\end{algorithm}
\end{frame}

