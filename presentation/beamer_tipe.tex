\documentclass[compress]{beamer}

% Inclusion des packages
%================== ENCODAGE & LANGUE ==================
\usepackage[utf8]{inputenc}
\usepackage[T1]{fontenc}
%\usepackage[french]{babel} % Optionnel

%================== MATHS & SYMBOLES ===================
\usepackage{amsmath, amssymb, amsfonts}
\usepackage{yhmath, mathdots, cancel}
\usepackage{siunitx}
\usepackage{gensymb}
\usepackage{textcomp}
\usepackage{pifont}
\usepackage{xspace}


%================== TABLEAUX ============================
\usepackage{array, tabularx, multirow, booktabs}

%================== COULEURS & GRAPHISMES ===============
\usepackage{color}
\usepackage{tikz}
\usetikzlibrary{
  shapes.geometric,
  backgrounds,
  fadings,
  patterns,
  shadows.blur,
  shapes,
  positioning,
  decorations.pathreplacing
}
\usepackage{xcolor} 

%================== MISE EN PAGE ========================
\usepackage{changepage}
\usepackage{calc}
\usepackage{caption}
\usepackage{xspace}
\usepackage{ragged2e}
\usepackage{amsmath, amsfonts, mathtools, amsthm, amssymb}
\usepackage{adjustbox}
\usepackage{caption}

%================== ALGORITHMIQUE =======================
\usepackage[ruled,vlined]{algorithm2e}
\SetAlgorithmName{Algorithme}{Algo}{Liste des algorithmes}
\SetFuncArgSty{textup}
\SetArgSty{textup}
\SetKwFor{ForEach}{pour tout}{faire}{}
\SetKwFor{For}{pour}{faire}{finpour}
\SetKwIF{Si}{SinonSi}{Sinon}{si}{alors}{sinon si}{sinon}{}
\SetKwInput{Input}{Entrée}
\SetKwInput{Output}{Sortie}
\SetKwProg{myproc}{Procédure}{:}{}
\SetKw{Return}{retourner}
\SetKwComment{tcc}{(*}{*)}
\SetKwFor{Tq}{tant que}{faire}{}
\SetKwRepeat{Repeter}{répéter}{jusqu’à}

%================== AUTRES ==============================
\usepackage[clock]{ifsym}



% Définir les initiales pour affichage dans l'en-tête (à utiliser dans thème perso si besoin)
\newcommand{\initiales}{L.-H. Cuingnet}

% Commandes utilitaires
\newcommand{\lin}[1]{\texttt{#1}} % Évite minted si pas nécessaire
\newcommand{\flch}{\item[$\rightarrow$]}
\newcommand{\dc}{{\usebeamercolor[fg]{structure}$\hookrightarrow$}}
\newcommand{\ok}{\textcolor{green}{\checkmark}}
\newcommand{\point}{{\usebeamercolor[fg]{structure}$\bullet\enskip$}}
\newcommand{\Point}{\point}
\newcommand{\couleur}[1]{{\usebeamercolor[fg]{structure}#1}}
\newcommand{\important}[1]{\couleur{\textbf{#1}}}
\newcommand{\remarque}[1]{\textit{\textrm{#1}}}

% Paramètres Beamer et thème (à personnaliser dans ce fichier)
\usetheme{CambridgeUS}
\usecolortheme{seahorse}


%--------marges
\setbeamersize{text margin left= 0.7cm}
\setbeamersize{text margin right= 0.7cm}

%--------tête et pieds
\setbeamertemplate{navigation symbols}{}
\setbeamertemplate{footline}[frame number]
\setbeamertemplate{headline}{
  %la premiere ligne
  	\begin{beamercolorbox}[ht=0.42cm, vmode]{section in head/foot}
	\hspace{0.4cm} \insertshorttitle 
	\hspace*{0.1cm}- \initiales - {\insertshortdate}
	\vspace*{0.08cm} 
  	\end{beamercolorbox}
  %la deuxième ligne
	\begin{beamercolorbox}[ht=0.4cm, vmode]{subsection in head/foot}
	%titre de la section si elle est pas 0
		\ifnum\value{section}=0{} 
		\else{ \hspace{0.8cm} \thesection - \insertsectionhead }
		\fi
	%séparateur + titre de la sous-section si elle est pas 0
		\ifnum\value{subsection}=0{} 
		\else{ 
			\hspace*{0.1cm} \couleur{$\bullet$} \hspace*{0.1cm} 
			\thesection.\thesubsection \, \insertsubsectionhead
		}
		\fi
		\vspace*{0.12cm}
\end{beamercolorbox}
%\vspace*{-0.03cm} %pour pas qu'il y ait d'espace avec la ligne de frametitle
 }
%\setbeamertemplate{frametitleheigth}{4cm}
\setbeamertemplate{frametitle}{
	\vspace*{-0.04cm} 
	\begin{beamercolorbox}[ht=0.8cm,wd=\paperwidth, vmode]{frametitle}
		\hspace{0.3cm} \insertframetitle \vspace*{0.1cm}
	\end{beamercolorbox}
}

%commande pour ajuster l'alignement vertical des titres sans lettres descendantes
%\newcommand{\esp}{\\[0.1cm]} %--version qui marche sans le package minted
\newcommand{\esp}{\\[-0.5cm]} %--version qui marche avec le package minted


%--------couleurs
\setbeamercolor{structure}{fg=turquoiseFonce!70!black} 

\setbeamercolor{block title}{fg=turquoiseFonce!70!black,bg=vertdEau}
\setbeamercolor{block body}{bg=vertdEau!10!white}

\setbeamercolor{block title alerted}{bg=vertdEau!85!white,fg=turquoiseFonce!80!black}
\setbeamercolor{block body alerted}{bg=vertdEau!8!white}
%\setbeamercolor{alerted text}{fg=red}

\setbeamercolor{block title example}{bg=vertdEau,fg=turquoiseFonce}
\setbeamercolor{block body example}{bg=vertdEau!10!white}
\setbeamercolor{example text}{fg=blue!20!turquoise}

%-------- TOC
\setbeamertemplate{section in toc}[sections numbered]

%-----------------------------------------------
%Plan qui s'affiche au début de chaque section %|
\AtBeginSection[]{                             %|
\begin{frame}[plain]                           %|
\frametitle{Plan\\[0.1cm]}                     %|
\tableofcontents[                              %|
currentsection,                                %|
hideothersubsections,                          %|
subsubsectionstyle=hide]                       %|
\addtocounter{framenumber}{-1}                 %|
\end{frame}}                                   %|
%-----------------------------------------------




%-------- commande pour les ref sur les slides
\newcommand{\bandeauREF}[1]{
\noindent\makebox[\textwidth][l]{%
\hspace{-\dimexpr\oddsidemargin+1in}%
\colorbox{expli!20!white}{%
\parbox{\dimexpr\paperwidth-2\fboxsep}{
\footnotesize\textcolor{expli!80!black}{#1}
}}}}
% Commandes utilitaires
\tikzset{every picture/.style={line width=0.75pt}} %set default line width to 0.75pt        
\newcommand{\imageFrame}{
  \draw [line width=0.75] (261.98,12.03) -- (262.02,142.84) -- (150.52,236.47) -- (150.47,105.66) -- cycle ;
}
\newcommand{\lin}[1]{\texttt{#1}} % Évite minted si pas nécessaire
\newcommand{\flch}{\item[$\rightarrow$]}
\newcommand{\dc}{{\usebeamercolor[fg]{structure}$\hookrightarrow$}}
\newcommand{\ok}{\textcolor{green}{\checkmark}}
\newcommand{\point}{{\usebeamercolor[fg]{structure}$\bullet\enskip$}}
\newcommand{\Point}{\point}
\newcommand{\couleur}[1]{{\usebeamercolor[fg]{structure}#1}}
\newcommand{\important}[1]{\couleur{\textbf{#1}}}
\newcommand{\remarque}[1]{\textit{\textrm{#1}}}
\newcommand{\cmark}{\ding{51}\xspace} % check ✓
\newcommand{\xmark}{\ding{55}\xspace} % cross ✗

% Palette de couleurs personnalisée
\input{la_palette.tex}
\definecolor{hellseahorse}{RGB}{204, 204, 255}
\definecolor{seahorse}{RGB}{204,180, 255}
\definecolor{darkseahorse}{RGB}{83, 74, 196}

%================== DÉBUT DU DOCUMENT ===================
\begin{document}

\title[Reconstruction 3D]{Reconstruction d’objets convexes à partir de photographies}
\author{
  \large Présentation de \important{Lucie-Hélène Cuingnet}\\[0.2cm]
  \footnotesize Travail réalisé avec \couleur{Barnabé Baruchel}
}
\date[Mai 2025]{}

\begin{frame}[plain]
  \titlepage
  \addtocounter{framenumber}{-1}
\end{frame}

% Slides principales (actives selon les fichiers présents)
%%%%%%%%%%%%%%%%%%%%%%%%%%%%%%%%%%%%%%%%%%%%%%%%%
\section[Selection et Appariement]{Selection et Appariement}
%------------------------------------------------
\begin{frame}
\frametitle{Titre d'une slide avant la sous-section\esp}
	Ici on n'a pas encore de titre de sous-section dans le badeau du haut.
	\label{slides_hors_subsec}
\end{frame}


%++++++++++++++++++++++++++++++++++++++++++++++++
\subsection{Selection}
%++++++++++++++++++++++++++++++++++++++++++++++++
%------------------------------------------------
\begin{frame}
\frametitle{Algorithme type \lin{Moravec}}
\[
\mathrm{Var}_{(dx, dy)}(x, y) = \frac{1}{N} \sum_{i=-w}^{w} I(x + i \cdot dx,\ y + i \cdot dy)^2 - \left( \frac{1}{N} \sum_{i=-w}^{w} I(x + i \cdot dx,\ y + i \cdot dy) \right)^2
\]

où :
\begin{itemize}
  \item $I(x + i \cdot dx,\ y + i \cdot dy)$ est l’intensité du $i$eme pixel dans la direction $(dx, dy)$,
  \item $N$ est le nombre de pixels valides (dans l’image) dans la fenêtre centrée en $(x, y)$,
  \item $w$ est le demi-rayon de la fenêtre .
\end{itemize}
\end{frame}
%------------------------------------------------
\begin{frame}
\frametitle{Algorithme type \lin{Moravec}}

\vspace{0.5em}
Le score du pixel : \textbf{minimum des variances dans 4 directions} :
\[
\text{score}(x, y) = \min \left\{ \mathrm{Var}_{(0,1)},\ \mathrm{Var}_{(1,0)},\ \mathrm{Var}_{(1,1)},\ \mathrm{Var}_{(1,-1)} \right\}
\]

Un pixel est considéré comme un \textbf{point d’intérêt} si :
\[
\text{score}(x, y) > T
\]
avec $T$ un seuil fixé.
\end{frame}


\begin{frame}{Animation stable et centrée}
  \begin{center}
    \begin{tikzpicture}[remember picture, scale=1]
      \only<1>{

\tikzset{every picture/.style={line width=0.75pt}} %set default line width to 0.75pt        

%uncomment if require: \path (0,202); %set diagram left start at 0, and has height of 202

%Shape: Rectangle [id:dp8976716894219737] 
\draw  [color={rgb, 255:red, 0; green, 0; blue, 0 }  ,draw opacity=1 ][fill={rgb, 255:red, 200; green, 43; blue, 43 }  ,fill opacity=0.23 ] (279.6,109) -- (299.6,109) -- (299.6,129) -- (279.6,129) -- cycle ;
%Shape: Rectangle [id:dp9242842609380733] 
\draw  [color={rgb, 255:red, 0; green, 0; blue, 0 }  ,draw opacity=1 ][fill={rgb, 255:red, 200; green, 43; blue, 43 }  ,fill opacity=0.64 ] (299.6,129) -- (319.6,129) -- (319.6,149) -- (299.6,149) -- cycle ;
%Shape: Rectangle [id:dp20566257715084624] 
\draw  [color={rgb, 255:red, 0; green, 0; blue, 0 }  ,draw opacity=1 ][fill={rgb, 255:red, 200; green, 43; blue, 43 }  ,fill opacity=0.64 ] (279.6,129) -- (299.6,129) -- (299.6,149) -- (279.6,149) -- cycle ;
%Shape: Rectangle [id:dp30919519111032157] 
\draw  [color={rgb, 255:red, 0; green, 0; blue, 0 }  ,draw opacity=1 ][fill={rgb, 255:red, 200; green, 43; blue, 43 }  ,fill opacity=0.69 ] (299.6,149) -- (319.6,149) -- (319.6,169) -- (299.6,169) -- cycle ;
%Shape: Rectangle [id:dp3685950018453733] 
\draw  [color={rgb, 255:red, 0; green, 0; blue, 0 }  ,draw opacity=1 ][fill={rgb, 255:red, 200; green, 43; blue, 43 }  ,fill opacity=0.68 ] (319.6,129) -- (339.6,129) -- (339.6,149) -- (319.6,149) -- cycle ;
%Shape: Rectangle [id:dp8054725442861926] 
\draw  [color={rgb, 255:red, 0; green, 0; blue, 0 }  ,draw opacity=1 ][fill={rgb, 255:red, 200; green, 43; blue, 43 }  ,fill opacity=0.32 ] (299.6,89) -- (319.6,89) -- (319.6,109) -- (299.6,109) -- cycle ;
%Shape: Rectangle [id:dp3185720649820162] 
\draw  [color={rgb, 255:red, 0; green, 0; blue, 0 }  ,draw opacity=1 ][fill={rgb, 255:red, 200; green, 43; blue, 43 }  ,fill opacity=0.64 ] (299.6,109) -- (319.6,109) -- (319.6,129) -- (299.6,129) -- cycle ;
%Shape: Rectangle [id:dp07104960494410428] 
\draw  [color={rgb, 255:red, 0; green, 0; blue, 0 }  ,draw opacity=1 ][fill={rgb, 255:red, 200; green, 43; blue, 43 }  ,fill opacity=0.23 ] (239.6,69) -- (259.6,69) -- (259.6,89) -- (239.6,89) -- cycle ;
%Shape: Rectangle [id:dp5441547618011067] 
\draw  [color={rgb, 255:red, 0; green, 0; blue, 0 }  ,draw opacity=1 ][fill={rgb, 255:red, 200; green, 43; blue, 43 }  ,fill opacity=0.21 ] (259.6,89) -- (279.6,89) -- (279.6,109) -- (259.6,109) -- cycle ;
%Shape: Rectangle [id:dp18624620459313024] 
\draw  [color={rgb, 255:red, 0; green, 0; blue, 0 }  ,draw opacity=1 ][fill={rgb, 255:red, 200; green, 43; blue, 43 }  ,fill opacity=0.25 ] (239.6,89) -- (259.6,89) -- (259.6,109) -- (239.6,109) -- cycle ;
%Shape: Rectangle [id:dp5364348089877012] 
\draw  [color={rgb, 255:red, 0; green, 0; blue, 0 }  ,draw opacity=1 ][fill={rgb, 255:red, 200; green, 43; blue, 43 }  ,fill opacity=0.28 ] (239.6,109) -- (259.6,109) -- (259.6,129) -- (239.6,129) -- cycle ;
%Shape: Rectangle [id:dp9541604569667589] 
\draw  [color={rgb, 255:red, 0; green, 0; blue, 0 }  ,draw opacity=1 ][fill={rgb, 255:red, 200; green, 43; blue, 43 }  ,fill opacity=0.66 ] (259.6,129) -- (279.6,129) -- (279.6,149) -- (259.6,149) -- cycle ;
%Shape: Rectangle [id:dp11982540210791848] 
\draw  [color={rgb, 255:red, 0; green, 0; blue, 0 }  ,draw opacity=1 ][fill={rgb, 255:red, 200; green, 43; blue, 43 }  ,fill opacity=0.77 ] (279.6,149) -- (299.6,149) -- (299.6,169) -- (279.6,169) -- cycle ;
%Shape: Rectangle [id:dp12026638886160845] 
\draw  [color={rgb, 255:red, 0; green, 0; blue, 0 }  ,draw opacity=1 ][fill={rgb, 255:red, 200; green, 43; blue, 43 }  ,fill opacity=0.24 ] (259.6,109) -- (279.6,109) -- (279.6,129) -- (259.6,129) -- cycle ;
%Shape: Rectangle [id:dp16801008580529164] 
\draw  [color={rgb, 255:red, 0; green, 0; blue, 0 }  ,draw opacity=1 ][fill={rgb, 255:red, 200; green, 43; blue, 43 }  ,fill opacity=0.26 ] (259.6,69) -- (279.6,69) -- (279.6,89) -- (259.6,89) -- cycle ;
%Shape: Rectangle [id:dp4471922901641272] 
\draw  [color={rgb, 255:red, 0; green, 0; blue, 0 }  ,draw opacity=1 ][fill={rgb, 255:red, 200; green, 43; blue, 43 }  ,fill opacity=0.22 ] (279.6,89) -- (299.6,89) -- (299.6,109) -- (279.6,109) -- cycle ;
%Shape: Rectangle [id:dp5193535071326038] 
\draw  [color={rgb, 255:red, 0; green, 0; blue, 0 }  ,draw opacity=1 ][fill={rgb, 255:red, 200; green, 43; blue, 43 }  ,fill opacity=0.29 ] (279.6,69) -- (299.6,69) -- (299.6,89) -- (279.6,89) -- cycle ;
%Shape: Rectangle [id:dp24429375415865406] 
\draw  [color={rgb, 255:red, 0; green, 0; blue, 0 }  ,draw opacity=1 ][fill={rgb, 255:red, 200; green, 43; blue, 43 }  ,fill opacity=0.28 ] (299.6,69) -- (319.6,69) -- (319.6,89) -- (299.6,89) -- cycle ;
%Shape: Rectangle [id:dp7327922213381906] 
\draw  [color={rgb, 255:red, 0; green, 0; blue, 0 }  ,draw opacity=1 ][fill={rgb, 255:red, 200; green, 43; blue, 43 }  ,fill opacity=0.65 ] (319.6,109) -- (339.6,109) -- (339.6,129) -- (319.6,129) -- cycle ;
%Shape: Rectangle [id:dp45563998070420675] 
\draw  [color={rgb, 255:red, 0; green, 0; blue, 0 }  ,draw opacity=1 ][fill={rgb, 255:red, 200; green, 43; blue, 43 }  ,fill opacity=0.64 ] (239.6,129) -- (259.6,129) -- (259.6,149) -- (239.6,149) -- cycle ;
%Shape: Rectangle [id:dp4081630774808085] 
\draw  [color={rgb, 255:red, 0; green, 0; blue, 0 }  ,draw opacity=1 ][fill={rgb, 255:red, 200; green, 43; blue, 43 }  ,fill opacity=0.73 ] (239.6,149) -- (259.6,149) -- (259.6,169) -- (239.6,169) -- cycle ;
%Shape: Rectangle [id:dp2684884892783119] 
\draw  [color={rgb, 255:red, 0; green, 0; blue, 0 }  ,draw opacity=1 ][fill={rgb, 255:red, 200; green, 43; blue, 43 }  ,fill opacity=0.74 ] (259.6,149) -- (279.6,149) -- (279.6,169) -- (259.6,169) -- cycle ;
%Shape: Rectangle [id:dp3091667704273665] 
\draw  [color={rgb, 255:red, 0; green, 0; blue, 0 }  ,draw opacity=1 ][fill={rgb, 255:red, 200; green, 43; blue, 43 }  ,fill opacity=0.68 ] (319.6,149) -- (339.6,149) -- (339.6,169) -- (319.6,169) -- cycle ;
%Shape: Rectangle [id:dp7449273860445551] 
\draw  [color={rgb, 255:red, 0; green, 0; blue, 0 }  ,draw opacity=1 ][fill={rgb, 255:red, 200; green, 43; blue, 43 }  ,fill opacity=0.7 ] (319.6,69) -- (339.6,69) -- (339.6,89) -- (319.6,89) -- cycle ;
%Shape: Rectangle [id:dp028827611328890224] 
\draw  [color={rgb, 255:red, 0; green, 0; blue, 0 }  ,draw opacity=1 ][fill={rgb, 255:red, 200; green, 43; blue, 43 }  ,fill opacity=0.63 ] (319.6,89) -- (339.6,89) -- (339.6,109) -- (319.6,109) -- cycle ;
%Shape: Rectangle [id:dp15038609219163346] 
\draw  [color={rgb, 255:red, 30; green, 14; blue, 222 }  ,draw opacity=1 ][fill={rgb, 255:red, 192; green, 177; blue, 255 }  ,fill opacity=0.34 ] (279.6,109) -- (299.6,109) -- (299.6,129) -- (279.6,129) -- cycle ;
%Shape: Rectangle [id:dp9252244711607317] 
\draw  [color={rgb, 255:red, 255; green, 255; blue, 255 }  ,draw opacity=1 ] (209,54.5) -- (451,54.5) -- (451,190.5) -- (209,190.5) -- cycle ;

% Text Node
\draw (352.6,70) node [anchor=north west][inner sep=0.75pt]  [font=\tiny,color={rgb, 255:red, 72; green, 36; blue, 227 }  ,opacity=1 ] [align=left] {pixel considéré};
% Text Node
\draw (304.8,136) node [anchor=north west][inner sep=0.75pt]  [font=\tiny] [align=left] {64};
% Text Node
\draw (324.8,155.2) node [anchor=north west][inner sep=0.75pt]  [font=\tiny] [align=left] {69};
% Text Node
\draw (305.2,115.6) node [anchor=north west][inner sep=0.75pt]  [font=\tiny] [align=left] {65};
% Text Node
\draw (305.2,95.6) node [anchor=north west][inner sep=0.75pt]  [font=\tiny] [align=left] {32};
% Text Node
\draw (285.2,95.6) node [anchor=north west][inner sep=0.75pt]  [font=\tiny] [align=left] {22};
% Text Node
\draw (325.2,115.6) node [anchor=north west][inner sep=0.75pt]  [font=\tiny] [align=left] {65};
% Text Node
\draw (244.4,76) node [anchor=north west][inner sep=0.75pt]  [font=\tiny] [align=left] {23};
% Text Node
\draw (265.2,96) node [anchor=north west][inner sep=0.75pt]  [font=\tiny] [align=left] {21};
% Text Node
\draw (244.8,95.6) node [anchor=north west][inner sep=0.75pt]  [font=\tiny] [align=left] {25};
% Text Node
\draw (264.8,116.4) node [anchor=north west][inner sep=0.75pt]  [font=\tiny] [align=left] {24};
% Text Node
\draw (284.8,135.6) node [anchor=north west][inner sep=0.75pt]  [font=\tiny] [align=left] {64};
% Text Node
\draw (304.8,155.2) node [anchor=north west][inner sep=0.75pt]  [font=\tiny] [align=left] {69};
% Text Node
\draw (264.4,75.6) node [anchor=north west][inner sep=0.75pt]  [font=\tiny] [align=left] {26};
% Text Node
\draw (285.2,75.6) node [anchor=north west][inner sep=0.75pt]  [font=\tiny] [align=left] {29};
% Text Node
\draw (304.8,75.2) node [anchor=north west][inner sep=0.75pt]  [font=\tiny] [align=left] {28};
% Text Node
\draw (324.8,95.2) node [anchor=north west][inner sep=0.75pt]  [font=\tiny] [align=left] {63};
% Text Node
\draw (324.8,76.8) node [anchor=north west][inner sep=0.75pt]  [font=\tiny] [align=left] {70};
% Text Node
\draw (244.8,115.6) node [anchor=north west][inner sep=0.75pt]  [font=\tiny] [align=left] {28};
% Text Node
\draw (264.8,135.6) node [anchor=north west][inner sep=0.75pt]  [font=\tiny] [align=left] {66};
% Text Node
\draw (284.8,155.2) node [anchor=north west][inner sep=0.75pt]  [font=\tiny] [align=left] {77};
% Text Node
\draw (244,136) node [anchor=north west][inner sep=0.75pt]  [font=\tiny] [align=left] {63};
% Text Node
\draw (264,155.2) node [anchor=north west][inner sep=0.75pt]  [font=\tiny] [align=left] {74};
% Text Node
\draw (245.2,155.6) node [anchor=north west][inner sep=0.75pt]  [font=\tiny] [align=left] {73};
% Text Node
\draw (324.8,136.4) node [anchor=north west][inner sep=0.75pt]  [font=\tiny] [align=left] {68};
% Text Node
\draw (285.2,116.4) node [anchor=north west][inner sep=0.75pt]  [font=\tiny] [align=left] {23};
% Text Node
\draw (354.5,86.4) node [anchor=north west][inner sep=0.75pt]  [font=\tiny]  {$w=2$};

%Shape: Rectangle [id:dp7383053127861628] 
\draw  [color={rgb, 255:red, 0; green, 0; blue, 0 }  ,draw opacity=1 ] (209,54.5) -- (451,54.5) -- (451,190.5) -- (209,190.5) -- cycle ;
}
      \only<2>{
\tikzset{every picture/.style={line width=0.75pt}} %set default line width to 0.75pt        

%Shape: Rectangle [id:dp08033051369790078] 
\draw  [color={rgb, 255:red, 0; green, 0; blue, 0 }  ,draw opacity=1 ][fill={rgb, 255:red, 200; green, 43; blue, 43 }  ,fill opacity=0.23 ] (279.6,109) -- (299.6,109) -- (299.6,129) -- (279.6,129) -- cycle ;
%Shape: Rectangle [id:dp49683966747337727] 
\draw  [color={rgb, 255:red, 0; green, 0; blue, 0 }  ,draw opacity=1 ][fill={rgb, 255:red, 200; green, 43; blue, 43 }  ,fill opacity=0.64 ] (299.6,129) -- (319.6,129) -- (319.6,149) -- (299.6,149) -- cycle ;
%Shape: Rectangle [id:dp23104622107220785] 
\draw  [color={rgb, 255:red, 0; green, 0; blue, 0 }  ,draw opacity=1 ][fill={rgb, 255:red, 200; green, 43; blue, 43 }  ,fill opacity=0.64 ] (279.6,129) -- (299.6,129) -- (299.6,149) -- (279.6,149) -- cycle ;
%Shape: Rectangle [id:dp5617749383810637] 
\draw  [color={rgb, 255:red, 0; green, 0; blue, 0 }  ,draw opacity=1 ][fill={rgb, 255:red, 200; green, 43; blue, 43 }  ,fill opacity=0.69 ] (299.6,149) -- (319.6,149) -- (319.6,169) -- (299.6,169) -- cycle ;
%Shape: Rectangle [id:dp5334311904090966] 
\draw  [color={rgb, 255:red, 0; green, 0; blue, 0 }  ,draw opacity=1 ][fill={rgb, 255:red, 200; green, 43; blue, 43 }  ,fill opacity=0.68 ] (319.6,129) -- (339.6,129) -- (339.6,149) -- (319.6,149) -- cycle ;
%Shape: Rectangle [id:dp1721875732765945] 
\draw  [color={rgb, 255:red, 0; green, 0; blue, 0 }  ,draw opacity=1 ][fill={rgb, 255:red, 200; green, 43; blue, 43 }  ,fill opacity=0.32 ] (299.6,89) -- (319.6,89) -- (319.6,109) -- (299.6,109) -- cycle ;
%Shape: Rectangle [id:dp27080417117801914] 
\draw  [color={rgb, 255:red, 0; green, 0; blue, 0 }  ,draw opacity=1 ][fill={rgb, 255:red, 200; green, 43; blue, 43 }  ,fill opacity=0.64 ] (299.6,109) -- (319.6,109) -- (319.6,129) -- (299.6,129) -- cycle ;
%Shape: Rectangle [id:dp12733374768288575] 
\draw  [color={rgb, 255:red, 0; green, 0; blue, 0 }  ,draw opacity=1 ][fill={rgb, 255:red, 200; green, 43; blue, 43 }  ,fill opacity=0.23 ] (239.6,69) -- (259.6,69) -- (259.6,89) -- (239.6,89) -- cycle ;
%Shape: Rectangle [id:dp8140813955207011] 
\draw  [color={rgb, 255:red, 0; green, 0; blue, 0 }  ,draw opacity=1 ][fill={rgb, 255:red, 200; green, 43; blue, 43 }  ,fill opacity=0.21 ] (259.6,89) -- (279.6,89) -- (279.6,109) -- (259.6,109) -- cycle ;
%Shape: Rectangle [id:dp5441728742071007] 
\draw  [color={rgb, 255:red, 0; green, 0; blue, 0 }  ,draw opacity=1 ][fill={rgb, 255:red, 200; green, 43; blue, 43 }  ,fill opacity=0.25 ] (239.6,89) -- (259.6,89) -- (259.6,109) -- (239.6,109) -- cycle ;
%Shape: Rectangle [id:dp4352090259725204] 
\draw  [color={rgb, 255:red, 0; green, 0; blue, 0 }  ,draw opacity=1 ][fill={rgb, 255:red, 200; green, 43; blue, 43 }  ,fill opacity=0.28 ] (239.6,109) -- (259.6,109) -- (259.6,129) -- (239.6,129) -- cycle ;
%Shape: Rectangle [id:dp8369983389318182] 
\draw  [color={rgb, 255:red, 0; green, 0; blue, 0 }  ,draw opacity=1 ][fill={rgb, 255:red, 200; green, 43; blue, 43 }  ,fill opacity=0.66 ] (259.6,129) -- (279.6,129) -- (279.6,149) -- (259.6,149) -- cycle ;
%Shape: Rectangle [id:dp3622508883600454] 
\draw  [color={rgb, 255:red, 0; green, 0; blue, 0 }  ,draw opacity=1 ][fill={rgb, 255:red, 200; green, 43; blue, 43 }  ,fill opacity=0.77 ] (279.6,149) -- (299.6,149) -- (299.6,169) -- (279.6,169) -- cycle ;
%Shape: Rectangle [id:dp4909680033056917] 
\draw  [color={rgb, 255:red, 0; green, 0; blue, 0 }  ,draw opacity=1 ][fill={rgb, 255:red, 200; green, 43; blue, 43 }  ,fill opacity=0.24 ] (259.6,109) -- (279.6,109) -- (279.6,129) -- (259.6,129) -- cycle ;
%Shape: Rectangle [id:dp32148408980275944] 
\draw  [color={rgb, 255:red, 0; green, 0; blue, 0 }  ,draw opacity=1 ][fill={rgb, 255:red, 200; green, 43; blue, 43 }  ,fill opacity=0.26 ] (259.6,69) -- (279.6,69) -- (279.6,89) -- (259.6,89) -- cycle ;
%Shape: Rectangle [id:dp9565080577380919] 
\draw  [color={rgb, 255:red, 0; green, 0; blue, 0 }  ,draw opacity=1 ][fill={rgb, 255:red, 200; green, 43; blue, 43 }  ,fill opacity=0.22 ] (279.6,89) -- (299.6,89) -- (299.6,109) -- (279.6,109) -- cycle ;
%Shape: Rectangle [id:dp6802012811969649] 
\draw  [color={rgb, 255:red, 0; green, 0; blue, 0 }  ,draw opacity=1 ][fill={rgb, 255:red, 200; green, 43; blue, 43 }  ,fill opacity=0.29 ] (279.6,69) -- (299.6,69) -- (299.6,89) -- (279.6,89) -- cycle ;
%Shape: Rectangle [id:dp3097842072193001] 
\draw  [color={rgb, 255:red, 0; green, 0; blue, 0 }  ,draw opacity=1 ][fill={rgb, 255:red, 200; green, 43; blue, 43 }  ,fill opacity=0.28 ] (299.6,69) -- (319.6,69) -- (319.6,89) -- (299.6,89) -- cycle ;
%Shape: Rectangle [id:dp7093765311745719] 
\draw  [color={rgb, 255:red, 0; green, 0; blue, 0 }  ,draw opacity=1 ][fill={rgb, 255:red, 200; green, 43; blue, 43 }  ,fill opacity=0.65 ] (319.6,109) -- (339.6,109) -- (339.6,129) -- (319.6,129) -- cycle ;
%Shape: Rectangle [id:dp6727225745032497] 
\draw  [color={rgb, 255:red, 0; green, 0; blue, 0 }  ,draw opacity=1 ][fill={rgb, 255:red, 200; green, 43; blue, 43 }  ,fill opacity=0.64 ] (239.6,129) -- (259.6,129) -- (259.6,149) -- (239.6,149) -- cycle ;
%Shape: Rectangle [id:dp2838905896546031] 
\draw  [color={rgb, 255:red, 0; green, 0; blue, 0 }  ,draw opacity=1 ][fill={rgb, 255:red, 200; green, 43; blue, 43 }  ,fill opacity=0.73 ] (239.6,149) -- (259.6,149) -- (259.6,169) -- (239.6,169) -- cycle ;
%Shape: Rectangle [id:dp5249713236193003] 
\draw  [color={rgb, 255:red, 0; green, 0; blue, 0 }  ,draw opacity=1 ][fill={rgb, 255:red, 200; green, 43; blue, 43 }  ,fill opacity=0.74 ] (259.6,149) -- (279.6,149) -- (279.6,169) -- (259.6,169) -- cycle ;
%Shape: Rectangle [id:dp11634068894640182] 
\draw  [color={rgb, 255:red, 0; green, 0; blue, 0 }  ,draw opacity=1 ][fill={rgb, 255:red, 200; green, 43; blue, 43 }  ,fill opacity=0.68 ] (319.6,149) -- (339.6,149) -- (339.6,169) -- (319.6,169) -- cycle ;
%Shape: Rectangle [id:dp11798884947196353] 
\draw  [color={rgb, 255:red, 0; green, 0; blue, 0 }  ,draw opacity=1 ][fill={rgb, 255:red, 200; green, 43; blue, 43 }  ,fill opacity=0.7 ] (319.6,69) -- (339.6,69) -- (339.6,89) -- (319.6,89) -- cycle ;
%Shape: Rectangle [id:dp3086351088617566] 
\draw  [color={rgb, 255:red, 0; green, 0; blue, 0 }  ,draw opacity=1 ][fill={rgb, 255:red, 200; green, 43; blue, 43 }  ,fill opacity=0.63 ] (319.6,89) -- (339.6,89) -- (339.6,109) -- (319.6,109) -- cycle ;
%Shape: Rectangle [id:dp8644140098494498] 
\draw  [color={rgb, 255:red, 30; green, 14; blue, 222 }  ,draw opacity=1 ][fill={rgb, 255:red, 192; green, 177; blue, 255 }  ,fill opacity=0.34 ] (279.6,109) -- (299.6,109) -- (299.6,129) -- (279.6,129) -- cycle ;
%Shape: Rectangle [id:dp37575572566639026] 
\draw  [color={rgb, 255:red, 255; green, 255; blue, 255 }  ,draw opacity=1 ] (209,54.5) -- (451,54.5) -- (451,190.5) -- (209,190.5) -- cycle ;
%Shape: Rectangle [id:dp17946493354937076] 
\draw  [color={rgb, 255:red, 222; green, 195; blue, 14 }  ,draw opacity=1 ][fill={rgb, 255:red, 192; green, 177; blue, 255 }  ,fill opacity=0.34 ] (239.6,109) -- (259.6,109) -- (259.6,129) -- (239.6,129) -- cycle ;
%Shape: Rectangle [id:dp947603637741915] 
\draw  [color={rgb, 255:red, 222; green, 195; blue, 14 }  ,draw opacity=1 ][fill={rgb, 255:red, 192; green, 177; blue, 255 }  ,fill opacity=0.34 ] (259.6,335) -- (279.6,335) -- (279.6,355) -- (259.6,355) -- cycle ;

% Text Node
\draw (352.6,70) node [anchor=north west][inner sep=0.75pt]  [font=\tiny,color={rgb, 255:red, 72; green, 36; blue, 227 }  ,opacity=1 ] [align=left] {pixel considéré};
% Text Node
\draw (304.8,136) node [anchor=north west][inner sep=0.75pt]  [font=\tiny] [align=left] {64};
% Text Node
\draw (324.8,155.2) node [anchor=north west][inner sep=0.75pt]  [font=\tiny] [align=left] {69};
% Text Node
\draw (305.2,115.6) node [anchor=north west][inner sep=0.75pt]  [font=\tiny] [align=left] {65};
% Text Node
\draw (305.2,95.6) node [anchor=north west][inner sep=0.75pt]  [font=\tiny] [align=left] {32};
% Text Node
\draw (285.2,95.6) node [anchor=north west][inner sep=0.75pt]  [font=\tiny] [align=left] {22};
% Text Node
\draw (325.2,115.6) node [anchor=north west][inner sep=0.75pt]  [font=\tiny] [align=left] {65};
% Text Node
\draw (244.4,76) node [anchor=north west][inner sep=0.75pt]  [font=\tiny] [align=left] {23};
% Text Node
\draw (265.2,96) node [anchor=north west][inner sep=0.75pt]  [font=\tiny] [align=left] {21};
% Text Node
\draw (244.8,95.6) node [anchor=north west][inner sep=0.75pt]  [font=\tiny] [align=left] {25};
% Text Node
\draw (264.8,116.4) node [anchor=north west][inner sep=0.75pt]  [font=\tiny] [align=left] {24};
% Text Node
\draw (284.8,135.6) node [anchor=north west][inner sep=0.75pt]  [font=\tiny] [align=left] {64};
% Text Node
\draw (304.8,155.2) node [anchor=north west][inner sep=0.75pt]  [font=\tiny] [align=left] {69};
% Text Node
\draw (264.4,75.6) node [anchor=north west][inner sep=0.75pt]  [font=\tiny] [align=left] {26};
% Text Node
\draw (285.2,75.6) node [anchor=north west][inner sep=0.75pt]  [font=\tiny] [align=left] {29};
% Text Node
\draw (304.8,75.2) node [anchor=north west][inner sep=0.75pt]  [font=\tiny] [align=left] {28};
% Text Node
\draw (324.8,95.2) node [anchor=north west][inner sep=0.75pt]  [font=\tiny] [align=left] {63};
% Text Node
\draw (324.8,76.8) node [anchor=north west][inner sep=0.75pt]  [font=\tiny] [align=left] {70};
% Text Node
\draw (244.8,115.6) node [anchor=north west][inner sep=0.75pt]  [font=\tiny] [align=left] {28};
% Text Node
\draw (264.8,135.6) node [anchor=north west][inner sep=0.75pt]  [font=\tiny] [align=left] {66};
% Text Node
\draw (284.8,155.2) node [anchor=north west][inner sep=0.75pt]  [font=\tiny] [align=left] {77};
% Text Node
\draw (244,136) node [anchor=north west][inner sep=0.75pt]  [font=\tiny] [align=left] {63};
% Text Node
\draw (264,155.2) node [anchor=north west][inner sep=0.75pt]  [font=\tiny] [align=left] {74};
% Text Node
\draw (245.2,155.6) node [anchor=north west][inner sep=0.75pt]  [font=\tiny] [align=left] {73};
% Text Node
\draw (324.8,136.4) node [anchor=north west][inner sep=0.75pt]  [font=\tiny] [align=left] {68};
% Text Node
\draw (285.2,116.4) node [anchor=north west][inner sep=0.75pt]  [font=\tiny] [align=left] {23};
% Text Node
\draw (354.5,86.4) node [anchor=north west][inner sep=0.75pt]  [font=\tiny]  {$w=2$};
% Text Node
\draw (357.3,135.4) node [anchor=north west][inner sep=0.75pt]  [font=\tiny]  {$i=-2$};
% Text Node
\draw (357.1,108.8) node [anchor=north west][inner sep=0.75pt]  [font=\tiny]  {$dx=1,\ dy=0$};
% Text Node
\draw (355.33,125.27) node [anchor=north west][inner sep=0.75pt]  [font=\tiny,color={rgb, 255:red, 72; green, 36; blue, 227 }  ,opacity=1 ] [align=left] {\textcolor[rgb]{0.55,0.64,0.02}{pixel comparé}};
% Text Node
\draw (357,98.13) node [anchor=north west][inner sep=0.75pt]  [font=\tiny,color={rgb, 255:red, 72; green, 36; blue, 227 }  ,opacity=1 ] [align=left] {direction horizontal};
% Text Node
\draw (357.3,154.8) node [anchor=north west][inner sep=0.75pt]  [font=\tiny]  {$S=28$};
% Text Node
\draw (393.3,154.4) node [anchor=north west][inner sep=0.75pt]  [font=\tiny]  {$S^{2} =784$};
%Shape: Rectangle [id:dp7383053127861628] 
\draw  [color={rgb, 255:red, 0; green, 0; blue, 0 }  ,draw opacity=1 ] (209,54.5) -- (451,54.5) -- (451,190.5) -- (209,190.5) -- cycle ;

}
      \only<3>{
\tikzset{every picture/.style={line width=0.75pt}} %set default line width to 0.75pt        



%Shape: Rectangle [id:dp41623934555948805] 
\draw  [color={rgb, 255:red, 0; green, 0; blue, 0 }  ,draw opacity=1 ][fill={rgb, 255:red, 200; green, 43; blue, 43 }  ,fill opacity=0.23 ] (279.6,109) -- (299.6,109) -- (299.6,129) -- (279.6,129) -- cycle ;
%Shape: Rectangle [id:dp15742549259652527] 
\draw  [color={rgb, 255:red, 0; green, 0; blue, 0 }  ,draw opacity=1 ][fill={rgb, 255:red, 200; green, 43; blue, 43 }  ,fill opacity=0.64 ] (299.6,129) -- (319.6,129) -- (319.6,149) -- (299.6,149) -- cycle ;
%Shape: Rectangle [id:dp3863982180716097] 
\draw  [color={rgb, 255:red, 0; green, 0; blue, 0 }  ,draw opacity=1 ][fill={rgb, 255:red, 200; green, 43; blue, 43 }  ,fill opacity=0.64 ] (279.6,129) -- (299.6,129) -- (299.6,149) -- (279.6,149) -- cycle ;
%Shape: Rectangle [id:dp478374436473631] 
\draw  [color={rgb, 255:red, 0; green, 0; blue, 0 }  ,draw opacity=1 ][fill={rgb, 255:red, 200; green, 43; blue, 43 }  ,fill opacity=0.69 ] (299.6,149) -- (319.6,149) -- (319.6,169) -- (299.6,169) -- cycle ;
%Shape: Rectangle [id:dp8911880470737273] 
\draw  [color={rgb, 255:red, 0; green, 0; blue, 0 }  ,draw opacity=1 ][fill={rgb, 255:red, 200; green, 43; blue, 43 }  ,fill opacity=0.68 ] (319.6,129) -- (339.6,129) -- (339.6,149) -- (319.6,149) -- cycle ;
%Shape: Rectangle [id:dp04503138886073499] 
\draw  [color={rgb, 255:red, 0; green, 0; blue, 0 }  ,draw opacity=1 ][fill={rgb, 255:red, 200; green, 43; blue, 43 }  ,fill opacity=0.32 ] (299.6,89) -- (319.6,89) -- (319.6,109) -- (299.6,109) -- cycle ;
%Shape: Rectangle [id:dp9644609206003013] 
\draw  [color={rgb, 255:red, 0; green, 0; blue, 0 }  ,draw opacity=1 ][fill={rgb, 255:red, 200; green, 43; blue, 43 }  ,fill opacity=0.64 ] (299.6,109) -- (319.6,109) -- (319.6,129) -- (299.6,129) -- cycle ;
%Shape: Rectangle [id:dp39507963239061805] 
\draw  [color={rgb, 255:red, 0; green, 0; blue, 0 }  ,draw opacity=1 ][fill={rgb, 255:red, 200; green, 43; blue, 43 }  ,fill opacity=0.23 ] (239.6,69) -- (259.6,69) -- (259.6,89) -- (239.6,89) -- cycle ;
%Shape: Rectangle [id:dp09172875488400045] 
\draw  [color={rgb, 255:red, 0; green, 0; blue, 0 }  ,draw opacity=1 ][fill={rgb, 255:red, 200; green, 43; blue, 43 }  ,fill opacity=0.21 ] (259.6,89) -- (279.6,89) -- (279.6,109) -- (259.6,109) -- cycle ;
%Shape: Rectangle [id:dp3660178952403952] 
\draw  [color={rgb, 255:red, 0; green, 0; blue, 0 }  ,draw opacity=1 ][fill={rgb, 255:red, 200; green, 43; blue, 43 }  ,fill opacity=0.25 ] (239.6,89) -- (259.6,89) -- (259.6,109) -- (239.6,109) -- cycle ;
%Shape: Rectangle [id:dp9520992808429042] 
\draw  [color={rgb, 255:red, 0; green, 0; blue, 0 }  ,draw opacity=1 ][fill={rgb, 255:red, 200; green, 43; blue, 43 }  ,fill opacity=0.28 ] (239.6,109) -- (259.6,109) -- (259.6,129) -- (239.6,129) -- cycle ;
%Shape: Rectangle [id:dp9715922571371791] 
\draw  [color={rgb, 255:red, 0; green, 0; blue, 0 }  ,draw opacity=1 ][fill={rgb, 255:red, 200; green, 43; blue, 43 }  ,fill opacity=0.66 ] (259.6,129) -- (279.6,129) -- (279.6,149) -- (259.6,149) -- cycle ;
%Shape: Rectangle [id:dp3583133966943417] 
\draw  [color={rgb, 255:red, 0; green, 0; blue, 0 }  ,draw opacity=1 ][fill={rgb, 255:red, 200; green, 43; blue, 43 }  ,fill opacity=0.77 ] (279.6,149) -- (299.6,149) -- (299.6,169) -- (279.6,169) -- cycle ;
%Shape: Rectangle [id:dp022168816978197836] 
\draw  [color={rgb, 255:red, 0; green, 0; blue, 0 }  ,draw opacity=1 ][fill={rgb, 255:red, 200; green, 43; blue, 43 }  ,fill opacity=0.24 ] (259.6,109) -- (279.6,109) -- (279.6,129) -- (259.6,129) -- cycle ;
%Shape: Rectangle [id:dp03377422539076902] 
\draw  [color={rgb, 255:red, 0; green, 0; blue, 0 }  ,draw opacity=1 ][fill={rgb, 255:red, 200; green, 43; blue, 43 }  ,fill opacity=0.26 ] (259.6,69) -- (279.6,69) -- (279.6,89) -- (259.6,89) -- cycle ;
%Shape: Rectangle [id:dp5054454472454241] 
\draw  [color={rgb, 255:red, 0; green, 0; blue, 0 }  ,draw opacity=1 ][fill={rgb, 255:red, 200; green, 43; blue, 43 }  ,fill opacity=0.22 ] (279.6,89) -- (299.6,89) -- (299.6,109) -- (279.6,109) -- cycle ;
%Shape: Rectangle [id:dp7342890782294983] 
\draw  [color={rgb, 255:red, 0; green, 0; blue, 0 }  ,draw opacity=1 ][fill={rgb, 255:red, 200; green, 43; blue, 43 }  ,fill opacity=0.29 ] (279.6,69) -- (299.6,69) -- (299.6,89) -- (279.6,89) -- cycle ;
%Shape: Rectangle [id:dp005507899969640073] 
\draw  [color={rgb, 255:red, 0; green, 0; blue, 0 }  ,draw opacity=1 ][fill={rgb, 255:red, 200; green, 43; blue, 43 }  ,fill opacity=0.28 ] (299.6,69) -- (319.6,69) -- (319.6,89) -- (299.6,89) -- cycle ;
%Shape: Rectangle [id:dp017524738654428162] 
\draw  [color={rgb, 255:red, 0; green, 0; blue, 0 }  ,draw opacity=1 ][fill={rgb, 255:red, 200; green, 43; blue, 43 }  ,fill opacity=0.65 ] (319.6,109) -- (339.6,109) -- (339.6,129) -- (319.6,129) -- cycle ;
%Shape: Rectangle [id:dp37178653266438877] 
\draw  [color={rgb, 255:red, 0; green, 0; blue, 0 }  ,draw opacity=1 ][fill={rgb, 255:red, 200; green, 43; blue, 43 }  ,fill opacity=0.64 ] (239.6,129) -- (259.6,129) -- (259.6,149) -- (239.6,149) -- cycle ;
%Shape: Rectangle [id:dp43511000750558937] 
\draw  [color={rgb, 255:red, 0; green, 0; blue, 0 }  ,draw opacity=1 ][fill={rgb, 255:red, 200; green, 43; blue, 43 }  ,fill opacity=0.73 ] (239.6,149) -- (259.6,149) -- (259.6,169) -- (239.6,169) -- cycle ;
%Shape: Rectangle [id:dp7129303286636162] 
\draw  [color={rgb, 255:red, 0; green, 0; blue, 0 }  ,draw opacity=1 ][fill={rgb, 255:red, 200; green, 43; blue, 43 }  ,fill opacity=0.74 ] (259.6,149) -- (279.6,149) -- (279.6,169) -- (259.6,169) -- cycle ;
%Shape: Rectangle [id:dp5549125040121842] 
\draw  [color={rgb, 255:red, 0; green, 0; blue, 0 }  ,draw opacity=1 ][fill={rgb, 255:red, 200; green, 43; blue, 43 }  ,fill opacity=0.68 ] (319.6,149) -- (339.6,149) -- (339.6,169) -- (319.6,169) -- cycle ;
%Shape: Rectangle [id:dp6510207368905343] 
\draw  [color={rgb, 255:red, 0; green, 0; blue, 0 }  ,draw opacity=1 ][fill={rgb, 255:red, 200; green, 43; blue, 43 }  ,fill opacity=0.7 ] (319.6,69) -- (339.6,69) -- (339.6,89) -- (319.6,89) -- cycle ;
%Shape: Rectangle [id:dp5732403849087179] 
\draw  [color={rgb, 255:red, 0; green, 0; blue, 0 }  ,draw opacity=1 ][fill={rgb, 255:red, 200; green, 43; blue, 43 }  ,fill opacity=0.63 ] (319.6,89) -- (339.6,89) -- (339.6,109) -- (319.6,109) -- cycle ;
%Shape: Rectangle [id:dp18147294007765313] 
\draw  [color={rgb, 255:red, 30; green, 14; blue, 222 }  ,draw opacity=1 ][fill={rgb, 255:red, 192; green, 177; blue, 255 }  ,fill opacity=0.34 ] (279.6,109) -- (299.6,109) -- (299.6,129) -- (279.6,129) -- cycle ;
%Shape: Rectangle [id:dp7383053127861628] 
\draw  [color={rgb, 255:red, 0; green, 0; blue, 0 }  ,draw opacity=1 ] (209,54.5) -- (451,54.5) -- (451,190.5) -- (209,190.5) -- cycle ;
%Shape: Rectangle [id:dp28157309764636895] 
\draw  [color={rgb, 255:red, 222; green, 195; blue, 14 }  ,draw opacity=1 ][fill={rgb, 255:red, 192; green, 177; blue, 255 }  ,fill opacity=0.34 ] (259.6,109) -- (279.6,109) -- (279.6,129) -- (259.6,129) -- cycle ;

% Text Node
\draw (352.6,70) node [anchor=north west][inner sep=0.75pt]  [font=\tiny,color={rgb, 255:red, 72; green, 36; blue, 227 }  ,opacity=1 ] [align=left] {pixel considéré};
% Text Node
\draw (304.8,136) node [anchor=north west][inner sep=0.75pt]  [font=\tiny] [align=left] {64};
% Text Node
\draw (324.8,155.2) node [anchor=north west][inner sep=0.75pt]  [font=\tiny] [align=left] {69};
% Text Node
\draw (305.2,115.6) node [anchor=north west][inner sep=0.75pt]  [font=\tiny] [align=left] {65};
% Text Node
\draw (305.2,95.6) node [anchor=north west][inner sep=0.75pt]  [font=\tiny] [align=left] {32};
% Text Node
\draw (285.2,95.6) node [anchor=north west][inner sep=0.75pt]  [font=\tiny] [align=left] {22};
% Text Node
\draw (325.2,115.6) node [anchor=north west][inner sep=0.75pt]  [font=\tiny] [align=left] {65};
% Text Node
\draw (244.4,76) node [anchor=north west][inner sep=0.75pt]  [font=\tiny] [align=left] {23};
% Text Node
\draw (265.2,96) node [anchor=north west][inner sep=0.75pt]  [font=\tiny] [align=left] {21};
% Text Node
\draw (244.8,95.6) node [anchor=north west][inner sep=0.75pt]  [font=\tiny] [align=left] {25};
% Text Node
\draw (264.8,116.4) node [anchor=north west][inner sep=0.75pt]  [font=\tiny] [align=left] {24};
% Text Node
\draw (284.8,135.6) node [anchor=north west][inner sep=0.75pt]  [font=\tiny] [align=left] {64};
% Text Node
\draw (304.8,155.2) node [anchor=north west][inner sep=0.75pt]  [font=\tiny] [align=left] {69};
% Text Node
\draw (264.4,75.6) node [anchor=north west][inner sep=0.75pt]  [font=\tiny] [align=left] {26};
% Text Node
\draw (285.2,75.6) node [anchor=north west][inner sep=0.75pt]  [font=\tiny] [align=left] {29};
% Text Node
\draw (304.8,75.2) node [anchor=north west][inner sep=0.75pt]  [font=\tiny] [align=left] {28};
% Text Node
\draw (324.8,95.2) node [anchor=north west][inner sep=0.75pt]  [font=\tiny] [align=left] {63};
% Text Node
\draw (324.8,76.8) node [anchor=north west][inner sep=0.75pt]  [font=\tiny] [align=left] {70};
% Text Node
\draw (244.8,115.6) node [anchor=north west][inner sep=0.75pt]  [font=\tiny] [align=left] {28};
% Text Node
\draw (264.8,135.6) node [anchor=north west][inner sep=0.75pt]  [font=\tiny] [align=left] {66};
% Text Node
\draw (284.8,155.2) node [anchor=north west][inner sep=0.75pt]  [font=\tiny] [align=left] {77};
% Text Node
\draw (244,136) node [anchor=north west][inner sep=0.75pt]  [font=\tiny] [align=left] {63};
% Text Node
\draw (264,155.2) node [anchor=north west][inner sep=0.75pt]  [font=\tiny] [align=left] {74};
% Text Node
\draw (245.2,155.6) node [anchor=north west][inner sep=0.75pt]  [font=\tiny] [align=left] {73};
% Text Node
\draw (324.8,136.4) node [anchor=north west][inner sep=0.75pt]  [font=\tiny] [align=left] {68};
% Text Node
\draw (285.2,116.4) node [anchor=north west][inner sep=0.75pt]  [font=\tiny] [align=left] {23};
% Text Node
\draw (354.5,86.4) node [anchor=north west][inner sep=0.75pt]  [font=\tiny]  {$w=2$};
% Text Node
\draw (357.3,135.4) node [anchor=north west][inner sep=0.75pt]  [font=\tiny]  {$i=-1$};
% Text Node
\draw (357.1,108.8) node [anchor=north west][inner sep=0.75pt]  [font=\tiny]  {$dx=1,\ dy=0$};
% Text Node
\draw (355.33,125.27) node [anchor=north west][inner sep=0.75pt]  [font=\tiny,color={rgb, 255:red, 72; green, 36; blue, 227 }  ,opacity=1 ] [align=left] {\textcolor[rgb]{0.55,0.64,0.02}{pixel comparé}};
% Text Node
\draw (357,98.13) node [anchor=north west][inner sep=0.75pt]  [font=\tiny,color={rgb, 255:red, 72; green, 36; blue, 227 }  ,opacity=1 ] [align=left] {direction horizontal};
% Text Node
\draw (357.3,154.8) node [anchor=north west][inner sep=0.75pt]  [font=\tiny]  {$S=52$};
% Text Node
\draw (393.3,154.4) node [anchor=north west][inner sep=0.75pt]  [font=\tiny]  {$S^{2} =1248$};

}
      \only<4>{\input{donnees/s1/tikz1_4.tex}}
      \only<5>{
\tikzset{every picture/.style={line width=0.75pt}} %set default line width to 0.75pt        


%Shape: Rectangle [id:dp5375598584667333] 
\draw  [color={rgb, 255:red, 0; green, 0; blue, 0 }  ,draw opacity=1 ][fill={rgb, 255:red, 200; green, 43; blue, 43 }  ,fill opacity=0.23 ] (279.6,109) -- (299.6,109) -- (299.6,129) -- (279.6,129) -- cycle ;
%Shape: Rectangle [id:dp6362028799834906] 
\draw  [color={rgb, 255:red, 0; green, 0; blue, 0 }  ,draw opacity=1 ][fill={rgb, 255:red, 200; green, 43; blue, 43 }  ,fill opacity=0.64 ] (299.6,129) -- (319.6,129) -- (319.6,149) -- (299.6,149) -- cycle ;
%Shape: Rectangle [id:dp17138982287354876] 
\draw  [color={rgb, 255:red, 0; green, 0; blue, 0 }  ,draw opacity=1 ][fill={rgb, 255:red, 200; green, 43; blue, 43 }  ,fill opacity=0.64 ] (279.6,129) -- (299.6,129) -- (299.6,149) -- (279.6,149) -- cycle ;
%Shape: Rectangle [id:dp11168678440472213] 
\draw  [color={rgb, 255:red, 0; green, 0; blue, 0 }  ,draw opacity=1 ][fill={rgb, 255:red, 200; green, 43; blue, 43 }  ,fill opacity=0.69 ] (299.6,149) -- (319.6,149) -- (319.6,169) -- (299.6,169) -- cycle ;
%Shape: Rectangle [id:dp10868115608692452] 
\draw  [color={rgb, 255:red, 0; green, 0; blue, 0 }  ,draw opacity=1 ][fill={rgb, 255:red, 200; green, 43; blue, 43 }  ,fill opacity=0.68 ] (319.6,129) -- (339.6,129) -- (339.6,149) -- (319.6,149) -- cycle ;
%Shape: Rectangle [id:dp768485299456727] 
\draw  [color={rgb, 255:red, 0; green, 0; blue, 0 }  ,draw opacity=1 ][fill={rgb, 255:red, 200; green, 43; blue, 43 }  ,fill opacity=0.32 ] (299.6,89) -- (319.6,89) -- (319.6,109) -- (299.6,109) -- cycle ;
%Shape: Rectangle [id:dp04068165547025682] 
\draw  [color={rgb, 255:red, 0; green, 0; blue, 0 }  ,draw opacity=1 ][fill={rgb, 255:red, 200; green, 43; blue, 43 }  ,fill opacity=0.64 ] (299.6,109) -- (319.6,109) -- (319.6,129) -- (299.6,129) -- cycle ;
%Shape: Rectangle [id:dp4604312220958826] 
\draw  [color={rgb, 255:red, 0; green, 0; blue, 0 }  ,draw opacity=1 ][fill={rgb, 255:red, 200; green, 43; blue, 43 }  ,fill opacity=0.23 ] (239.6,69) -- (259.6,69) -- (259.6,89) -- (239.6,89) -- cycle ;
%Shape: Rectangle [id:dp8858359898164444] 
\draw  [color={rgb, 255:red, 0; green, 0; blue, 0 }  ,draw opacity=1 ][fill={rgb, 255:red, 200; green, 43; blue, 43 }  ,fill opacity=0.21 ] (259.6,89) -- (279.6,89) -- (279.6,109) -- (259.6,109) -- cycle ;
%Shape: Rectangle [id:dp13618530730853728] 
\draw  [color={rgb, 255:red, 0; green, 0; blue, 0 }  ,draw opacity=1 ][fill={rgb, 255:red, 200; green, 43; blue, 43 }  ,fill opacity=0.25 ] (239.6,89) -- (259.6,89) -- (259.6,109) -- (239.6,109) -- cycle ;
%Shape: Rectangle [id:dp1678578319512094] 
\draw  [color={rgb, 255:red, 0; green, 0; blue, 0 }  ,draw opacity=1 ][fill={rgb, 255:red, 200; green, 43; blue, 43 }  ,fill opacity=0.28 ] (239.6,109) -- (259.6,109) -- (259.6,129) -- (239.6,129) -- cycle ;
%Shape: Rectangle [id:dp061963626898036694] 
\draw  [color={rgb, 255:red, 0; green, 0; blue, 0 }  ,draw opacity=1 ][fill={rgb, 255:red, 200; green, 43; blue, 43 }  ,fill opacity=0.66 ] (259.6,129) -- (279.6,129) -- (279.6,149) -- (259.6,149) -- cycle ;
%Shape: Rectangle [id:dp45445804554823566] 
\draw  [color={rgb, 255:red, 0; green, 0; blue, 0 }  ,draw opacity=1 ][fill={rgb, 255:red, 200; green, 43; blue, 43 }  ,fill opacity=0.77 ] (279.6,149) -- (299.6,149) -- (299.6,169) -- (279.6,169) -- cycle ;
%Shape: Rectangle [id:dp8471899987892493] 
\draw  [color={rgb, 255:red, 0; green, 0; blue, 0 }  ,draw opacity=1 ][fill={rgb, 255:red, 200; green, 43; blue, 43 }  ,fill opacity=0.24 ] (259.6,109) -- (279.6,109) -- (279.6,129) -- (259.6,129) -- cycle ;
%Shape: Rectangle [id:dp7454863327917263] 
\draw  [color={rgb, 255:red, 0; green, 0; blue, 0 }  ,draw opacity=1 ][fill={rgb, 255:red, 200; green, 43; blue, 43 }  ,fill opacity=0.26 ] (259.6,69) -- (279.6,69) -- (279.6,89) -- (259.6,89) -- cycle ;
%Shape: Rectangle [id:dp6027156226511302] 
\draw  [color={rgb, 255:red, 0; green, 0; blue, 0 }  ,draw opacity=1 ][fill={rgb, 255:red, 200; green, 43; blue, 43 }  ,fill opacity=0.22 ] (279.6,89) -- (299.6,89) -- (299.6,109) -- (279.6,109) -- cycle ;
%Shape: Rectangle [id:dp2772118388277819] 
\draw  [color={rgb, 255:red, 0; green, 0; blue, 0 }  ,draw opacity=1 ][fill={rgb, 255:red, 200; green, 43; blue, 43 }  ,fill opacity=0.29 ] (279.6,69) -- (299.6,69) -- (299.6,89) -- (279.6,89) -- cycle ;
%Shape: Rectangle [id:dp7719372959811857] 
\draw  [color={rgb, 255:red, 0; green, 0; blue, 0 }  ,draw opacity=1 ][fill={rgb, 255:red, 200; green, 43; blue, 43 }  ,fill opacity=0.28 ] (299.6,69) -- (319.6,69) -- (319.6,89) -- (299.6,89) -- cycle ;
%Shape: Rectangle [id:dp5124780871937471] 
\draw  [color={rgb, 255:red, 0; green, 0; blue, 0 }  ,draw opacity=1 ][fill={rgb, 255:red, 200; green, 43; blue, 43 }  ,fill opacity=0.65 ] (319.6,109) -- (339.6,109) -- (339.6,129) -- (319.6,129) -- cycle ;
%Shape: Rectangle [id:dp3617179862334128] 
\draw  [color={rgb, 255:red, 0; green, 0; blue, 0 }  ,draw opacity=1 ][fill={rgb, 255:red, 200; green, 43; blue, 43 }  ,fill opacity=0.64 ] (239.6,129) -- (259.6,129) -- (259.6,149) -- (239.6,149) -- cycle ;
%Shape: Rectangle [id:dp32793534866540996] 
\draw  [color={rgb, 255:red, 0; green, 0; blue, 0 }  ,draw opacity=1 ][fill={rgb, 255:red, 200; green, 43; blue, 43 }  ,fill opacity=0.73 ] (239.6,149) -- (259.6,149) -- (259.6,169) -- (239.6,169) -- cycle ;
%Shape: Rectangle [id:dp36554937028089873] 
\draw  [color={rgb, 255:red, 0; green, 0; blue, 0 }  ,draw opacity=1 ][fill={rgb, 255:red, 200; green, 43; blue, 43 }  ,fill opacity=0.74 ] (259.6,149) -- (279.6,149) -- (279.6,169) -- (259.6,169) -- cycle ;
%Shape: Rectangle [id:dp17667518825747697] 
\draw  [color={rgb, 255:red, 0; green, 0; blue, 0 }  ,draw opacity=1 ][fill={rgb, 255:red, 200; green, 43; blue, 43 }  ,fill opacity=0.68 ] (319.6,149) -- (339.6,149) -- (339.6,169) -- (319.6,169) -- cycle ;
%Shape: Rectangle [id:dp4256535156440978] 
\draw  [color={rgb, 255:red, 0; green, 0; blue, 0 }  ,draw opacity=1 ][fill={rgb, 255:red, 200; green, 43; blue, 43 }  ,fill opacity=0.7 ] (319.6,69) -- (339.6,69) -- (339.6,89) -- (319.6,89) -- cycle ;
%Shape: Rectangle [id:dp3346814044539125] 
\draw  [color={rgb, 255:red, 0; green, 0; blue, 0 }  ,draw opacity=1 ][fill={rgb, 255:red, 200; green, 43; blue, 43 }  ,fill opacity=0.63 ] (319.6,89) -- (339.6,89) -- (339.6,109) -- (319.6,109) -- cycle ;
%Shape: Rectangle [id:dp39668089199593637] 
\draw  [color={rgb, 255:red, 30; green, 14; blue, 222 }  ,draw opacity=1 ][fill={rgb, 255:red, 192; green, 177; blue, 255 }  ,fill opacity=0.34 ] (279.6,109) -- (299.6,109) -- (299.6,129) -- (279.6,129) -- cycle ;
%Shape: Rectangle [id:dp17425338332474016] 
\draw  [color={rgb, 255:red, 255; green, 255; blue, 255 }  ,draw opacity=1 ] (209,54.5) -- (451,54.5) -- (451,190.5) -- (209,190.5) -- cycle ;
%Shape: Rectangle [id:dp23241515965389126] 
\draw  [color={rgb, 255:red, 222; green, 195; blue, 14 }  ,draw opacity=1 ][fill={rgb, 255:red, 192; green, 177; blue, 255 }  ,fill opacity=0.34 ] (319.6,109) -- (339.6,109) -- (339.6,129) -- (319.6,129) -- cycle ;

% Text Node
\draw (352.6,70) node [anchor=north west][inner sep=0.75pt]  [font=\tiny,color={rgb, 255:red, 72; green, 36; blue, 227 }  ,opacity=1 ] [align=left] {pixel considéré};
% Text Node
\draw (304.8,136) node [anchor=north west][inner sep=0.75pt]  [font=\tiny] [align=left] {64};
% Text Node
\draw (324.8,155.2) node [anchor=north west][inner sep=0.75pt]  [font=\tiny] [align=left] {69};
% Text Node
\draw (305.2,115.6) node [anchor=north west][inner sep=0.75pt]  [font=\tiny] [align=left] {65};
% Text Node
\draw (305.2,95.6) node [anchor=north west][inner sep=0.75pt]  [font=\tiny] [align=left] {32};
% Text Node
\draw (285.2,95.6) node [anchor=north west][inner sep=0.75pt]  [font=\tiny] [align=left] {22};
% Text Node
\draw (325.2,115.6) node [anchor=north west][inner sep=0.75pt]  [font=\tiny] [align=left] {65};
% Text Node
\draw (244.4,76) node [anchor=north west][inner sep=0.75pt]  [font=\tiny] [align=left] {23};
% Text Node
\draw (265.2,96) node [anchor=north west][inner sep=0.75pt]  [font=\tiny] [align=left] {21};
% Text Node
\draw (244.8,95.6) node [anchor=north west][inner sep=0.75pt]  [font=\tiny] [align=left] {25};
% Text Node
\draw (264.8,116.4) node [anchor=north west][inner sep=0.75pt]  [font=\tiny] [align=left] {24};
% Text Node
\draw (284.8,135.6) node [anchor=north west][inner sep=0.75pt]  [font=\tiny] [align=left] {64};
% Text Node
\draw (304.8,155.2) node [anchor=north west][inner sep=0.75pt]  [font=\tiny] [align=left] {69};
% Text Node
\draw (264.4,75.6) node [anchor=north west][inner sep=0.75pt]  [font=\tiny] [align=left] {26};
% Text Node
\draw (285.2,75.6) node [anchor=north west][inner sep=0.75pt]  [font=\tiny] [align=left] {29};
% Text Node
\draw (304.8,75.2) node [anchor=north west][inner sep=0.75pt]  [font=\tiny] [align=left] {28};
% Text Node
\draw (324.8,95.2) node [anchor=north west][inner sep=0.75pt]  [font=\tiny] [align=left] {63};
% Text Node
\draw (324.8,76.8) node [anchor=north west][inner sep=0.75pt]  [font=\tiny] [align=left] {70};
% Text Node
\draw (244.8,115.6) node [anchor=north west][inner sep=0.75pt]  [font=\tiny] [align=left] {28};
% Text Node
\draw (264.8,135.6) node [anchor=north west][inner sep=0.75pt]  [font=\tiny] [align=left] {66};
% Text Node
\draw (284.8,155.2) node [anchor=north west][inner sep=0.75pt]  [font=\tiny] [align=left] {77};
% Text Node
\draw (244,136) node [anchor=north west][inner sep=0.75pt]  [font=\tiny] [align=left] {63};
% Text Node
\draw (264,155.2) node [anchor=north west][inner sep=0.75pt]  [font=\tiny] [align=left] {74};
% Text Node
\draw (245.2,155.6) node [anchor=north west][inner sep=0.75pt]  [font=\tiny] [align=left] {73};
% Text Node
\draw (324.8,136.4) node [anchor=north west][inner sep=0.75pt]  [font=\tiny] [align=left] {68};
% Text Node
\draw (285.2,116.4) node [anchor=north west][inner sep=0.75pt]  [font=\tiny] [align=left] {23};
% Text Node
\draw (354.5,86.4) node [anchor=north west][inner sep=0.75pt]  [font=\tiny]  {$w=2$};
% Text Node
\draw (354.8,114.33) node [anchor=north west][inner sep=0.75pt]  [font=\tiny,color={rgb, 255:red, 72; green, 36; blue, 227 }  ,opacity=1 ] [align=left] {Var(1,0)=386.8};
% Text Node
\draw (355.9,98.2) node [anchor=north west][inner sep=0.75pt]  [font=\tiny]  {$T=300$};
% Text Node
\draw (355.4,124.33) node [anchor=north west][inner sep=0.75pt]  [font=\tiny,color={rgb, 255:red, 72; green, 36; blue, 227 }  ,opacity=1 ] [align=left] {Var(0,1)=526.8};
% Text Node
\draw (354.6,133.73) node [anchor=north west][inner sep=0.75pt]  [font=\tiny,color={rgb, 255:red, 72; green, 36; blue, 227 }  ,opacity=1 ] [align=left] {Var(1,1)=439.7};
% Text Node
\draw (355,143.93) node [anchor=north west][inner sep=0.75pt]  [font=\tiny,color={rgb, 255:red, 72; green, 36; blue, 227 }  ,opacity=1 ] [align=left] {Var(1,-1)=471.2};
% Text Node
\draw (356.23,170.2) node [anchor=north west][inner sep=0.75pt]  [font=\tiny]  {$S >T\Longrightarrow $};
% Text Node
\draw (395.93,170) node [anchor=north west][inner sep=0.75pt]  [font=\tiny,color={rgb, 255:red, 0; green, 0; blue, 0 }  ,opacity=1 ] [align=left] {point d'intérêt};
% Text Node
\draw (356.57,157.53) node [anchor=north west][inner sep=0.75pt]  [font=\tiny]  {$S=386.8$};
%Shape: Rectangle [id:dp7383053127861628] 
\draw  [color={rgb, 255:red, 0; green, 0; blue, 0 }  ,draw opacity=1 ] (209,54.5) -- (451,54.5) -- (451,190.5) -- (209,190.5) -- cycle ;
}
    \end{tikzpicture}
  \end{center}
\end{frame}

%------------------------------------------------

\begin{frame}
	\small
\frametitle{Algorithme type \lin{Moravec}}
\begin{algorithm}[H]
    \caption{\textsf{Moravec (minimum des variances)}}
    \Input{Image d’intensité \texttt{image}}
    \Output{Liste des coins détectés}
    \BlankLine
    \ForEach{pixel $(x, y)$ dans l’image}{
        $scores \gets$ liste vide\;
        \ForEach{direction $(dx, dy)$ parmi : verticale, horizontale, diagonales}{
            Calculer la variance locale autour de $(x, y)$ dans la direction $(dx, dy)$\;
            Ajouter la variance à $scores$\;
        }
        $score \gets \min(scores)$\;
        \If{$score >$ \texttt{SEUIL}}{
            Marquer $(x, y)$ comme coin
        }
    }
    \Return{Liste des points marqués}
\end{algorithm}


\end{frame}


%++++++++++++++++++++++++++++++++++++++++++++++++
\subsection{Appariement}
%------------------------------------------------
\begin{frame}
\frametitle{Titre de la slide sans lettre descendant sous la baseline}
	Pour régler ce problème, utiliser la commande \lin{\esp} à la fin du titre, \textit{Cf.} slide suivante
\end{frame}



%%%%%%%%%%%%%%%%%%%%%%%%%%%%%%%%%%%%%%%%%%%%%%%%%
\section[Reconstruction]{Reconstruction}
%------------------------------------------------
%++++++++++++++++++++++++++++++++++++++++++++++++
\subsection{Modélisation théorique}
%++++++++++++++++++++++++++++++++++++++++++++++++

\begin{frame}
\frametitle{Les différents repères}

\begin{minipage}{0.48\textwidth}
    \centering
    \includegraphics[width=\linewidth]{capture/cube_tikz.pdf}
    \vspace{0.5em}
    
    {\footnotesize\textbf{Représentation du cube (vue 3D)}}
\end{minipage}
\hfill
\begin{minipage}{0.48\textwidth}
    \centering
    \includegraphics[width=\linewidth]{capture/cube_repere.png}
    \vspace{0.5em}

    {\footnotesize\textbf{Cube sur une image}}
\end{minipage}

\end{frame}


\begin{frame}
\frametitle{Les différents repères}

\begin{minipage}[c]{0.48\linewidth}
  \centering
  \begin{overlayarea}{0.9\linewidth}{4cm}
    \hspace*{-1cm}
    \begin{tikzpicture}[x=0.75pt,y=0.75pt,yscale=-1,xscale=1, scale=0.6]
    \only<1>{


\tikzset{every picture/.style={line width=0.75pt}} %set default line width to 0.75pt        

\begin{tikzpicture}[x=0.75pt,y=0.75pt,yscale=-1,xscale=1]
%uncomment if require: \path (0,300); %set diagram left start at 0, and has height of 300

%Shape: Rectangle [id:dp9723727103024109] 
\draw  [line width=0.75]  (273.11,2.68) -- (273.16,155.68) -- (150.53,258.66) -- (150.47,105.66) -- cycle ;
%Shape: Circle [id:dp9239162914529644] 
\draw   (170.87,132.32) .. controls (172.23,131.96) and (173.33,132.77) .. (173.33,134.13) .. controls (173.33,135.49) and (172.23,136.89) .. (170.87,137.26) .. controls (169.5,137.62) and (168.4,136.81) .. (168.4,135.45) .. controls (168.4,134.09) and (169.5,132.69) .. (170.87,132.32) -- cycle ;
%Straight Lines [id:da4507743094695521] 
\draw  [dash pattern={on 0.84pt off 2.51pt}]  (211.82,130.67) -- (45,134.7) ;
%Shape: Circle [id:dp9335082203579057] 
\draw   (499.62,131.92) .. controls (500.98,131.55) and (502.08,132.36) .. (502.08,133.72) .. controls (502.08,135.09) and (500.98,136.49) .. (499.62,136.85) .. controls (498.25,137.22) and (497.15,136.41) .. (497.15,135.05) .. controls (497.15,133.68) and (498.25,132.28) .. (499.62,131.92) -- cycle ;
%Shape: Circle [id:dp5162114026007054] 
\draw  [color={rgb, 255:red, 241; green, 53; blue, 53 }  ,draw opacity=1 ] (512.82,105.92) .. controls (514.18,105.55) and (515.28,106.36) .. (515.28,107.72) .. controls (515.28,109.09) and (514.18,110.49) .. (512.82,110.85) .. controls (511.45,111.22) and (510.35,110.41) .. (510.35,109.05) .. controls (510.35,107.68) and (511.45,106.28) .. (512.82,105.92) -- cycle ;
%Shape: Circle [id:dp4319786220460511] 
\draw  [fill={rgb, 255:red, 0; green, 0; blue, 0 }  ,fill opacity=1 ] (321.12,132.42) .. controls (322.48,132.42) and (323.58,133.52) .. (323.58,134.88) .. controls (323.58,136.25) and (322.48,137.35) .. (321.12,137.35) .. controls (319.75,137.35) and (318.65,136.25) .. (318.65,134.88) .. controls (318.65,133.52) and (319.75,132.42) .. (321.12,132.42) -- cycle ;
%Straight Lines [id:da5164589906920809] 
\draw    (321.12,134.88) -- (321.97,70.75) ;
\draw [shift={(322,68.75)}, rotate = 90.77] [color={rgb, 255:red, 0; green, 0; blue, 0 }  ][line width=0.75]    (10.93,-3.29) .. controls (6.95,-1.4) and (3.31,-0.3) .. (0,0) .. controls (3.31,0.3) and (6.95,1.4) .. (10.93,3.29)   ;
%Straight Lines [id:da7854412907443913] 
\draw    (321.12,134.88) -- (253.5,134.75) ;
\draw [shift={(251.5,134.75)}, rotate = 0.11] [color={rgb, 255:red, 0; green, 0; blue, 0 }  ][line width=0.75]    (10.93,-3.29) .. controls (6.95,-1.4) and (3.31,-0.3) .. (0,0) .. controls (3.31,0.3) and (6.95,1.4) .. (10.93,3.29)   ;
%Straight Lines [id:da3058610918194634] 
\draw    (321.12,134.88) -- (272.48,176.16) ;
\draw [shift={(270.95,177.45)}, rotate = 319.68] [color={rgb, 255:red, 0; green, 0; blue, 0 }  ][line width=0.75]    (10.93,-3.29) .. controls (6.95,-1.4) and (3.31,-0.3) .. (0,0) .. controls (3.31,0.3) and (6.95,1.4) .. (10.93,3.29)   ;
%Straight Lines [id:da5453100201962917] 
\draw  [dash pattern={on 4.5pt off 4.5pt}]  (149.63,172.62) -- (150.47,105.66) ;
\draw [shift={(149.61,174.62)}, rotate = 270.72] [color={rgb, 255:red, 0; green, 0; blue, 0 }  ][line width=0.75]    (10.93,-3.29) .. controls (6.95,-1.4) and (3.31,-0.3) .. (0,0) .. controls (3.31,0.3) and (6.95,1.4) .. (10.93,3.29)   ;
%Straight Lines [id:da4691798453437107] 
\draw  [dash pattern={on 4.5pt off 4.5pt}]  (199.11,64.39) -- (150.47,105.66) ;
\draw [shift={(200.64,63.1)}, rotate = 139.68] [color={rgb, 255:red, 0; green, 0; blue, 0 }  ][line width=0.75]    (10.93,-3.29) .. controls (6.95,-1.4) and (3.31,-0.3) .. (0,0) .. controls (3.31,0.3) and (6.95,1.4) .. (10.93,3.29)   ;
%Straight Lines [id:da9224741393814412] 
\draw [color={rgb, 255:red, 241; green, 45; blue, 45 }  ,draw opacity=1 ]   (512.82,108.38) -- (192.48,152.97) ;
\draw [shift={(190.5,153.25)}, rotate = 352.08] [color={rgb, 255:red, 241; green, 45; blue, 45 }  ,draw opacity=1 ][line width=0.75]    (10.93,-3.29) .. controls (6.95,-1.4) and (3.31,-0.3) .. (0,0) .. controls (3.31,0.3) and (6.95,1.4) .. (10.93,3.29)   ;
%Shape: Circle [id:dp7585208638434278] 
\draw  [color={rgb, 255:red, 241; green, 53; blue, 53 }  ,draw opacity=1 ] (190.5,150.78) .. controls (191.86,150.42) and (192.97,151.23) .. (192.97,152.59) .. controls (192.97,153.95) and (191.86,155.35) .. (190.5,155.72) .. controls (189.14,156.08) and (188.03,155.27) .. (188.03,153.91) .. controls (188.03,152.55) and (189.14,151.15) .. (190.5,150.78) -- cycle ;
%Shape: Cube [id:dp7714367112041981] 
\draw   (499.62,131.92) -- (520.32,111.22) -- (568.62,111.22) -- (568.62,160.52) -- (547.92,181.22) -- (499.62,181.22) -- cycle ; \draw   (568.62,111.22) -- (547.92,131.92) -- (499.62,131.92) ; \draw   (547.92,131.92) -- (547.92,181.22) ;

% Text Node
\draw (517,97.9) node [anchor=north west][inner sep=0.75pt]  [font=\footnotesize,color={rgb, 255:red, 167; green, 17; blue, 17 }  ,opacity=1 ]  {$P$};
% Text Node
\draw (324,59.9) node [anchor=north west][inner sep=0.75pt]  [font=\footnotesize]  {$\vec{j}$};
% Text Node
\draw (280.5,178.9) node [anchor=north west][inner sep=0.75pt]  [font=\footnotesize]  {$\vec{i}$};
% Text Node
\draw (324.43,139.84) node [anchor=north west][inner sep=0.75pt]  [font=\footnotesize]  {$O$};
% Text Node
\draw (252,112.9) node [anchor=north west][inner sep=0.75pt]  [font=\footnotesize]  {$\vec{k}$};
% Text Node
\draw (158.65,118.28) node [anchor=north west][inner sep=0.75pt]  [font=\footnotesize]  {$C$};
% Text Node
\draw (156,62.9) node [anchor=north west][inner sep=0.75pt]  [font=\footnotesize]  {$\vec{u}$};
% Text Node
\draw (125.5,114.4) node [anchor=north west][inner sep=0.75pt]  [font=\footnotesize]  {$\vec{v}$};
% Text Node
\draw (184.5,155.9) node [anchor=north west][inner sep=0.75pt]  [font=\footnotesize,color={rgb, 255:red, 159; green, 16; blue, 16 }  ,opacity=1 ]  {$P'$};


\end{tikzpicture}}
    \only<2>{
%Shape: Rectangle [id:dp4318964674614302] 
\draw  [line width=0.75]  (261.98,12.03) -- (262.02,142.84) -- (150.52,236.47) -- (150.47,105.66) -- cycle ;
%Shape: Circle [id:dp8619322586517022] 
\draw  [color={rgb, 255:red, 16; green, 18; blue, 125 }  ,draw opacity=1 ][fill={rgb, 255:red, 16; green, 18; blue, 125 }  ,fill opacity=1 ] (350.47,93.07) .. controls (351.83,92.71) and (352.93,93.52) .. (352.93,94.88) .. controls (352.93,96.24) and (351.83,97.64) .. (350.47,98.01) .. controls (349.1,98.37) and (348,97.56) .. (348,96.2) .. controls (348,94.84) and (349.1,93.44) .. (350.47,93.07) -- cycle ;
%Shape: Cube [id:dp5712797611314612] 
\draw   (325.13,88.8) -- (342.27,71.67) -- (382.25,71.67) -- (382.25,112.75) -- (365.12,129.88) -- (325.13,129.88) -- cycle ; \draw   (382.25,71.67) -- (365.12,88.8) -- (325.13,88.8) ; \draw   (365.12,88.8) -- (365.12,129.88) ;
%Shape: Rectangle [id:dp6697517945708218] 
\draw  [color={rgb, 255:red, 255; green, 255; blue, 255 }  ,draw opacity=1 ] (41.02,5.34) -- (469,5.34) -- (469,280.34) -- (41.02,280.34) -- cycle ;

% Text Node
\draw (358,50.9) node [anchor=north west][inner sep=0.75pt]  [font=\footnotesize,color={rgb, 255:red, 16; green, 18; blue, 125 }  ,opacity=1 ]  {$M$};}
    \only<3>{\input{donnees/s2_3.tex}}
    \only<4>{\input{donnees/s2_4.tex}}
    \only<5>{
\draw  [line width=0.75]  (261.98,12.03) -- (262.02,142.84) -- (150.52,236.47) -- (150.47,105.66) -- cycle ;
%Shape: Circle [id:dp0391503239902703] 
\draw  [color={rgb, 255:red, 16; green, 18; blue, 125 }  ,draw opacity=1 ][fill={rgb, 255:red, 16; green, 18; blue, 125 }  ,fill opacity=1 ] (350.47,93.07) .. controls (351.83,92.71) and (352.93,93.52) .. (352.93,94.88) .. controls (352.93,96.24) and (351.83,97.64) .. (350.47,98.01) .. controls (349.1,98.37) and (348,97.56) .. (348,96.2) .. controls (348,94.84) and (349.1,93.44) .. (350.47,93.07) -- cycle ;
%Shape: Circle [id:dp8654606077158083] 
\draw  [fill={rgb, 255:red, 0; green, 0; blue, 0 }  ,fill opacity=1 ] (365.12,130.42) .. controls (366.48,130.42) and (367.58,131.52) .. (367.58,132.88) .. controls (367.58,134.25) and (366.48,135.35) .. (365.12,135.35) .. controls (363.75,135.35) and (362.65,134.25) .. (362.65,132.88) .. controls (362.65,131.52) and (363.75,130.42) .. (365.12,130.42) -- cycle ;
%Straight Lines [id:da3675337375445349] 
\draw  [dash pattern={on 4.5pt off 4.5pt}]  (150.02,143.52) -- (150.47,105.66) ;
\draw [shift={(150,145.52)}, rotate = 270.68] [color={rgb, 255:red, 0; green, 0; blue, 0 }  ][line width=0.75]    (10.93,-3.29) .. controls (6.95,-1.4) and (3.31,-0.3) .. (0,0) .. controls (3.31,0.3) and (6.95,1.4) .. (10.93,3.29)   ;
%Straight Lines [id:da2135082456633841] 
\draw  [dash pattern={on 4.5pt off 4.5pt}]  (182.47,78.8) -- (150.47,105.66) ;
\draw [shift={(184,77.52)}, rotate = 139.99] [color={rgb, 255:red, 0; green, 0; blue, 0 }  ][line width=0.75]    (10.93,-3.29) .. controls (6.95,-1.4) and (3.31,-0.3) .. (0,0) .. controls (3.31,0.3) and (6.95,1.4) .. (10.93,3.29)   ;
%Shape: Circle [id:dp3919067214179274] 
\draw  [color={rgb, 255:red, 16; green, 18; blue, 125 }  ,draw opacity=1 ] (209.53,118.71) .. controls (210.9,118.35) and (212,119.15) .. (212,120.52) .. controls (212,121.88) and (210.9,123.28) .. (209.53,123.64) .. controls (208.17,124.01) and (207.07,123.2) .. (207.07,121.84) .. controls (207.07,120.48) and (208.17,119.08) .. (209.53,118.71) -- cycle ;
%Straight Lines [id:da13961882146045834] 
\draw [color={rgb, 255:red, 16; green, 18; blue, 125 }  ,draw opacity=1 ][fill={rgb, 255:red, 16; green, 18; blue, 125 }  ,fill opacity=1 ]   (350.47,95.54) -- (212,120.52) ;
%Straight Lines [id:da6009765896576962] 
\draw    (365.12,130.42) -- (331,129.27) ;
\draw [shift={(329,129.2)}, rotate = 1.93] [color={rgb, 255:red, 0; green, 0; blue, 0 }  ][line width=0.75]    (10.93,-3.29) .. controls (6.95,-1.4) and (3.31,-0.3) .. (0,0) .. controls (3.31,0.3) and (6.95,1.4) .. (10.93,3.29)   ;
%Straight Lines [id:da13460515919081129] 
\draw    (365.12,130.42) -- (380.86,114.19) ;
\draw [shift={(382.25,112.75)}, rotate = 134.12] [color={rgb, 255:red, 0; green, 0; blue, 0 }  ][line width=0.75]    (10.93,-3.29) .. controls (6.95,-1.4) and (3.31,-0.3) .. (0,0) .. controls (3.31,0.3) and (6.95,1.4) .. (10.93,3.29)   ;
%Straight Lines [id:da5115010168409273] 
\draw    (364.65,132.88) -- (364.04,100.2) ;
\draw [shift={(364,98.2)}, rotate = 88.93] [color={rgb, 255:red, 0; green, 0; blue, 0 }  ][line width=0.75]    (10.93,-3.29) .. controls (6.95,-1.4) and (3.31,-0.3) .. (0,0) .. controls (3.31,0.3) and (6.95,1.4) .. (10.93,3.29)   ;
%Shape: Cube [id:dp47417096557677685] 
\draw   (325.13,88.8) -- (342.27,71.67) -- (382.25,71.67) -- (382.25,112.75) -- (365.12,129.88) -- (325.13,129.88) -- cycle ; \draw   (382.25,71.67) -- (365.12,88.8) -- (325.13,88.8) ; \draw   (365.12,88.8) -- (365.12,129.88) ;
%Shape: Rectangle [id:dp12317704811349806] 
\draw  [color={rgb, 255:red, 255; green, 255; blue, 255 }  ,draw opacity=1 ] (41.02,5.34) -- (469,5.34) -- (469,280.34) -- (41.02,280.34) -- cycle ;

% Text Node
\draw (358,50.9) node [anchor=north west][inner sep=0.75pt]  [font=\footnotesize,color={rgb, 255:red, 16; green, 18; blue, 125 }  ,opacity=1 ]  {$M$};
% Text Node
\draw (364.65,136.28) node [anchor=north west][inner sep=0.75pt]  [font=\footnotesize]  {$W$};
% Text Node
\draw (154,71.9) node [anchor=north west][inner sep=0.75pt]  [font=\footnotesize]  {$u'$};
% Text Node
\draw (129.5,109.4) node [anchor=north west][inner sep=0.75pt]  [font=\footnotesize]  {$v'$};
% Text Node
\draw (210,99.9) node [anchor=north west][inner sep=0.75pt]  [font=\footnotesize,color={rgb, 255:red, 167; green, 17; blue, 17 }  ,opacity=1 ]  {$\textcolor[rgb]{0.06,0.07,0.49}{m}$};
}
    \only<6>{
%Shape: Rectangle [id:dp22091086801975413] 
\draw  [line width=0.75]  (261.98,12.03) -- (262.02,142.84) -- (150.52,236.47) -- (150.47,105.66) -- cycle ;
%Shape: Circle [id:dp3960031807548443] 
\draw  [color={rgb, 255:red, 16; green, 18; blue, 125 }  ,draw opacity=1 ][fill={rgb, 255:red, 16; green, 18; blue, 125 }  ,fill opacity=1 ] (350.47,93.07) .. controls (351.83,92.71) and (352.93,93.52) .. (352.93,94.88) .. controls (352.93,96.24) and (351.83,97.64) .. (350.47,98.01) .. controls (349.1,98.37) and (348,97.56) .. (348,96.2) .. controls (348,94.84) and (349.1,93.44) .. (350.47,93.07) -- cycle ;
%Shape: Circle [id:dp7170017338469822] 
\draw  [fill={rgb, 255:red, 0; green, 0; blue, 0 }  ,fill opacity=1 ] (365.12,130.42) .. controls (366.48,130.42) and (367.58,131.52) .. (367.58,132.88) .. controls (367.58,134.25) and (366.48,135.35) .. (365.12,135.35) .. controls (363.75,135.35) and (362.65,134.25) .. (362.65,132.88) .. controls (362.65,131.52) and (363.75,130.42) .. (365.12,130.42) -- cycle ;
%Straight Lines [id:da925005109202037] 
\draw  [dash pattern={on 4.5pt off 4.5pt}]  (150.02,143.52) -- (150.47,105.66) ;
\draw [shift={(150,145.52)}, rotate = 270.68] [color={rgb, 255:red, 0; green, 0; blue, 0 }  ][line width=0.75]    (10.93,-3.29) .. controls (6.95,-1.4) and (3.31,-0.3) .. (0,0) .. controls (3.31,0.3) and (6.95,1.4) .. (10.93,3.29)   ;
%Straight Lines [id:da7341104591578826] 
\draw  [dash pattern={on 4.5pt off 4.5pt}]  (182.47,78.8) -- (150.47,105.66) ;
\draw [shift={(184,77.52)}, rotate = 139.99] [color={rgb, 255:red, 0; green, 0; blue, 0 }  ][line width=0.75]    (10.93,-3.29) .. controls (6.95,-1.4) and (3.31,-0.3) .. (0,0) .. controls (3.31,0.3) and (6.95,1.4) .. (10.93,3.29)   ;
%Shape: Circle [id:dp054443755291766927] 
\draw  [color={rgb, 255:red, 16; green, 18; blue, 125 }  ,draw opacity=1 ] (209.53,118.71) .. controls (210.9,118.35) and (212,119.15) .. (212,120.52) .. controls (212,121.88) and (210.9,123.28) .. (209.53,123.64) .. controls (208.17,124.01) and (207.07,123.2) .. (207.07,121.84) .. controls (207.07,120.48) and (208.17,119.08) .. (209.53,118.71) -- cycle ;
%Straight Lines [id:da5500031803707174] 
\draw [color={rgb, 255:red, 16; green, 18; blue, 125 }  ,draw opacity=1 ][fill={rgb, 255:red, 16; green, 18; blue, 125 }  ,fill opacity=1 ]   (350.47,95.54) -- (212,120.52) ;
%Straight Lines [id:da19418058541638872] 
\draw    (365.12,130.42) -- (331,129.27) ;
\draw [shift={(329,129.2)}, rotate = 1.93] [color={rgb, 255:red, 0; green, 0; blue, 0 }  ][line width=0.75]    (10.93,-3.29) .. controls (6.95,-1.4) and (3.31,-0.3) .. (0,0) .. controls (3.31,0.3) and (6.95,1.4) .. (10.93,3.29)   ;
%Straight Lines [id:da06456179839803622] 
\draw    (365.12,130.42) -- (380.86,114.19) ;
\draw [shift={(382.25,112.75)}, rotate = 134.12] [color={rgb, 255:red, 0; green, 0; blue, 0 }  ][line width=0.75]    (10.93,-3.29) .. controls (6.95,-1.4) and (3.31,-0.3) .. (0,0) .. controls (3.31,0.3) and (6.95,1.4) .. (10.93,3.29)   ;
%Straight Lines [id:da6878817592231475] 
\draw    (364.65,132.88) -- (364.04,100.2) ;
\draw [shift={(364,98.2)}, rotate = 88.93] [color={rgb, 255:red, 0; green, 0; blue, 0 }  ][line width=0.75]    (10.93,-3.29) .. controls (6.95,-1.4) and (3.31,-0.3) .. (0,0) .. controls (3.31,0.3) and (6.95,1.4) .. (10.93,3.29)   ;
%Shape: Cube [id:dp39297068054343187] 
\draw   (325.13,88.8) -- (342.27,71.67) -- (382.25,71.67) -- (382.25,112.75) -- (365.12,129.88) -- (325.13,129.88) -- cycle ; \draw   (382.25,71.67) -- (365.12,88.8) -- (325.13,88.8) ; \draw   (365.12,88.8) -- (365.12,129.88) ;
%Straight Lines [id:da3701505179737167] 
\draw [color={rgb, 255:red, 0; green, 0; blue, 0 }  ,draw opacity=1 ]   (75.12,154.88) -- (75.97,90.75) ;
\draw [shift={(76,88.75)}, rotate = 90.77] [color={rgb, 255:red, 0; green, 0; blue, 0 }  ,draw opacity=1 ][line width=0.75]    (10.93,-3.29) .. controls (6.95,-1.4) and (3.31,-0.3) .. (0,0) .. controls (3.31,0.3) and (6.95,1.4) .. (10.93,3.29)   ;
%Straight Lines [id:da9863793705895071] 
\draw [color={rgb, 255:red, 0; green, 0; blue, 0 }  ,draw opacity=1 ]   (75.12,154.88) -- (116.31,128.59) ;
\draw [shift={(118,127.52)}, rotate = 147.45] [color={rgb, 255:red, 0; green, 0; blue, 0 }  ,draw opacity=1 ][line width=0.75]    (10.93,-3.29) .. controls (6.95,-1.4) and (3.31,-0.3) .. (0,0) .. controls (3.31,0.3) and (6.95,1.4) .. (10.93,3.29)   ;
%Straight Lines [id:da21691112687890424] 
\draw [color={rgb, 255:red, 0; green, 0; blue, 0 }  ,draw opacity=1 ]   (75.12,154.88) -- (128,153.57) ;
\draw [shift={(130,153.52)}, rotate = 178.57] [color={rgb, 255:red, 0; green, 0; blue, 0 }  ,draw opacity=1 ][line width=0.75]    (10.93,-3.29) .. controls (6.95,-1.4) and (3.31,-0.3) .. (0,0) .. controls (3.31,0.3) and (6.95,1.4) .. (10.93,3.29)   ;
%Straight Lines [id:da5973484313893659] 
\draw [color={rgb, 255:red, 16; green, 18; blue, 125 }  ,draw opacity=1 ]   (157,133.52) -- (75.12,154.88) ;
%Straight Lines [id:da9607014603660214] 
\draw [color={rgb, 255:red, 0; green, 0; blue, 0 }  ,draw opacity=1 ] [dash pattern={on 0.84pt off 2.51pt}]  (211.07,120.84) -- (157,133.52) ;
%Shape: Rectangle [id:dp47930549648846155] 
\draw  [color={rgb, 255:red, 255; green, 255; blue, 255 }  ,draw opacity=1 ] (41.02,5.34) -- (469,5.34) -- (469,280.34) -- (41.02,280.34) -- cycle ;

% Text Node
\draw (358,50.9) node [anchor=north west][inner sep=0.75pt]  [font=\footnotesize,color={rgb, 255:red, 16; green, 18; blue, 125 }  ,opacity=1 ]  {$M$};
% Text Node
\draw (364.65,136.28) node [anchor=north west][inner sep=0.75pt]  [font=\footnotesize]  {$W$};
% Text Node
\draw (154,71.9) node [anchor=north west][inner sep=0.75pt]  [font=\footnotesize]  {$u'$};
% Text Node
\draw (129.5,109.4) node [anchor=north west][inner sep=0.75pt]  [font=\footnotesize]  {$v'$};
% Text Node
\draw (210,99.9) node [anchor=north west][inner sep=0.75pt]  [font=\footnotesize,color={rgb, 255:red, 167; green, 17; blue, 17 }  ,opacity=1 ]  {$\textcolor[rgb]{0.06,0.07,0.49}{m}$};
% Text Node
\draw (56.65,159.28) node [anchor=north west][inner sep=0.75pt]  [font=\footnotesize]  {$C$};
% Text Node
\draw (61,70.9) node [anchor=north west][inner sep=0.75pt]  [font=\footnotesize]  {$y$};
}
    \only<7>{\input{donnees/s2_7.tex}}
    \only<8>{
%Shape: Rectangle [id:dp20518050829306933] 
\draw  [line width=0.75]  (261.98,12.03) -- (262.02,142.84) -- (150.52,236.47) -- (150.47,105.66) -- cycle ;
%Shape: Circle [id:dp48036249842364387] 
\draw  [color={rgb, 255:red, 16; green, 18; blue, 125 }  ,draw opacity=1 ][fill={rgb, 255:red, 16; green, 18; blue, 125 }  ,fill opacity=1 ] (350.47,93.07) .. controls (351.83,92.71) and (352.93,93.52) .. (352.93,94.88) .. controls (352.93,96.24) and (351.83,97.64) .. (350.47,98.01) .. controls (349.1,98.37) and (348,97.56) .. (348,96.2) .. controls (348,94.84) and (349.1,93.44) .. (350.47,93.07) -- cycle ;
%Shape: Circle [id:dp15975553735025072] 
\draw  [fill={rgb, 255:red, 0; green, 0; blue, 0 }  ,fill opacity=1 ] (365.12,130.42) .. controls (366.48,130.42) and (367.58,131.52) .. (367.58,132.88) .. controls (367.58,134.25) and (366.48,135.35) .. (365.12,135.35) .. controls (363.75,135.35) and (362.65,134.25) .. (362.65,132.88) .. controls (362.65,131.52) and (363.75,130.42) .. (365.12,130.42) -- cycle ;
%Straight Lines [id:da04569264629085701] 
\draw  [dash pattern={on 4.5pt off 4.5pt}]  (150.02,143.52) -- (150.47,105.66) ;
\draw [shift={(150,145.52)}, rotate = 270.68] [color={rgb, 255:red, 0; green, 0; blue, 0 }  ][line width=0.75]    (10.93,-3.29) .. controls (6.95,-1.4) and (3.31,-0.3) .. (0,0) .. controls (3.31,0.3) and (6.95,1.4) .. (10.93,3.29)   ;
%Straight Lines [id:da34135028023903646] 
\draw  [dash pattern={on 4.5pt off 4.5pt}]  (182.47,78.8) -- (150.47,105.66) ;
\draw [shift={(184,77.52)}, rotate = 139.99] [color={rgb, 255:red, 0; green, 0; blue, 0 }  ][line width=0.75]    (10.93,-3.29) .. controls (6.95,-1.4) and (3.31,-0.3) .. (0,0) .. controls (3.31,0.3) and (6.95,1.4) .. (10.93,3.29)   ;
%Shape: Circle [id:dp9804788379329312] 
\draw  [color={rgb, 255:red, 16; green, 18; blue, 125 }  ,draw opacity=1 ] (209.53,118.71) .. controls (210.9,118.35) and (212,119.15) .. (212,120.52) .. controls (212,121.88) and (210.9,123.28) .. (209.53,123.64) .. controls (208.17,124.01) and (207.07,123.2) .. (207.07,121.84) .. controls (207.07,120.48) and (208.17,119.08) .. (209.53,118.71) -- cycle ;
%Straight Lines [id:da9492695065393072] 
\draw [color={rgb, 255:red, 16; green, 18; blue, 125 }  ,draw opacity=1 ][fill={rgb, 255:red, 16; green, 18; blue, 125 }  ,fill opacity=1 ]   (350.47,95.54) -- (212,120.52) ;
%Straight Lines [id:da3349950281049371] 
\draw    (365.12,130.42) -- (331,129.27) ;
\draw [shift={(329,129.2)}, rotate = 1.93] [color={rgb, 255:red, 0; green, 0; blue, 0 }  ][line width=0.75]    (10.93,-3.29) .. controls (6.95,-1.4) and (3.31,-0.3) .. (0,0) .. controls (3.31,0.3) and (6.95,1.4) .. (10.93,3.29)   ;
%Straight Lines [id:da036984224071377914] 
\draw    (365.12,130.42) -- (380.86,114.19) ;
\draw [shift={(382.25,112.75)}, rotate = 134.12] [color={rgb, 255:red, 0; green, 0; blue, 0 }  ][line width=0.75]    (10.93,-3.29) .. controls (6.95,-1.4) and (3.31,-0.3) .. (0,0) .. controls (3.31,0.3) and (6.95,1.4) .. (10.93,3.29)   ;
%Straight Lines [id:da6787819066580656] 
\draw    (364.65,132.88) -- (364.04,100.2) ;
\draw [shift={(364,98.2)}, rotate = 88.93] [color={rgb, 255:red, 0; green, 0; blue, 0 }  ][line width=0.75]    (10.93,-3.29) .. controls (6.95,-1.4) and (3.31,-0.3) .. (0,0) .. controls (3.31,0.3) and (6.95,1.4) .. (10.93,3.29)   ;
%Shape: Cube [id:dp8344554965778904] 
\draw   (325.13,88.8) -- (342.27,71.67) -- (382.25,71.67) -- (382.25,112.75) -- (365.12,129.88) -- (325.13,129.88) -- cycle ; \draw   (382.25,71.67) -- (365.12,88.8) -- (325.13,88.8) ; \draw   (365.12,88.8) -- (365.12,129.88) ;
%Straight Lines [id:da9940442974254676] 
\draw [color={rgb, 255:red, 0; green, 0; blue, 0 }  ,draw opacity=1 ] [dash pattern={on 0.84pt off 2.51pt}]  (338,148.2) -- (75.12,154.88) ;
%Straight Lines [id:da09913909765003492] 
\draw [color={rgb, 255:red, 0; green, 0; blue, 0 }  ,draw opacity=1 ]   (75.12,154.88) -- (75.97,90.75) ;
\draw [shift={(76,88.75)}, rotate = 90.77] [color={rgb, 255:red, 0; green, 0; blue, 0 }  ,draw opacity=1 ][line width=0.75]    (10.93,-3.29) .. controls (6.95,-1.4) and (3.31,-0.3) .. (0,0) .. controls (3.31,0.3) and (6.95,1.4) .. (10.93,3.29)   ;
%Straight Lines [id:da954541250044646] 
\draw [color={rgb, 255:red, 0; green, 0; blue, 0 }  ,draw opacity=1 ]   (75.12,154.88) -- (116.31,128.59) ;
\draw [shift={(118,127.52)}, rotate = 147.45] [color={rgb, 255:red, 0; green, 0; blue, 0 }  ,draw opacity=1 ][line width=0.75]    (10.93,-3.29) .. controls (6.95,-1.4) and (3.31,-0.3) .. (0,0) .. controls (3.31,0.3) and (6.95,1.4) .. (10.93,3.29)   ;
%Straight Lines [id:da8173811873321303] 
\draw [color={rgb, 255:red, 0; green, 0; blue, 0 }  ,draw opacity=1 ]   (75.12,154.88) -- (128,153.57) ;
\draw [shift={(130,153.52)}, rotate = 178.57] [color={rgb, 255:red, 0; green, 0; blue, 0 }  ,draw opacity=1 ][line width=0.75]    (10.93,-3.29) .. controls (6.95,-1.4) and (3.31,-0.3) .. (0,0) .. controls (3.31,0.3) and (6.95,1.4) .. (10.93,3.29)   ;
%Straight Lines [id:da6590312752180065] 
\draw [color={rgb, 255:red, 16; green, 18; blue, 125 }  ,draw opacity=1 ]   (157,133.52) -- (75.12,154.88) ;
%Straight Lines [id:da9534946249718639] 
\draw [color={rgb, 255:red, 0; green, 0; blue, 0 }  ,draw opacity=1 ]   (86,179.5) -- (192,178.53) ;
\draw [shift={(194,178.52)}, rotate = 179.48] [color={rgb, 255:red, 0; green, 0; blue, 0 }  ,draw opacity=1 ][line width=0.75]    (10.93,-3.29) .. controls (6.95,-1.4) and (3.31,-0.3) .. (0,0) .. controls (3.31,0.3) and (6.95,1.4) .. (10.93,3.29)   ;
\draw [shift={(84,179.52)}, rotate = 359.48] [color={rgb, 255:red, 0; green, 0; blue, 0 }  ,draw opacity=1 ][line width=0.75]    (10.93,-3.29) .. controls (6.95,-1.4) and (3.31,-0.3) .. (0,0) .. controls (3.31,0.3) and (6.95,1.4) .. (10.93,3.29)   ;
%Straight Lines [id:da2603521647004613] 
\draw [color={rgb, 255:red, 0; green, 0; blue, 0 }  ,draw opacity=1 ] [dash pattern={on 0.84pt off 2.51pt}]  (211.07,120.84) -- (157,133.52) ;
%Shape: Rectangle [id:dp15198164582240803] 
\draw  [color={rgb, 255:red, 255; green, 255; blue, 255 }  ,draw opacity=1 ] (41.02,5.34) -- (469,5.34) -- (469,280.34) -- (41.02,280.34) -- cycle ;
%Straight Lines [id:da6760200769193868] 
\draw    (202.02,152.01) -- (201.04,102.52) ;
\draw [shift={(201,100.52)}, rotate = 88.86] [color={rgb, 255:red, 0; green, 0; blue, 0 }  ][line width=0.75]    (10.93,-3.29) .. controls (6.95,-1.4) and (3.31,-0.3) .. (0,0) .. controls (3.31,0.3) and (6.95,1.4) .. (10.93,3.29)   ;
%Straight Lines [id:da9313328046524404] 
\draw    (202.56,151.13) -- (235.26,132.51) ;
\draw [shift={(237,131.52)}, rotate = 150.34] [color={rgb, 255:red, 0; green, 0; blue, 0 }  ][line width=0.75]    (10.93,-3.29) .. controls (6.95,-1.4) and (3.31,-0.3) .. (0,0) .. controls (3.31,0.3) and (6.95,1.4) .. (10.93,3.29)   ;

% Text Node
\draw (358,50.9) node [anchor=north west][inner sep=0.75pt]  [font=\footnotesize,color={rgb, 255:red, 16; green, 18; blue, 125 }  ,opacity=1 ]  {$M$};
% Text Node
\draw (364.65,136.28) node [anchor=north west][inner sep=0.75pt]  [font=\footnotesize]  {$W$};
% Text Node
\draw (154,71.9) node [anchor=north west][inner sep=0.75pt]  [font=\footnotesize]  {$u'$};
% Text Node
\draw (129.5,109.4) node [anchor=north west][inner sep=0.75pt]  [font=\footnotesize]  {$v'$};
% Text Node
% Text Node
\draw (210,99.9) node [anchor=north west][inner sep=0.75pt]  [font=\footnotesize,color={rgb, 255:red, 167; green, 17; blue, 17 }  ,opacity=1 ]  {$\textcolor[rgb]{0.06,0.07,0.49}{m}$};
% Text Node
\draw (294,159) node [anchor=north west][inner sep=0.75pt]   [align=left] {axe optique};
% Text Node
\draw (126,192.9) node [anchor=north west][inner sep=0.75pt]  [font=\footnotesize]  {$f$};
% Text Node
\draw (56.65,159.28) node [anchor=north west][inner sep=0.75pt]  [font=\footnotesize]  {$C$};
% Text Node
\draw (61,70.9) node [anchor=north west][inner sep=0.75pt]  [font=\footnotesize]  {$y$};
% Text Node
\draw (101,112.9) node [anchor=north west][inner sep=0.75pt]  [font=\footnotesize]  {$x$};
% Text Node
\draw (184.65,156.6) node [anchor=north west][inner sep=0.75pt]  [font=\footnotesize]  {$C'$};
% Text Node
\draw (241,123.9) node [anchor=north west][inner sep=0.75pt]  [font=\footnotesize]  {$v$};
% Text Node
\draw (197,79.92) node [anchor=north west][inner sep=0.75pt]  [font=\footnotesize]  {$u$};

}
    \end{tikzpicture}
  \end{overlayarea}
\end{minipage}
\hfill
\begin{minipage}[c]{0.48\linewidth}
  \vspace*{\fill}
  \begin{itemize}
    \item<2-> $M$ : point réel
    \item<3-> $W$ : origine du repère du monde
    \item<4-> $(u', v')$ : coordonnées dans le plan image en pixels
    \item<5-> $m$ : projection de $M$ dans le plan image
    \item<6-> $C$ : origine du repère de la caméra
    \item<8-> $C'$ : origine du repère de l'image par projection de C
  \end{itemize}
  \vspace*{\fill}
\end{minipage}
\end{frame}

\begin{frame}{Projection d’un point 3D sur le plan image}
\begin{minipage}[c]{0.48\linewidth}
  \begin{overlayarea}{\linewidth}{5cm}
    \only<1>{
      Par le théorème de Thalès (projection perspective) \\[0.3em]
      $\begin{array}{rcl}
      u &=& f x_c \\
      v &=& f y_c \\
      w &=& z_c
      \end{array}$
    }

    \only<2>{
      Par le théorème de Thalès (projection perspective) \\[0.3em]
      $\begin{array}{rcl}
      u &=& f x_c \\
      v &=& f y_c \\
      w &=& z_c
      \end{array}$
      \\[0.5em]
      $\displaystyle
      \begin{bmatrix}
      u \\ v \\ w
      \end{bmatrix}
      =
      \begin{bmatrix}
      f & 0 & 0 & 0 \\
      0 & f & 0 & 0 \\
      0 & 0 & 1 & 0
      \end{bmatrix}
      \begin{bmatrix}
      x_c \\ y_c \\ z_c \\ 1
      \end{bmatrix}
      $
    }

    \only<3>{
      Le changement de repère s’écrit avec une transformation homogène :\\[0.5em]
      $\displaystyle
      \begin{bmatrix}
      x_c \\ y_c \\ z_c \\ 1
      \end{bmatrix}
      =
      \begin{bmatrix}
      R & T \\
      0 & 1
      \end{bmatrix}
      \begin{bmatrix}
      X_w \\ Y_w \\ Z_w \\ 1
      \end{bmatrix}
      $
  
      Où $R \in \mathbb{R}^{3\times3}$ est une rotation, $T \in \mathbb{R}^3$ une translation.
    }
  \end{overlayarea}
\end{minipage}
\hfill
\begin{minipage}[c]{0.48\linewidth}
  \centering
  \begin{overlayarea}{0.9\linewidth}{4cm}
    \hspace*{-1cm}
    \begin{tikzpicture}[x=0.75pt,y=0.75pt,yscale=-1,xscale=1, scale=0.6]
      \input{donnees/s2_9.tex}
    \end{tikzpicture}
  \end{overlayarea}
\end{minipage}
\end{frame}


\begin{frame}{Projection d’un point 3D sur le plan image}
  \centering
    $\displaystyle
    \begin{bmatrix}
    u \\ v \\ w
    \end{bmatrix}
    =
    \begin{bmatrix}
    f & 0 & 0 & 0 \\
    0 & f & 0 & 0 \\
    0 & 0 & 1 & 0
    \end{bmatrix}
    \begin{bmatrix}
    R & T \\
    0 & 1
    \end{bmatrix}
    \begin{bmatrix}
    X_w \\ Y_w \\ Z_w \\ 1
    \end{bmatrix}
    $
\end{frame}

%-----------------------------------------------
\begin{frame}
\frametitle{Les différents repères}

\[
\lambda_i 
\begin{pmatrix}
u^{(i)} \\
v^{(i)} \\
1
\end{pmatrix}
=
\begin{pmatrix}
p_{11} & p_{12} & p_{13} & p_{14} \\
p_{21} & p_{22} & p_{23} & p_{24} \\
p_{31} & p_{32} & p_{33} & p_{34}
\end{pmatrix}
\begin{pmatrix}
x_C^{(i)} \\
y_C^{(i)} \\
z_C^{(i)} \\
1
\end{pmatrix}
\]
\end{frame}


\begin{frame}{Modèle de projection — Matrice \( P \)}

\begin{itemize}
  \item<1-> On considère un point 3D \( M = (X, Y, Z) \)
  \item<2-> Il se projette sur un point image \( m = (u, v) \)

  \item<3-> On cherche une relation linéaire homogène :
  \[
  \lambda
  \begin{bmatrix}
  u \cr v \cr 1
  \end{bmatrix}
  P
  =
  \begin{bmatrix}
  X \cr Y \cr Z \cr 1
  \end{bmatrix}
  \]

  \item<4-> \( P \) est une matrice \( 3 \times 4 \), avec 12 inconnues
  \item<5-> En développant les lignes :
  \[
  \lambda u = p_{11}X + p_{12}Y + p_{13}Z + p_{14}
  \]
  \[
  \lambda v = p_{21}X + p_{22}Y + p_{23}Z + p_{24}
  \]
  \[
  \lambda   = p_{31}X + p_{32}Y + p_{33}Z + p_{34}
  \]
\end{itemize}
\end{frame}

%------------------------

\begin{frame}{Équations sans \(\lambda\) et système matriciel}

\begin{itemize}
  \item<1-> Pour un point donné, on élimine \( \lambda \) :
  
  \[
  u (p_{31}X + p_{32}Y + p_{33}Z + p_{34}) = p_{11}X + p_{12}Y + p_{13}Z + p_{14}
  \]
  \[
  v (p_{31}X + p_{32}Y + p_{33}Z + p_{34}) = p_{21}X + p_{22}Y + p_{23}Z + p_{24}
  \]

  \vspace{0.5em}
   \item<2-> Cela donne un système homogène ...
   \[
   \scriptsize
   \begin{cases}
0= & p_{11} x_{C} +p_{12} y_{C}  +p_{13} z_{C} +p_{14} -p_{31} u x_{C} -p_{32} u  z_{C}  -p_{33} u  z_{C}  -p_{34} u \\
0= &  p_{21} x_{C}  +p_{22} y_{C}  +p_{23} z_{C}  +p_{24} -p_{31} v  x_{C}  -p_{32} v  z_{C}  -p_{33} v  z_{C}  -p_{34} v 
\end{cases}
   \]
  \item<3-> En faisant cela pour n points on obtient un système ...
\end{itemize}

\end{frame}

\begin{frame}{Équations sans \(\lambda\) et système matriciel}
  
\[
\resizebox{\textwidth}{!}{$
\left(
\begin{array}{cccccccccccc}
    x_{C}^{(1)} & y_{C}^{(1)} & z_{C}^{(1)} & 1 & 0 & 0 & 0 & 0 & -u^{( 1)} x{C}^{(1)} & -u^{( 1)}y{C}^{(1)} & -u^{( 1)}z_{C}^{(1)} & -u^{(1)}\\
0 & 0 & 0 & 0 & x_{C}^{( 1)} & y_{C}^{( 1)} & z_{C}^{( 1)} & 1 & -v^{( 1)} x_{C}^{( 1)} & -v^{( 1)} y_{C}^{( 1)} & -v^{( 1)} z_{C}^{( 1)} & -v^{( 1)}\\
\vdots  & \vdots  & \vdots  & \vdots  & \vdots  & \vdots  & \vdots  & \vdots  & \vdots  & \vdots  & \vdots  & \vdots \\
x_{C}^{( i)} & y_{C}^{( i)} & z_{C}^{( i)} & 1 & 0 & 0 & 0 & 0 & -u^{( i)} x_{C}^{( i)} & -u^{( i)} y_{C}^{( i)} & -u^{( i)} z_{C}^{( i)} & -u^{( i)}\\
0 & 0 & 0 & 0 & x_{C}^{( i)} & y_{C}^{( i)} & z_{C}^{( i)} & 1 & -v^{( i)} x_{C}^{( i)} & -v^{( i)} y_{C}^{( i)} & -v^{( i)} z_{C}^{( i)} & -v^{( i)}\\
\vdots  & \vdots  & \vdots  & \vdots  & \vdots  & \vdots  & \vdots  & \vdots  & \vdots  & \vdots  & \vdots  & \vdots \\
x_{C}^{( 6)} & y_{C}^{( 6)} & z_{C}^{( 6)} & 1 & 0 & 0 & 0 & 0 & -u^{( 6)} x_{C}^{( 6)} & -u^{( 6)} y_{C}^{( 6)} & -u^{( 6)} z_{C}^{( 6)} & -u^{( 6)}\\
0 & 0 & 0 & 0 & x_{C}^{( 6)} & y_{C}^{( 6)} & z_{C}^{( 6)} & 1 & -v^{( 6)} x_{C}^{( 6)} & -v^{( 6)} y_{C}^{( 6)} & -v^{( 6)} z_{C}^{( 6)} & -v^{( 6)}
\end{array}
\right)
\begin{pmatrix}
p_{11}\\
p_{12}\\
p_{13}\\
p_{14}\\
p_{21}\\
p_{22}\\
p_{23}\\
p_{24}\\
p_{31}\\
p_{32}\\
p_{33}\\
p_{34}
\end{pmatrix} =\begin{pmatrix}
0\\
0\\
0\\
0\\
0\\
0\\
0\\
0\\
0\\
0\\
0\\
0
\end{pmatrix}
$}
\]
\end{frame}

\subsection{En pratiques}

\begin{frame}{Shootin photo : importance de la prise de vue}
  \centering
  \begin{minipage}{0.48\linewidth}
    \centering
    \begin{figure}
      \centering
      \includegraphics[width=0.48\linewidth]{capture/dodec0.jpg}%
      \includegraphics[width=0.48\linewidth]{capture/dodec1.jpg} \\
      \includegraphics[width=0.48\linewidth]{capture/dodec2.jpg}%
      \includegraphics[width=0.48\linewidth]{capture/dodec3.jpg}
      {\footnotesize\textbf{Vues initiales}}
    \end{figure}
  \end{minipage}
  \hfill
  \begin{minipage}{0.48\linewidth}
    \centering
    \begin{figure}
      \centering
      \includegraphics[width=0.48\linewidth]{capture/dodecf0.jpg}%
      \includegraphics[width=0.48\linewidth]{capture/dodecf1.jpg} \\
      \includegraphics[width=0.48\linewidth]{capture/dodecf2.jpg}%
      \includegraphics[width=0.48\linewidth]{capture/dodecf3.jpg}
      {\footnotesize\textbf{Vues améliorées}}
    \end{figure}
  \end{minipage}
\end{frame}


\begin{frame}{Calibration}
 \centering
  \begin{minipage}{0.48\linewidth}
    \centering
    \begin{figure}
      \centering
      \includegraphics[width=0.7\linewidth]{capture/cube_calibrage.jpg}%
      \caption{Cube calibrage}
    \end{figure}
  \end{minipage}
  \hfill
  \begin{minipage}{0.48\linewidth}
    \centering
    \begin{figure}
      \centering
      \includegraphics[width=0.8\linewidth]{capture/selection.png}%
      \caption{Selection des points}
    \end{figure}
  \end{minipage}
\end{frame}



%%%%%%%%%%%%%%%%%%%%%%%%%%%%%%%%%%%%%%%%%%%%%%%%%
\section[Enveloppe Convexe]{Enveloppe Convexe}
%------------------------------------------------
%++++++++++++++++++++++++++++++++++++++++++++++++
%++++++++++++++++++++++++++++++++++++++++++++++++
%------------------------------------------------

\captionsetup[figure]{labelformat=empty}
\tikzset{every picture/.style={line width=0.75pt}} %set default line width to 0.75pt
\begin{frame}{Triangulation}

\begin{center}
\hspace{0.05\textwidth}
\begin{minipage}{0.45\textwidth}
\begin{tikzpicture}[scale=1, x={(1cm,0cm)}, y={(0.5cm,0.5cm)}, z={(0cm,1cm)}]
  % Axes
  \draw[->, thick] (0,0,0) -- (3,0,0) node[anchor=north east] {\(x\)};
  \draw[->, thick] (0,0,0) -- (0,3,0) node[anchor=north west] {\(y\)};
  \draw[->, thick] (0,0,0) -- (0,0,3) node[anchor=south] {\(z\)};

  % Sommets
  \foreach \x/\y/\z/\name in {
    0/0/0/A, 2/0/0/B, 2/2/0/C, 0/2/0/D,
    0/0/2/E, 2/0/2/F, 2/2/2/G, 0/2/2/H}
    {
      \filldraw[black] (\x,\y,\z) circle (1pt) node[anchor=south east] {\tiny \name};
    }

  % Arêtes
  \draw[dashed] (0,0,0) -- (2,0,0) -- (2,2,0) -- (0,2,0) -- cycle;
  \draw[dashed] (0,0,2) -- (2,0,2) -- (2,2,2) -- (0,2,2) -- cycle;
  \draw[dashed] (0,0,0) -- (0,0,2);
  \draw[dashed] (2,0,0) -- (2,0,2);
  \draw[dashed] (2,2,0) -- (2,2,2);
  \draw[dashed] (0,2,0) -- (0,2,2);
\end{tikzpicture}
\end{minipage}
\begin{minipage}{0.45\textwidth}
\begin{tikzpicture}[scale=1, line join=round]
  % Points du cube
  \coordinate (A) at (0,0,0);
  \coordinate (B) at (2,0,0);
  \coordinate (C) at (2,2,0);
  \coordinate (D) at (0,2,0);
  \coordinate (E) at (0,0,2);
  \coordinate (F) at (2,0,2);
  \coordinate (G) at (2,2,2);
  \coordinate (H) at (0,2,2);

  % Triangles - chaque face = 2 triangles
  \fill[blue!50]  (A) -- (B) -- (C) -- cycle;
  \fill[blue!10]  (A) -- (C) -- (D) -- cycle;

  \fill[green!60] (A) -- (B) -- (F) -- cycle;
  \fill[green!20] (A) -- (F) -- (E) -- cycle;

  \fill[red!60]   (B) -- (C) -- (G) -- cycle;
  \fill[red!20]   (B) -- (G) -- (F) -- cycle;

  \fill[yellow!60] (C) -- (D) -- (H) -- cycle;
  \fill[yellow!30] (C) -- (H) -- (G) -- cycle;

 % \fill[orange!50] (D) -- (A) -- (E) -- cycle;
 % \fill[orange!10] (D) -- (E) -- (H) -- cycle;

  \fill[gray!50]   (E) -- (F) -- (G) -- cycle;
  \fill[gray!10]   (E) -- (G) -- (H) -- cycle;

  % Arêtes
%  \draw[thick] (A) -- (B) -- (C) -- (D) -- cycle;
 % \draw[thick] (E) -- (F) -- (G) -- (H) -- cycle;
 % \draw[thick] (A) -- (E);
%  \draw[thick] (B) -- (F);
%  \draw[thick] (C) -- (G);
%  \draw[thick] (D) -- (H);
\end{tikzpicture}
\end{minipage}
\end{center}
\end{frame}

\begin{frame}[fragile]
  \frametitle{Le format \texttt{STL}}

  \begin{columns}
    \column{0.45\textwidth}
    \scriptsize
\begin{verbatim}
solid cube
  facet normal 0 0 1 // Face supérieure
    outer loop
      vertex 0 0 1
      vertex 1 0 1
      vertex 0 1 1
    endloop
  endfacet
  facet normal 0 0 1
    outer loop
      vertex 1 0 1
      vertex 1 1 1
      vertex 0 1 1
    endloop
  endfacet
  // Autres faces...
endsolid cube
\end{verbatim}
    \column{0.55\textwidth}
    \begin{figure}
      \centering
      \includegraphics[width=3.5cm]{capture/cubestl.png} % Remplace par ton image
      \caption{\tiny fichier STL visualisé avec viewstl.com}
    \end{figure}
  \end{columns}
\end{frame}

%%%%%%%%%%%%%%%%%%%%%%%%%%%%%%%%%%%%%%%%%%%%%%%%%
\section[Analyse des résultats]{Analyse des résultats}

%------------------------------------------------
\begin{frame}
\frametitle{Titre d'une slide avant la sous-section}
Ici, on n’a pas encore de titre de sous-section dans le bandeau du haut.\\
\end{frame}


%++++++++++++++++++++++++++++++++++++++++++++++++
\subsection{Quelques exemples}

%------------------------------------------------
\begin{frame}{Aspect pratique}
Ici, on a un titre de sous-section, contrairement à la slide.\\[1.5cm]
Voir le code ici pour référencer une slide avec et la citer avec son numéro via .
\end{frame}

%++++++++++++++++++++++++++++++++++++++++++++++++
\subsection{Critiques}

%------------------------------------------------
\begin{frame}
\frametitle{Titre sans lettre descendant sous la baseline}
Ici c'est mieux, non ?
\end{frame}


% Slide de test/conclusion
\begin{frame}
  test
\end{frame}

\end{document}
