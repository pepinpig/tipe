%%%%%%%%%%%%%%%%%%%%%%%%%%%%%%%%%%%%%%%%%%%%%%%%%
\section[Reconstruction]{Reconstruction}
%------------------------------------------------
\begin{frame}
\frametitle{Titre d'une slide avant la sous-section}
    Ici on n'a pas encore de titre de sous-section dans le bandeau du haut.
    \label{slides_hors_subsec}
\end{frame}

%++++++++++++++++++++++++++++++++++++++++++++++++
\subsection{Modélisation théorique}
%++++++++++++++++++++++++++++++++++++++++++++++++
\begin{frame}
\frametitle{Les différents repères}
  \begin{minipage}{0.58\textwidth}
        \scalebox{0.65}{


\tikzset{every picture/.style={line width=0.75pt}} %set default line width to 0.75pt        

\begin{tikzpicture}[x=0.75pt,y=0.75pt,yscale=-1,xscale=1]
%uncomment if require: \path (0,300); %set diagram left start at 0, and has height of 300

%Shape: Rectangle [id:dp9723727103024109] 
\draw  [line width=0.75]  (273.11,2.68) -- (273.16,155.68) -- (150.53,258.66) -- (150.47,105.66) -- cycle ;
%Shape: Circle [id:dp9239162914529644] 
\draw   (170.87,132.32) .. controls (172.23,131.96) and (173.33,132.77) .. (173.33,134.13) .. controls (173.33,135.49) and (172.23,136.89) .. (170.87,137.26) .. controls (169.5,137.62) and (168.4,136.81) .. (168.4,135.45) .. controls (168.4,134.09) and (169.5,132.69) .. (170.87,132.32) -- cycle ;
%Straight Lines [id:da4507743094695521] 
\draw  [dash pattern={on 0.84pt off 2.51pt}]  (211.82,130.67) -- (45,134.7) ;
%Shape: Circle [id:dp9335082203579057] 
\draw   (499.62,131.92) .. controls (500.98,131.55) and (502.08,132.36) .. (502.08,133.72) .. controls (502.08,135.09) and (500.98,136.49) .. (499.62,136.85) .. controls (498.25,137.22) and (497.15,136.41) .. (497.15,135.05) .. controls (497.15,133.68) and (498.25,132.28) .. (499.62,131.92) -- cycle ;
%Shape: Circle [id:dp5162114026007054] 
\draw  [color={rgb, 255:red, 241; green, 53; blue, 53 }  ,draw opacity=1 ] (512.82,105.92) .. controls (514.18,105.55) and (515.28,106.36) .. (515.28,107.72) .. controls (515.28,109.09) and (514.18,110.49) .. (512.82,110.85) .. controls (511.45,111.22) and (510.35,110.41) .. (510.35,109.05) .. controls (510.35,107.68) and (511.45,106.28) .. (512.82,105.92) -- cycle ;
%Shape: Circle [id:dp4319786220460511] 
\draw  [fill={rgb, 255:red, 0; green, 0; blue, 0 }  ,fill opacity=1 ] (321.12,132.42) .. controls (322.48,132.42) and (323.58,133.52) .. (323.58,134.88) .. controls (323.58,136.25) and (322.48,137.35) .. (321.12,137.35) .. controls (319.75,137.35) and (318.65,136.25) .. (318.65,134.88) .. controls (318.65,133.52) and (319.75,132.42) .. (321.12,132.42) -- cycle ;
%Straight Lines [id:da5164589906920809] 
\draw    (321.12,134.88) -- (321.97,70.75) ;
\draw [shift={(322,68.75)}, rotate = 90.77] [color={rgb, 255:red, 0; green, 0; blue, 0 }  ][line width=0.75]    (10.93,-3.29) .. controls (6.95,-1.4) and (3.31,-0.3) .. (0,0) .. controls (3.31,0.3) and (6.95,1.4) .. (10.93,3.29)   ;
%Straight Lines [id:da7854412907443913] 
\draw    (321.12,134.88) -- (253.5,134.75) ;
\draw [shift={(251.5,134.75)}, rotate = 0.11] [color={rgb, 255:red, 0; green, 0; blue, 0 }  ][line width=0.75]    (10.93,-3.29) .. controls (6.95,-1.4) and (3.31,-0.3) .. (0,0) .. controls (3.31,0.3) and (6.95,1.4) .. (10.93,3.29)   ;
%Straight Lines [id:da3058610918194634] 
\draw    (321.12,134.88) -- (272.48,176.16) ;
\draw [shift={(270.95,177.45)}, rotate = 319.68] [color={rgb, 255:red, 0; green, 0; blue, 0 }  ][line width=0.75]    (10.93,-3.29) .. controls (6.95,-1.4) and (3.31,-0.3) .. (0,0) .. controls (3.31,0.3) and (6.95,1.4) .. (10.93,3.29)   ;
%Straight Lines [id:da5453100201962917] 
\draw  [dash pattern={on 4.5pt off 4.5pt}]  (149.63,172.62) -- (150.47,105.66) ;
\draw [shift={(149.61,174.62)}, rotate = 270.72] [color={rgb, 255:red, 0; green, 0; blue, 0 }  ][line width=0.75]    (10.93,-3.29) .. controls (6.95,-1.4) and (3.31,-0.3) .. (0,0) .. controls (3.31,0.3) and (6.95,1.4) .. (10.93,3.29)   ;
%Straight Lines [id:da4691798453437107] 
\draw  [dash pattern={on 4.5pt off 4.5pt}]  (199.11,64.39) -- (150.47,105.66) ;
\draw [shift={(200.64,63.1)}, rotate = 139.68] [color={rgb, 255:red, 0; green, 0; blue, 0 }  ][line width=0.75]    (10.93,-3.29) .. controls (6.95,-1.4) and (3.31,-0.3) .. (0,0) .. controls (3.31,0.3) and (6.95,1.4) .. (10.93,3.29)   ;
%Straight Lines [id:da9224741393814412] 
\draw [color={rgb, 255:red, 241; green, 45; blue, 45 }  ,draw opacity=1 ]   (512.82,108.38) -- (192.48,152.97) ;
\draw [shift={(190.5,153.25)}, rotate = 352.08] [color={rgb, 255:red, 241; green, 45; blue, 45 }  ,draw opacity=1 ][line width=0.75]    (10.93,-3.29) .. controls (6.95,-1.4) and (3.31,-0.3) .. (0,0) .. controls (3.31,0.3) and (6.95,1.4) .. (10.93,3.29)   ;
%Shape: Circle [id:dp7585208638434278] 
\draw  [color={rgb, 255:red, 241; green, 53; blue, 53 }  ,draw opacity=1 ] (190.5,150.78) .. controls (191.86,150.42) and (192.97,151.23) .. (192.97,152.59) .. controls (192.97,153.95) and (191.86,155.35) .. (190.5,155.72) .. controls (189.14,156.08) and (188.03,155.27) .. (188.03,153.91) .. controls (188.03,152.55) and (189.14,151.15) .. (190.5,150.78) -- cycle ;
%Shape: Cube [id:dp7714367112041981] 
\draw   (499.62,131.92) -- (520.32,111.22) -- (568.62,111.22) -- (568.62,160.52) -- (547.92,181.22) -- (499.62,181.22) -- cycle ; \draw   (568.62,111.22) -- (547.92,131.92) -- (499.62,131.92) ; \draw   (547.92,131.92) -- (547.92,181.22) ;

% Text Node
\draw (517,97.9) node [anchor=north west][inner sep=0.75pt]  [font=\footnotesize,color={rgb, 255:red, 167; green, 17; blue, 17 }  ,opacity=1 ]  {$P$};
% Text Node
\draw (324,59.9) node [anchor=north west][inner sep=0.75pt]  [font=\footnotesize]  {$\vec{j}$};
% Text Node
\draw (280.5,178.9) node [anchor=north west][inner sep=0.75pt]  [font=\footnotesize]  {$\vec{i}$};
% Text Node
\draw (324.43,139.84) node [anchor=north west][inner sep=0.75pt]  [font=\footnotesize]  {$O$};
% Text Node
\draw (252,112.9) node [anchor=north west][inner sep=0.75pt]  [font=\footnotesize]  {$\vec{k}$};
% Text Node
\draw (158.65,118.28) node [anchor=north west][inner sep=0.75pt]  [font=\footnotesize]  {$C$};
% Text Node
\draw (156,62.9) node [anchor=north west][inner sep=0.75pt]  [font=\footnotesize]  {$\vec{u}$};
% Text Node
\draw (125.5,114.4) node [anchor=north west][inner sep=0.75pt]  [font=\footnotesize]  {$\vec{v}$};
% Text Node
\draw (184.5,155.9) node [anchor=north west][inner sep=0.75pt]  [font=\footnotesize,color={rgb, 255:red, 159; green, 16; blue, 16 }  ,opacity=1 ]  {$P'$};


\end{tikzpicture}}
  \end{minipage}
\end{frame}
%-----------------------------------------------
\begin{frame}
\frametitle{Les différents repères}
\[
\lambda_i \begin{pmatrix}
u^{(i)} \\
v^{(i)} \\
1
\end{pmatrix}
=
\begin{pmatrix}
p_{11} & p_{12} & p_{13} & p_{14} \\
p_{21} & p_{22} & p_{23} & p_{24} \\
p_{31} & p_{32} & p_{33} & p_{34}
\end{pmatrix}
\begin{pmatrix}
x_C^{(i)} \\
y_C^{(i)} \\
z_C^{(i)} \\
1
\end{pmatrix}
\]
\end{frame}
%------------------------------------------------
\begin{frame}
\frametitle{Ce qui apparaît dans l'en-tête}
{\tiny
\[
\begin{pmatrix}
x_C^{(1)} & y_C^{(1)} & z_C^{(1)} & 1 & 0 & 0 & 0 & 0 & -u^{(1)}x_C^{(1)} & -u^{(1)}y_C^{(1)} & -u^{(1)}z_C^{(1)} & -u^{(1)} \\
0 & 0 & 0 & 0 & x_C^{(1)} & y_C^{(1)} & z_C^{(1)} & 1 & -v^{(1)}x_C^{(1)} & -v^{(1)}y_C^{(1)} & -v^{(1)}z_C^{(1)} & -v^{(1)} \\
\vdots & \vdots & \vdots & \vdots & \vdots & \vdots & \vdots & \vdots & \vdots & \vdots & \vdots & \vdots \\
x_C^{(6)} & y_C^{(6)} & z_C^{(6)} & 1 & 0 & 0 & 0 & 0 & -u^{(6)}x_C^{(6)} & -u^{(6)}y_C^{(6)} & -u^{(6)}z_C^{(6)} & -u^{(6)} \\
0 & 0 & 0 & 0 & x_C^{(6)} & y_C^{(6)} & z_C^{(6)} & 1 & -v^{(6)}x_C^{(6)} & -v^{(6)}y_C^{(6)} & -v^{(6)}z_C^{(6)} & -v^{(6)}
\end{pmatrix}
\begin{pmatrix}
p_{11}\\p_{12}\\p_{13}\\p_{14}\\
p_{21}\\p_{22}\\p_{23}\\p_{24}\\
p_{31}\\p_{32}\\p_{33}\\p_{34}
\end{pmatrix}
=
\begin{pmatrix}
0\\0\\0\\0\\0\\0\\0\\0\\0\\0\\0\\0
\end{pmatrix}
\]
}
\begin{center}
    soit \( AP = 0 \)
\end{center}
\end{frame}
%++++++++++++++++++++++++++++++++++++++++++++++++
\subsection{Résolution}
%------------------------------------------------
\begin{frame}
\frametitle{Titre de la slide sans lettre descendant sous la baseline}

On souhaite résoudre le système en évitant la solution triviale \( P = 0 \).  
Sachant que la matrice \( P \) ne peut être déterminée qu'à un facteur près, on peut imposer arbitrairement \( \|P\|^2 = 1 \), et reformuler le système comme un problème d'optimisation :

\[
\min_{\|p\|^2 = 1} \|Ap\|^2 = \min_{\|p\|^2 = 1} p^T A^T A p
\]

On introduit :
- \( f(p) = p^T A^T A p \)
- \( g(p) = p^T p - 1 \)

D’après le théorème d’optimisation sous contrainte, au point optimal \( P^* \), il existe un scalaire \( \lambda \) tel que :

\[
\nabla f(P^*) = \lambda \nabla g(P^*)
\]

En posant \( M = A^T A \), alors :

\[
f(p) = \sum_{i=1}^n \sum_{j=1}^n p_i M_{ij} p_j
\]

\[
\frac{\partial f}{\partial p_k} = \sum_{j=1}^n M_{kj} p_j + \sum_{i=1}^n p_i M_{ik} = (Mp)_k + (M^T p)_k
\]

\[
\frac{\partial f}{\partial p} = 2Mp \quad \text{(car \( M \) est symétrique)}
\]

\[
\frac{\partial g}{\partial p} = 2p
\]

On a alors :

\[
\frac{\partial f}{\partial p} = \lambda \frac{\partial g}{\partial p} \quad \Rightarrow \quad \boxed{A^T A p = \lambda p}
\]

C’est une équation aux valeurs propres : \( p \) est un vecteur propre de \( A^T A \), et \( \lambda \) la valeur propre associée.
\end{frame}

%------------------------------------------------
\begin{frame}
\frametitle{Titre de la slide sans lettre descendant sous la baseline}
    Ici c'est mieux non?
\end{frame}

%------------------------------------------------
\begin{frame}[fragile]
\frametitle{Titre de la slide qui marche tout seul grâce au q et au g}
\end{frame}

%++++++++++++++++++++++++++++++++++++++++++++++++
\subsection{Résolution de système surdéterminé}
\subsubsection{SVD}
\subsubsection{Solution approchée}
\subsection{Reconstruction des points}
