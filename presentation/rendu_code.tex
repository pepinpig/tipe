\documentclass[compress]{beamer}

% Inclusion des packages
%================== ENCODAGE & LANGUE ==================
\usepackage[utf8]{inputenc}
\usepackage[T1]{fontenc}
%\usepackage[french]{babel} % Optionnel

%================== MATHS & SYMBOLES ===================
\usepackage{amsmath, amssymb, amsfonts}
\usepackage{yhmath, mathdots, cancel}
\usepackage{siunitx}
\usepackage{gensymb}
\usepackage{textcomp}
\usepackage{pifont}
\usepackage{xspace}


%================== TABLEAUX ============================
\usepackage{array, tabularx, multirow, booktabs}

%================== COULEURS & GRAPHISMES ===============
\usepackage{color}
\usepackage{tikz}
\usetikzlibrary{
  shapes.geometric,
  backgrounds,
  fadings,
  patterns,
  shadows.blur,
  shapes,
  positioning,
  decorations.pathreplacing
}
\usepackage{xcolor} 

%================== MISE EN PAGE ========================
\usepackage{changepage}
\usepackage{calc}
\usepackage{caption}
\usepackage{xspace}
\usepackage{ragged2e}
\usepackage{amsmath, amsfonts, mathtools, amsthm, amssymb}
\usepackage{adjustbox}
\usepackage{caption}

%================== ALGORITHMIQUE =======================
\usepackage[ruled,vlined]{algorithm2e}
\SetAlgorithmName{Algorithme}{Algo}{Liste des algorithmes}
\SetFuncArgSty{textup}
\SetArgSty{textup}
\SetKwFor{ForEach}{pour tout}{faire}{}
\SetKwFor{For}{pour}{faire}{finpour}
\SetKwIF{Si}{SinonSi}{Sinon}{si}{alors}{sinon si}{sinon}{}
\SetKwInput{Input}{Entrée}
\SetKwInput{Output}{Sortie}
\SetKwProg{myproc}{Procédure}{:}{}
\SetKw{Return}{retourner}
\SetKwComment{tcc}{(*}{*)}
\SetKwFor{Tq}{tant que}{faire}{}
\SetKwRepeat{Repeter}{répéter}{jusqu’à}

%================== AUTRES ==============================
\usepackage[clock]{ifsym}



% Définir les initiales pour affichage dans l'en-tête (à utiliser dans thème perso si besoin)
\newcommand{\initiales}{L.-H. Cuingnet}

% Paramètres Beamer et thème (à personnaliser dans ce fichier)
\usetheme{CambridgeUS}
\usecolortheme{seahorse}


%--------marges
\setbeamersize{text margin left= 0.7cm}
\setbeamersize{text margin right= 0.7cm}

%--------tête et pieds
\setbeamertemplate{navigation symbols}{}
\setbeamertemplate{footline}[frame number]
\setbeamertemplate{headline}{
  %la premiere ligne
  	\begin{beamercolorbox}[ht=0.42cm, vmode]{section in head/foot}
	\hspace{0.4cm} \insertshorttitle 
	\hspace*{0.1cm}- \initiales - {\insertshortdate}
	\vspace*{0.08cm} 
  	\end{beamercolorbox}
  %la deuxième ligne
	\begin{beamercolorbox}[ht=0.4cm, vmode]{subsection in head/foot}
	%titre de la section si elle est pas 0
		\ifnum\value{section}=0{} 
		\else{ \hspace{0.8cm} \thesection - \insertsectionhead }
		\fi
	%séparateur + titre de la sous-section si elle est pas 0
		\ifnum\value{subsection}=0{} 
		\else{ 
			\hspace*{0.1cm} \couleur{$\bullet$} \hspace*{0.1cm} 
			\thesection.\thesubsection \, \insertsubsectionhead
		}
		\fi
		\vspace*{0.12cm}
\end{beamercolorbox}
%\vspace*{-0.03cm} %pour pas qu'il y ait d'espace avec la ligne de frametitle
 }
%\setbeamertemplate{frametitleheigth}{4cm}
\setbeamertemplate{frametitle}{
	\vspace*{-0.04cm} 
	\begin{beamercolorbox}[ht=0.8cm,wd=\paperwidth, vmode]{frametitle}
		\hspace{0.3cm} \insertframetitle \vspace*{0.1cm}
	\end{beamercolorbox}
}

%commande pour ajuster l'alignement vertical des titres sans lettres descendantes
%\newcommand{\esp}{\\[0.1cm]} %--version qui marche sans le package minted
\newcommand{\esp}{\\[-0.5cm]} %--version qui marche avec le package minted


%--------couleurs
\setbeamercolor{structure}{fg=turquoiseFonce!70!black} 

\setbeamercolor{block title}{fg=turquoiseFonce!70!black,bg=vertdEau}
\setbeamercolor{block body}{bg=vertdEau!10!white}

\setbeamercolor{block title alerted}{bg=vertdEau!85!white,fg=turquoiseFonce!80!black}
\setbeamercolor{block body alerted}{bg=vertdEau!8!white}
%\setbeamercolor{alerted text}{fg=red}

\setbeamercolor{block title example}{bg=vertdEau,fg=turquoiseFonce}
\setbeamercolor{block body example}{bg=vertdEau!10!white}
\setbeamercolor{example text}{fg=blue!20!turquoise}

%-------- TOC
\setbeamertemplate{section in toc}[sections numbered]

%-----------------------------------------------
%Plan qui s'affiche au début de chaque section %|
\AtBeginSection[]{                             %|
\begin{frame}[plain]                           %|
\frametitle{Plan\\[0.1cm]}                     %|
\tableofcontents[                              %|
currentsection,                                %|
hideothersubsections,                          %|
subsubsectionstyle=hide]                       %|
\addtocounter{framenumber}{-1}                 %|
\end{frame}}                                   %|
%-----------------------------------------------




%-------- commande pour les ref sur les slides
\newcommand{\bandeauREF}[1]{
\noindent\makebox[\textwidth][l]{%
\hspace{-\dimexpr\oddsidemargin+1in}%
\colorbox{expli!20!white}{%
\parbox{\dimexpr\paperwidth-2\fboxsep}{
\footnotesize\textcolor{expli!80!black}{#1}
}}}}
% Commandes utilitaires
\tikzset{every picture/.style={line width=0.75pt}} %set default line width to 0.75pt        
\newcommand{\imageFrame}{
  \draw [line width=0.75] (261.98,12.03) -- (262.02,142.84) -- (150.52,236.47) -- (150.47,105.66) -- cycle ;
}
\newcommand{\lin}[1]{\texttt{#1}} % Évite minted si pas nécessaire
\newcommand{\flch}{\item[$\rightarrow$]}
\newcommand{\dc}{{\usebeamercolor[fg]{structure}$\hookrightarrow$}}
\newcommand{\ok}{\textcolor{green}{\checkmark}}
\newcommand{\point}{{\usebeamercolor[fg]{structure}$\bullet\enskip$}}
\newcommand{\Point}{\point}
\newcommand{\couleur}[1]{{\usebeamercolor[fg]{structure}#1}}
\newcommand{\important}[1]{\couleur{\textbf{#1}}}
\newcommand{\remarque}[1]{\textit{\textrm{#1}}}
\newcommand{\cmark}{\ding{51}\xspace} % check ✓
\newcommand{\xmark}{\ding{55}\xspace} % cross ✗

% Palette de couleurs personnalisée
\input{la_palette.tex}
\definecolor{hellseahorse}{RGB}{204, 204, 255}
\definecolor{seahorse}{RGB}{204,180, 255}
\definecolor{darkseahorse}{RGB}{83, 74, 196}

\title[Reconstruction 3D]{Reconstruction d’objets convexes à partir de photographies}
\author{
  \texorpdfstring{
    \large Présentation de \important{Lucie-Hélène Cuingnet}\\[0.2cm]
    \footnotesize Travail réalisé avec \couleur{Barnabé Baruchel}
    }{Lucie-Hélène Cuingnet et Barnabé Baruchel}
}
\date[Mai 2025]{TIPE 2025}
%\setbeameroption{show notes on second screen=right}
%================== DÉBUT DU DOCUMENT ===================
\begin{document}

\section*{Appendix}
\subsection*{Compléments}
\begin{frame}[fragile]{appariement.c (partie 1)}
\vspace{-1em}
\inputminted[
  firstline=1,
  lastline=30,
  breaklines=true,
  fontsize=\tiny,
  frame=lines,
  framesep=1mm
]{c}{../appariement.c}
\end{frame}

\begin{frame}[fragile]{appariement.c (partie 2)}
\vspace{-1em}
\inputminted[
  firstline=31,
  lastline=60,
  breaklines=true,
  fontsize=\tiny,
  frame=lines,
  framesep=1mm
]{c}{../appariement.c}
\end{frame}

\begin{frame}[fragile]{appariement.c (partie 3)}
\vspace{-1em}
\inputminted[
  firstline=61,
  lastline=90,
  breaklines=true,
  fontsize=\tiny,
  frame=lines,
  framesep=1mm
]{c}{../appariement.c}
\end{frame}

\begin{frame}[fragile]{appariement.c (partie 4)}
\vspace{-1em}
\inputminted[
  firstline=91,
  lastline=120,
  breaklines=true,
  fontsize=\tiny,
  frame=lines,
  framesep=1mm
]{c}{../appariement.c}
\end{frame}

\begin{frame}[fragile]{appariement.c (partie 5)}
\vspace{-1em}
\inputminted[
  firstline=121,
  lastline=150,
  breaklines=true,
  fontsize=\tiny,
  frame=lines,
  framesep=1mm
]{c}{../appariement.c}
\end{frame}

\begin{frame}[fragile]{appariement.c (partie 6)}
\vspace{-1em}
\inputminted[
  firstline=151,
  lastline=164,
  breaklines=true,
  fontsize=\tiny,
  frame=lines,
  framesep=1mm
]{c}{../appariement.c}
\end{frame}


\begin{frame}[fragile]{camera\_calibration.c (partie 1)}
\vspace{-1em}
\inputminted[
  firstline=1,
  lastline=30,
  breaklines=true,
  fontsize=\tiny,
  frame=lines,
  framesep=1mm
]{c}{../camera_calibration.c}
\end{frame}

\begin{frame}[fragile]{camera\_calibration.c (partie 2)}
\vspace{-1em}
\inputminted[
  firstline=31,
  lastline=60,
  breaklines=true,
  fontsize=\tiny,
  frame=lines,
  framesep=1mm
]{c}{../camera_calibration.c}
\end{frame}

\begin{frame}[fragile]{camera\_calibration.c (partie 3)}
\vspace{-1em}
\inputminted[
  firstline=61,
  lastline=90,
  breaklines=true,
  fontsize=\tiny,
  frame=lines,
  framesep=1mm
]{c}{../camera_calibration.c}
\end{frame}

\begin{frame}[fragile]{camera\_calibration.c (partie 4)}
\vspace{-1em}
\inputminted[
  firstline=91,
  lastline=120,
  breaklines=true,
  fontsize=\tiny,
  frame=lines,
  framesep=1mm
]{c}{../camera_calibration.c}
\end{frame}

\begin{frame}[fragile]{camera\_calibration.c (partie 5)}
\vspace{-1em}
\inputminted[
  firstline=121,
  lastline=134,
  breaklines=true,
  fontsize=\tiny,
  frame=lines,
  framesep=1mm
]{c}{../camera_calibration.c}
\end{frame}


\begin{frame}[fragile]{constante.c (partie 1)}
\vspace{-1em}
\inputminted[
  firstline=1,
  lastline=23,
  breaklines=true,
  fontsize=\tiny,
  frame=lines,
  framesep=1mm
]{c}{../constante.c}
\end{frame}


\begin{frame}[fragile]{detection.c (partie 1)}
\vspace{-1em}
\inputminted[
  firstline=1,
  lastline=30,
  breaklines=true,
  fontsize=\tiny,
  frame=lines,
  framesep=1mm
]{c}{../detection.c}
\end{frame}

\begin{frame}[fragile]{detection.c (partie 2)}
\vspace{-1em}
\inputminted[
  firstline=31,
  lastline=60,
  breaklines=true,
  fontsize=\tiny,
  frame=lines,
  framesep=1mm
]{c}{../detection.c}
\end{frame}

\begin{frame}[fragile]{detection.c (partie 3)}
\vspace{-1em}
\inputminted[
  firstline=61,
  lastline=90,
  breaklines=true,
  fontsize=\tiny,
  frame=lines,
  framesep=1mm
]{c}{../detection.c}
\end{frame}

\begin{frame}[fragile]{detection.c (partie 4)}
\vspace{-1em}
\inputminted[
  firstline=91,
  lastline=120,
  breaklines=true,
  fontsize=\tiny,
  frame=lines,
  framesep=1mm
]{c}{../detection.c}
\end{frame}

\begin{frame}[fragile]{detection.c (partie 5)}
\vspace{-1em}
\inputminted[
  firstline=121,
  lastline=130,
  breaklines=true,
  fontsize=\tiny,
  frame=lines,
  framesep=1mm
]{c}{../detection.c}
\end{frame}


\input{manipulation_fichier.tex}
\begin{frame}[fragile]{matrice.c (partie 1)}
\vspace{-1em}
\inputminted[
  firstline=1,
  lastline=30,
  breaklines=true,
  fontsize=\tiny,
  frame=lines,
  framesep=1mm
]{c}{../matrice.c}
\end{frame}

\begin{frame}[fragile]{matrice.c (partie 2)}
\vspace{-1em}
\inputminted[
  firstline=31,
  lastline=60,
  breaklines=true,
  fontsize=\tiny,
  frame=lines,
  framesep=1mm
]{c}{../matrice.c}
\end{frame}

\begin{frame}[fragile]{matrice.c (partie 3)}
\vspace{-1em}
\inputminted[
  firstline=61,
  lastline=90,
  breaklines=true,
  fontsize=\tiny,
  frame=lines,
  framesep=1mm
]{c}{../matrice.c}
\end{frame}

\begin{frame}[fragile]{matrice.c (partie 4)}
\vspace{-1em}
\inputminted[
  firstline=91,
  lastline=120,
  breaklines=true,
  fontsize=\tiny,
  frame=lines,
  framesep=1mm
]{c}{../matrice.c}
\end{frame}

\begin{frame}[fragile]{matrice.c (partie 5)}
\vspace{-1em}
\inputminted[
  firstline=121,
  lastline=150,
  breaklines=true,
  fontsize=\tiny,
  frame=lines,
  framesep=1mm
]{c}{../matrice.c}
\end{frame}

\begin{frame}[fragile]{matrice.c (partie 6)}
\vspace{-1em}
\inputminted[
  firstline=151,
  lastline=180,
  breaklines=true,
  fontsize=\tiny,
  frame=lines,
  framesep=1mm
]{c}{../matrice.c}
\end{frame}

\begin{frame}[fragile]{matrice.c (partie 7)}
\vspace{-1em}
\inputminted[
  firstline=181,
  lastline=210,
  breaklines=true,
  fontsize=\tiny,
  frame=lines,
  framesep=1mm
]{c}{../matrice.c}
\end{frame}

\begin{frame}[fragile]{matrice.c (partie 8)}
\vspace{-1em}
\inputminted[
  firstline=211,
  lastline=240,
  breaklines=true,
  fontsize=\tiny,
  frame=lines,
  framesep=1mm
]{c}{../matrice.c}
\end{frame}

\begin{frame}[fragile]{matrice.c (partie 9)}
\vspace{-1em}
\inputminted[
  firstline=241,
  lastline=270,
  breaklines=true,
  fontsize=\tiny,
  frame=lines,
  framesep=1mm
]{c}{../matrice.c}
\end{frame}

\begin{frame}[fragile]{matrice.c (partie 10)}
\vspace{-1em}
\inputminted[
  firstline=271,
  lastline=300,
  breaklines=true,
  fontsize=\tiny,
  frame=lines,
  framesep=1mm
]{c}{../matrice.c}
\end{frame}

\begin{frame}[fragile]{matrice.c (partie 11)}
\vspace{-1em}
\inputminted[
  firstline=301,
  lastline=329,
  breaklines=true,
  fontsize=\tiny,
  frame=lines,
  framesep=1mm
]{c}{../matrice.c}
\end{frame}


\begin{frame}[fragile]{moravec.c (partie 1)}
\vspace{-1em}
\inputminted[
  firstline=1,
  lastline=30,
  breaklines=true,
  fontsize=\tiny,
  frame=lines,
  framesep=1mm
]{c}{../moravec.c}
\end{frame}

\begin{frame}[fragile]{moravec.c (partie 2)}
\vspace{-1em}
\inputminted[
  firstline=31,
  lastline=60,
  breaklines=true,
  fontsize=\tiny,
  frame=lines,
  framesep=1mm
]{c}{../moravec.c}
\end{frame}

\begin{frame}[fragile]{moravec.c (partie 3)}
\vspace{-1em}
\inputminted[
  firstline=61,
  lastline=70,
  breaklines=true,
  fontsize=\tiny,
  frame=lines,
  framesep=1mm
]{c}{../moravec.c}
\end{frame}


\begin{frame}[fragile]{ransac.c (partie 1)}
\vspace{-1em}
\inputminted[
  firstline=1,
  lastline=30,
  breaklines=true,
  fontsize=\tiny,
  frame=lines,
  framesep=1mm
]{c}{../ransac.c}
\end{frame}

\begin{frame}[fragile]{ransac.c (partie 2)}
\vspace{-1em}
\inputminted[
  firstline=31,
  lastline=60,
  breaklines=true,
  fontsize=\tiny,
  frame=lines,
  framesep=1mm
]{c}{../ransac.c}
\end{frame}

\begin{frame}[fragile]{ransac.c (partie 3)}
\vspace{-1em}
\inputminted[
  firstline=61,
  lastline=75,
  breaklines=true,
  fontsize=\tiny,
  frame=lines,
  framesep=1mm
]{c}{../ransac.c}
\end{frame}


\begin{frame}[fragile]{reconstruction.c (partie 1)}
\vspace{-1em}
\inputminted[
  firstline=1,
  lastline=30,
  breaklines=true,
  fontsize=\tiny,
  frame=lines,
  framesep=1mm
]{c}{../reconstruction.c}
\end{frame}

\begin{frame}[fragile]{reconstruction.c (partie 2)}
\vspace{-1em}
\inputminted[
  firstline=31,
  lastline=60,
  breaklines=true,
  fontsize=\tiny,
  frame=lines,
  framesep=1mm
]{c}{../reconstruction.c}
\end{frame}

\begin{frame}[fragile]{reconstruction.c (partie 3)}
\vspace{-1em}
\inputminted[
  firstline=61,
  lastline=90,
  breaklines=true,
  fontsize=\tiny,
  frame=lines,
  framesep=1mm
]{c}{../reconstruction.c}
\end{frame}

\begin{frame}[fragile]{reconstruction.c (partie 4)}
\vspace{-1em}
\inputminted[
  firstline=91,
  lastline=99,
  breaklines=true,
  fontsize=\tiny,
  frame=lines,
  framesep=1mm
]{c}{../reconstruction.c}
\end{frame}


\input{SVD.tex}
\begin{frame}[fragile]{test\_camera\_calibration.c (partie 1)}
\vspace{-1em}
\inputminted[
  firstline=1,
  lastline=30,
  breaklines=true,
  fontsize=\tiny,
  frame=lines,
  framesep=1mm
]{c}{../test_camera_calibration.c}
\end{frame}

\begin{frame}[fragile]{test\_camera\_calibration.c (partie 2)}
\vspace{-1em}
\inputminted[
  firstline=31,
  lastline=57,
  breaklines=true,
  fontsize=\tiny,
  frame=lines,
  framesep=1mm
]{c}{../test_camera_calibration.c}
\end{frame}


\begin{frame}[fragile]{test\_detection.c (partie 1)}
\vspace{-1em}
\inputminted[
  firstline=1,
  lastline=30,
  breaklines=true,
  fontsize=\tiny,
  frame=lines,
  framesep=1mm
]{c}{../test_detection.c}
\end{frame}

\begin{frame}[fragile]{test\_detection.c (partie 2)}
\vspace{-1em}
\inputminted[
  firstline=31,
  lastline=41,
  breaklines=true,
  fontsize=\tiny,
  frame=lines,
  framesep=1mm
]{c}{../test_detection.c}
\end{frame}


\begin{frame}[fragile]{test\_moravec.c (partie 1)}
\vspace{-1em}
\inputminted[
  firstline=1,
  lastline=30,
  breaklines=true,
  fontsize=\tiny,
  frame=lines,
  framesep=1mm
]{c}{../test_moravec.c}
\end{frame}

\begin{frame}[fragile]{test\_moravec.c (partie 2)}
\vspace{-1em}
\inputminted[
  firstline=31,
  lastline=56,
  breaklines=true,
  fontsize=\tiny,
  frame=lines,
  framesep=1mm
]{c}{../test_moravec.c}
\end{frame}


\begin{frame}[fragile]{test\_reconstruction\_mult.c (partie 1)}
\vspace{-1em}
\inputminted[
  firstline=1,
  lastline=27,
  breaklines=true,
  fontsize=\tiny,
  frame=lines,
  framesep=1mm
]{c}{../test_reconstruction_mult.c}
\end{frame}


\begin{frame}[fragile]{test\_triangulation.c (partie 1)}
\vspace{-1em}
\inputminted[
  firstline=1,
  lastline=30,
  breaklines=true,
  fontsize=\tiny,
  frame=lines,
  framesep=1mm
]{c}{../test_triangulation.c}
\end{frame}

\begin{frame}[fragile]{test\_triangulation.c (partie 2)}
\vspace{-1em}
\inputminted[
  firstline=31,
  lastline=49,
  breaklines=true,
  fontsize=\tiny,
  frame=lines,
  framesep=1mm
]{c}{../test_triangulation.c}
\end{frame}


\begin{frame}[fragile]{triangle.c (partie 1)}
\vspace{-1em}
\inputminted[
  firstline=1,
  lastline=30,
  breaklines=true,
  fontsize=\tiny,
  frame=lines,
  framesep=1mm
]{c}{../triangle.c}
\end{frame}

\begin{frame}[fragile]{triangle.c (partie 2)}
\vspace{-1em}
\inputminted[
  firstline=31,
  lastline=60,
  breaklines=true,
  fontsize=\tiny,
  frame=lines,
  framesep=1mm
]{c}{../triangle.c}
\end{frame}

\begin{frame}[fragile]{triangle.c (partie 3)}
\vspace{-1em}
\inputminted[
  firstline=61,
  lastline=90,
  breaklines=true,
  fontsize=\tiny,
  frame=lines,
  framesep=1mm
]{c}{../triangle.c}
\end{frame}

\begin{frame}[fragile]{triangle.c (partie 4)}
\vspace{-1em}
\inputminted[
  firstline=91,
  lastline=120,
  breaklines=true,
  fontsize=\tiny,
  frame=lines,
  framesep=1mm
]{c}{../triangle.c}
\end{frame}

\begin{frame}[fragile]{triangle.c (partie 5)}
\vspace{-1em}
\inputminted[
  firstline=121,
  lastline=150,
  breaklines=true,
  fontsize=\tiny,
  frame=lines,
  framesep=1mm
]{c}{../triangle.c}
\end{frame}

\begin{frame}[fragile]{triangle.c (partie 6)}
\vspace{-1em}
\inputminted[
  firstline=151,
  lastline=180,
  breaklines=true,
  fontsize=\tiny,
  frame=lines,
  framesep=1mm
]{c}{../triangle.c}
\end{frame}

\begin{frame}[fragile]{triangle.c (partie 7)}
\vspace{-1em}
\inputminted[
  firstline=181,
  lastline=210,
  breaklines=true,
  fontsize=\tiny,
  frame=lines,
  framesep=1mm
]{c}{../triangle.c}
\end{frame}

\begin{frame}[fragile]{triangle.c (partie 8)}
\vspace{-1em}
\inputminted[
  firstline=211,
  lastline=240,
  breaklines=true,
  fontsize=\tiny,
  frame=lines,
  framesep=1mm
]{c}{../triangle.c}
\end{frame}

\begin{frame}[fragile]{triangle.c (partie 9)}
\vspace{-1em}
\inputminted[
  firstline=241,
  lastline=246,
  breaklines=true,
  fontsize=\tiny,
  frame=lines,
  framesep=1mm
]{c}{../triangle.c}
\end{frame}


\input{trouve_coin.tex}
\begin{frame}[fragile]{Makefile (partie 1)}
\vspace{-1em}
\inputminted[
  firstline=1,
  lastline=30,
  breaklines=true,
  fontsize=\tiny,
  frame=lines,
  framesep=1mm
]{c}{../Makefile}
\end{frame}

\begin{frame}[fragile]{Makefile (partie 2)}
\vspace{-1em}
\inputminted[
  firstline=31,
  lastline=59,
  breaklines=true,
  fontsize=\tiny,
  frame=lines,
  framesep=1mm
]{c}{../Makefile}
\end{frame}


\begin{frame}[fragile]{jpg\_to\_txt\_color.py (partie 1)}
\vspace{-1em}
\inputminted[
  firstline=1,
  lastline=30,
  breaklines=true,
  fontsize=\tiny,
  frame=lines,
  framesep=1mm
]{c}{../jpg_to_txt_color.py}
\end{frame}

\begin{frame}[fragile]{jpg\_to\_txt\_color.py (partie 2)}
\vspace{-1em}
\inputminted[
  firstline=31,
  lastline=59,
  breaklines=true,
  fontsize=\tiny,
  frame=lines,
  framesep=1mm
]{c}{../jpg_to_txt_color.py}
\end{frame}


\begin{frame}[fragile]{jpg\_to\_txt.py (partie 1)}
\vspace{-1em}
\inputminted[
  firstline=1,
  lastline=30,
  breaklines=true,
  fontsize=\tiny,
  frame=lines,
  framesep=1mm
]{c}{../jpg_to_txt.py}
\end{frame}

\begin{frame}[fragile]{jpg\_to\_txt.py (partie 2)}
\vspace{-1em}
\inputminted[
  firstline=31,
  lastline=36,
  breaklines=true,
  fontsize=\tiny,
  frame=lines,
  framesep=1mm
]{c}{../jpg_to_txt.py}
\end{frame}


\begin{frame}[fragile]{plot\_detect.py (partie 1)}
\vspace{-1em}
\inputminted[
  firstline=1,
  lastline=30,
  breaklines=true,
  fontsize=\tiny,
  frame=lines,
  framesep=1mm
]{c}{../plot_detect.py}
\end{frame}

\begin{frame}[fragile]{plot\_detect.py (partie 2)}
\vspace{-1em}
\inputminted[
  firstline=31,
  lastline=60,
  breaklines=true,
  fontsize=\tiny,
  frame=lines,
  framesep=1mm
]{c}{../plot_detect.py}
\end{frame}

\begin{frame}[fragile]{plot\_detect.py (partie 3)}
\vspace{-1em}
\inputminted[
  firstline=61,
  lastline=73,
  breaklines=true,
  fontsize=\tiny,
  frame=lines,
  framesep=1mm
]{c}{../plot_detect.py}
\end{frame}


\begin{frame}[fragile]{plot\_detect\_un.py (partie 1)}
\vspace{-1em}
\inputminted[
  firstline=1,
  lastline=30,
  breaklines=true,
  fontsize=\tiny,
  frame=lines,
  framesep=1mm
]{c}{../plot_detect_un.py}
\end{frame}

\begin{frame}[fragile]{plot\_detect\_un.py (partie 2)}
\vspace{-1em}
\inputminted[
  firstline=31,
  lastline=58,
  breaklines=true,
  fontsize=\tiny,
  frame=lines,
  framesep=1mm
]{c}{../plot_detect_un.py}
\end{frame}


\begin{frame}[fragile]{plot\_points\_3D.py (partie 1)}
\vspace{-1em}
\inputminted[
  firstline=1,
  lastline=30,
  breaklines=true,
  fontsize=\tiny,
  frame=lines,
  framesep=1mm
]{c}{../plot_points_3D.py}
\end{frame}

\begin{frame}[fragile]{plot\_points\_3D.py (partie 2)}
\vspace{-1em}
\inputminted[
  firstline=31,
  lastline=60,
  breaklines=true,
  fontsize=\tiny,
  frame=lines,
  framesep=1mm
]{c}{../plot_points_3D.py}
\end{frame}

\begin{frame}[fragile]{plot\_points\_3D.py (partie 3)}
\vspace{-1em}
\inputminted[
  firstline=61,
  lastline=71,
  breaklines=true,
  fontsize=\tiny,
  frame=lines,
  framesep=1mm
]{c}{../plot_points_3D.py}
\end{frame}


\begin{frame}[fragile]{plot\_points\_ap.py (partie 1)}
\vspace{-1em}
\inputminted[
  firstline=1,
  lastline=30,
  breaklines=true,
  fontsize=\tiny,
  frame=lines,
  framesep=1mm
]{c}{../plot_points_ap.py}
\end{frame}

\begin{frame}[fragile]{plot\_points\_ap.py (partie 2)}
\vspace{-1em}
\inputminted[
  firstline=31,
  lastline=60,
  breaklines=true,
  fontsize=\tiny,
  frame=lines,
  framesep=1mm
]{c}{../plot_points_ap.py}
\end{frame}

\begin{frame}[fragile]{plot\_points\_ap.py (partie 3)}
\vspace{-1em}
\inputminted[
  firstline=61,
  lastline=80,
  breaklines=true,
  fontsize=\tiny,
  frame=lines,
  framesep=1mm
]{c}{../plot_points_ap.py}
\end{frame}


\begin{frame}[fragile]{select\_deux.py (partie 1)}
\vspace{-1em}
\inputminted[
  firstline=1,
  lastline=30,
  breaklines=true,
  fontsize=\tiny,
  frame=lines,
  framesep=1mm
]{c}{../select_deux.py}
\end{frame}

\begin{frame}[fragile]{select\_deux.py (partie 2)}
\vspace{-1em}
\inputminted[
  firstline=31,
  lastline=60,
  breaklines=true,
  fontsize=\tiny,
  frame=lines,
  framesep=1mm
]{c}{../select_deux.py}
\end{frame}

\begin{frame}[fragile]{select\_deux.py (partie 3)}
\vspace{-1em}
\inputminted[
  firstline=61,
  lastline=85,
  breaklines=true,
  fontsize=\tiny,
  frame=lines,
  framesep=1mm
]{c}{../select_deux.py}
\end{frame}


\begin{frame}[fragile]{select\_un.py (partie 1)}
\vspace{-1em}
\inputminted[
  firstline=1,
  lastline=30,
  breaklines=true,
  fontsize=\tiny,
  frame=lines,
  framesep=1mm
]{c}{../select_un.py}
\end{frame}

\begin{frame}[fragile]{select\_un.py (partie 2)}
\vspace{-1em}
\inputminted[
  firstline=31,
  lastline=59,
  breaklines=true,
  fontsize=\tiny,
  frame=lines,
  framesep=1mm
]{c}{../select_un.py}
\end{frame}


\end{document}
