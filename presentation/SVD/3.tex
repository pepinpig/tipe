% \textbf{Amélioration d'implémentation pour QR}
% \begin{itemize}
%   \item \textbf{Objectif} : obtenir la SVD d'une matrice \( A \in \mathbb{R}^{m \times n} \)
%   \pause
%   \item \textbf{Étape clé} : décomposition QR de \( A^T A \)
%   \pause
%   \item \textbf{Deux méthodes testées pour la QR} :
%   \begin{itemize}
%     \item \textbf{Gram-Schmidt} classique \\
%     \hspace{1em}  \xmark sensible aux erreurs d'arrondi \\
%     \hspace{1em}  \xmark instable numériquement
%     \pause
%     \item \textbf{Réflexions de Householder} \\
%     \hspace{1em}  \cmark plus stable numériquement \\
%     \hspace{1em}  \cmark meilleure orthogonalité garantie
%   \end{itemize}
%   \pause
%   \item \textbf{Conclusion} : Householder utilisé dans la version finale
%   \pause
%   \item Autres choix :
%   \begin{itemize}
%     \item Seuil de convergence \( \varepsilon = 10^{-12} \)
%     \item Maximum d’itérations fixé à 1000
%     \item Normalisation explicite de chaque colonne de \( U \)
%   \end{itemize}
% \end{itemize}
\textbf{Détail : Décomposition QR}

\vspace{0.5em}

\begin{itemize}
  \item Objectif : \( A^T A = Q R \)
  \item Deux méthodes testées :
  \begin{itemize}
    \item Gram-Schmidt classique → \xmark instable
    \item Réflexions de Householder → \cmark retenu
  \end{itemize}
  \item Propriétés : \( Q \) orthogonale, \( R \) triangulaire
  \item Utilisé dans l'algorithme QR pour les valeurs propres
\end{itemize}