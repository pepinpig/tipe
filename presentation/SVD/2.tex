
\begin{frame}{De la décomposition QR à la SVD}
\begin{itemize}
  \item \textbf{Décomposition QR} : toute matrice \( A \in \mathbb{R}^{n \times n} \) peut s’écrire :
  \[
    A = QR
  \]
  où :
  \begin{itemize}
    \pause
    \item \( Q \) est orthogonale \pause
    \item \( R \) est triangulaire supérieure
  \end{itemize}

  \pause
  \item \textbf{Algorithme QR} : utilisé pour diagonaliser \( A \)
  \begin{enumerate}
    \item Appliquer \( A = Q_0 R_0 \) \pause
    \item Reformer : \( A_1 = R_0 Q_0 \) \pause
    \item Répéter : \( A_{k+1} = R_k Q_k \) \pause
    \item Les \( A_k \) convergent vers une matrice diagonale si \( A \) est symétrique
  \end{enumerate}

  \pause
  \item \textbf{Vers la SVD} :
  \begin{itemize}
    \item On applique l'algorithme QR à \( A^T A \) \pause
    \item Les valeurs propres \( \lambda_i \) donnent les \( \sigma_i^2 \) \pause
    \item Les vecteurs propres forment \( V \), et \( U = \frac{1}{\sigma_i} A v_i \)
  \end{itemize}
\end{itemize}
\end{frame}