%%%%%%%%%%%%%%%%%%%%%%%%%%%%%%%%%%%%%%%%%%%%%%%%%
\section[Première section]{Selection et appariement}
%------------------------------------------------
\begin{frame}
\frametitle{Titre d'une slide avant la sous-section\esp}
	Ici on n'a pas encore de titre de sous-section dans le badeau du haut.
	\label{slides_hors_subsec}
\end{frame}


%++++++++++++++++++++++++++++++++++++++++++++++++
\subsection{À propos de l'en-tête}
%++++++++++++++++++++++++++++++++++++++++++++++++
\begin{frame}
\frametitle{Titre d'une slide dans la sous-section\esp}
	Ici on a un titre de sous-section, contrairement à la slide~\ref{slides_hors_subsec}\\[2cm]
	Regarder le code ici pour référencer une slide avec \lin{\label} et la citer avec son numéro grâce à \lin{\ref}
\end{frame}


%------------------------------------------------
\begin{frame}
\frametitle{Ce qui apparaît dans l'en-tête}
	Dans la \important{première ligne}:
	\begin{itemize}
	\flch la version courte du titre, précisée en option  de  \lin{\title}\\
	\remarque{( en option = entre crochets, avant les accolades)}
	\flch la version courte du nom, voire des initiales, redéfinir la commande
	\lin{\newcommand{\initiales}{Petit Nom}} 
	\flch la version courte de la date, précisée en option de \lin{\date}\\
	\end{itemize}
	\bigskip
	Dans la \important{deuxième ligne}:
	\begin{itemize}
	\flch le numéro et le titre de la section, sauf si le numéro est nul\\
	\remarque{s'il est précisé en option de} \lin{\section}
	\remarque{le titre court est utilisé}
	\flch le numéro et le titre de la sous-section, sauf si le numéro est nul\\
	\remarque{s'il est précisé en option de} \lin{\subsection}
	\remarque{le titre court est utilisé}
\end{itemize}

\end{frame}


%++++++++++++++++++++++++++++++++++++++++++++++++
\subsection{Selection des points d'intérêts}
%------------------------------------------------
\begin{frame}
\frametitle{Titre de la slide sans lettre descendant sous la baseline}
	Pour régler ce problème, utiliser la commande \lin{\esp} à la fin du titre, \textit{Cf.} slide suivante
\end{frame}


%------------------------------------------------
\begin{frame}
\frametitle{Titre de la slide sans lettre descendant sous la baseline\esp}
	Ici c'est mieux non?
\end{frame}


%------------------------------------------------
\begin{frame}[fragile]
\frametitle{Titre de la slide qui marche tout seul grâce au q et au g}
\end{frame}


%++++++++++++++++++++++++++++++++++++++++++++++++
\subsection{Appariement}
%------------------------------------------------
\begin{frame}
\frametitle{Titre de la slide sans lettre descendant sous la baseline}
	Pour régler ce problème, utiliser la commande \lin{\esp} à la fin du titre, \textit{Cf.} slide suivante
\end{frame}


%------------------------------------------------
\begin{frame}
\frametitle{Titre de la slide sans lettre descendant sous la baseline\esp}
	Ici c'est mieux non?
\end{frame}


%------------------------------------------------
\begin{frame}[fragile]
\frametitle{Titre de la slide qui marche tout seul grâce au q et au g}
\end{frame}
