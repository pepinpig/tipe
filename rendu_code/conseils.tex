\vspace*{0.4cm}
\noindent
Ceci est un modèle pour générer un PDF qui présente le code d'un TIPE.\\
Le fichier principal \texttt{rendu\_code.tex} est à compiler avec l'option \texttt{-shell-escape}.\\

\noindent
Conseils: 
\begin{itemize}
	\item[-] Commencez par faire du ménage dans votre code, supprimer les fichiers obsolètes, les brouillons... Si ce n'est pas fait, c'est peut-être aussi l'occasion de séparer son code en plusieurs fichiers...
	\\
	\item[-] Placez le dossier avec ce code latex à proximité de votre code dossier de code (pas dedans) et indiquez les fichiers de code à inclure grâce à un chemin relatif.
	Typiquement si votre dossier \texttt{TIPE} contient les dossiers \texttt{Code} et  \texttt{rendu\_code}, vous écrirez \texttt{../code/sous\_dossier/fichier.c}.\\
	Il est déconseillé de copier tout son code dans un dossier spécial pour générer le rendu de code, cela n'est pas très robuste aux modifications et ajouts de code ultérieurs.
	\\
	\item[-] 	Si vous importez le code du fichier en entier, pensez à supprimer les lignes vides en fin de fichier, les portions de code commenté devenu inutile...
	\\
	\item[-] 	Si au contraire vous utilisez l'import par blocs, il est conseillé de sauter des lignes entre les différentes fonctions de sorte que chaque bloc commence à une ligne multiple de 10 (par exemple).
	Cela permet, dans le cas où vous modifiez a posteriori une fonction, en ajoutant une ligne de commentaire par exemple, d'avoir à modifier seulement les numéros de lignes du bloc correspondant, et pas ceux de tous les blocs suivants.\\
	\item[-] Pensez à ajuster le modèle pour une version finale :
	\begin{itemize}
	\item[$\cdot$] renseignez nom, prénom et numéro de candidat dans le fichier \texttt{mise\_en\_forme\_MPI.tex}, pour la première page l.6 ET pour le pied de page l.32 ;
	\item[$\cdot$] adaptez le titre et la date ;
	\item[$\cdot$] supprimez cette section de conseils en commentant la l.20 ;
	\item[$\cdot$] à la fin il ne doit plus y avoir de rouge dans le PDF.\\
	\end{itemize}
	\item[-] Compilez plusieurs fois si nécessaire pour ajuster le nombre total de pages en pied de page.
	\\
	\item[-] Si vous ne réduisez pas la police, vous pouvez imprimer ce PDF livret (A4-> A5, en 2 pages par feuille...), ça reste lisible.
\end{itemize}
